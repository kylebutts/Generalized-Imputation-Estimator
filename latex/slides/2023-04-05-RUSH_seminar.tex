\documentclass[aspectratio=43,t]{beamer}
% \documentclass[aspectratio=169]{beamer}

% Title --------------------------------------------
\title{Generalized Imputation Estimators for Factor Models}
\date{\today}
\author{
Nicholas Brown and Kyle Butts
}

% xcolor and define colors -------------------------
\usepackage{xcolor}

% https://www.viget.com/articles/color-contrast/
\definecolor{purple}{HTML}{695693}
\definecolor{navy}{HTML}{567293}
\definecolor{ruby}{HTML}{9a2515}
\definecolor{alice}{HTML}{107895}
\definecolor{daisy}{HTML}{EBC944}
\definecolor{coral}{HTML}{F26D21}
\definecolor{kelly}{HTML}{829356}
\definecolor{cranberry}{HTML}{E64173}
\definecolor{jet}{HTML}{131516}
\definecolor{asher}{HTML}{555F61}
\definecolor{slate}{HTML}{314F4F}

% Main theme colors
\definecolor{accent}{HTML}{107895}
\definecolor{accent2}{HTML}{E64173}

\newcommand\navy[1]{{\color{navy}#1}}
\newcommand\purple[1]{{\color{purple}#1}}
\newcommand\kelly[1]{{\color{kelly}#1}}
\newcommand\ruby[1]{{\color{ruby}#1}}
\newcommand\alice[1]{{\color{alice}#1}}
\newcommand\daisy[1]{{\color{daisy}#1}}
\newcommand\coral[1]{{\color{coral}#1}}
\newcommand\cranberry[1]{{\color{cranberry}#1}}
\newcommand\slate[1]{{\color{slate}#1}}
\newcommand\jet[1]{{\color{jet}#1}}
\newcommand\asher[1]{{\color{asher}#1}}

\newcommand\bgNavy[1]{{\colorbox{navy!80!white}{#1}}}
\newcommand\bgPurple[1]{{\colorbox{purple!80!white}{#1}}}
\newcommand\bgKelly[1]{{\colorbox{kelly!80!white}{#1}}}
\newcommand\bgRuby[1]{{\colorbox{ruby!80!white}{#1}}}
\newcommand\bgAlice[1]{{\colorbox{alice!80!white}{#1}}}
\newcommand\bgDaisy[1]{{\colorbox{daisy!80!white}{#1}}}
\newcommand\bgCoral[1]{{\colorbox{coral!80!white}{#1}}}
\newcommand\bgCranberry[1]{{\colorbox{cranberry!80!white}{#1}}}


% Beamer Options -------------------------------------

% Background
\setbeamercolor{background canvas}{bg = white}

% Change text margins
\setbeamersize{text margin left = 15pt, text margin right = 15pt} 

% \alert
\setbeamercolor{alerted text}{fg = accent2}

% Frame title
\setbeamercolor{frametitle}{bg = white, fg = jet}
\setbeamercolor{framesubtitle}{bg = white, fg = accent}
\setbeamerfont{framesubtitle}{size = \small, shape = \itshape}

% Block
\setbeamercolor{block title}{fg = white, bg = accent2}
\setbeamercolor{block body}{fg = jet, bg = jet!10!white}

% Title page
\setbeamercolor{title}{fg = jet}
\setbeamercolor{subtitle}{fg = accent}

%% Custom \maketitle and \titlepage
\setbeamertemplate{title page}
{
    %\begin{centering}
        \vspace{20mm}
        {\Large \usebeamerfont{title}\usebeamercolor[fg]{title}\inserttitle}\\ \vskip0.25em%
        \ifx\insertsubtitle\@empty%
        \else%
          {\usebeamerfont{subtitle}\usebeamercolor[fg]{subtitle}\insertsubtitle\par}%
        \fi% 
        {\vspace{10mm}\insertauthor}\\
        {\color{asher}\small{\insertdate}}\\
    %\end{centering}
}

% Table of Contents
\setbeamercolor{section in toc}{fg = accent!70!jet}
\setbeamercolor{subsection in toc}{fg = jet}

% Button 
\setbeamercolor{button}{bg = accent}

% Remove navigation symbols
\setbeamertemplate{navigation symbols}{}

% Table and Figure captions
\setbeamercolor{caption}{fg=jet!70!white}
\setbeamercolor{caption name}{fg=jet}
\setbeamerfont{caption name}{shape = \itshape}

% Bullet points

%% Fix left-margins
\settowidth{\leftmargini}{\usebeamertemplate{itemize item}}
\addtolength{\leftmargini}{\labelsep}

%% enumerate item color
\setbeamercolor{enumerate item}{fg = accent}
\setbeamerfont{enumerate item}{size = \small}
\setbeamertemplate{enumerate item}{\insertenumlabel.}

%% itemize
\setbeamercolor{itemize item}{fg = accent!70!white}
\setbeamerfont{itemize item}{size = \small}
\setbeamertemplate{itemize item}[circle]

%% right arrow for subitems
\setbeamercolor{itemize subitem}{fg = accent!60!white}
\setbeamerfont{itemize subitem}{size = \small}
\setbeamertemplate{itemize subitem}{$\rightarrow$}

\setbeamertemplate{itemize subsubitem}[square]
\setbeamercolor{itemize subsubitem}{fg = jet}
\setbeamerfont{itemize subsubitem}{size = \small}

% References

%% Bibliography Font, roughly matching aea
\setbeamerfont{bibliography item}{size = \footnotesize}
\setbeamerfont{bibliography entry author}{size = \footnotesize, series = \bfseries}
\setbeamerfont{bibliography entry title}{size = \footnotesize}
\setbeamerfont{bibliography entry location}{size = \footnotesize, shape = \itshape}
\setbeamerfont{bibliography entry note}{size = \footnotesize}

\setbeamercolor{bibliography item}{fg = jet}
\setbeamercolor{bibliography entry author}{fg = accent!60!jet}
\setbeamercolor{bibliography entry title}{fg = jet}
\setbeamercolor{bibliography entry location}{fg = jet}
\setbeamercolor{bibliography entry note}{fg = jet}

%% Remove bibliography symbol in slides
\setbeamertemplate{bibliography item}{}





% Links ----------------------------------------------

\usepackage{hyperref}
\hypersetup{
  colorlinks = true,
  linkcolor = accent2,
  filecolor = accent2,
  urlcolor = accent2,
  citecolor = accent2,
}


% Line spacing --------------------------------------
\usepackage{setspace}
\setstretch{1.3}


% \begin{columns} -----------------------------------
\usepackage{multicol}


% Fonts ---------------------------------------------
% Beamer Option to use custom fonts
\usefonttheme{professionalfonts}

% \usepackage[utopia, smallerops, varg]{newtxmath}
% \usepackage{utopia}
\usepackage[sfdefault,light]{roboto}

% Small adjustments to text kerning
\usepackage{microtype}



% Remove annoying over-full box warnings -----------
\vfuzz2pt 
\hfuzz2pt


% Table of Contents with Sections
\setbeamerfont{myTOC}{series=\bfseries, size=\Large}
\AtBeginSection[]{
        \frame{
            \frametitle{Roadmap}
            \tableofcontents[current]   
        }
    }


% References ----------------------------------------
\usepackage[
    citestyle= authoryear,
    style = authoryear,
    natbib = true, 
    backend = biber
]{biblatex}

% Smaller font-size for references
\renewcommand*{\bibfont}{\small}

% Remove "In:"
\renewbibmacro{in:}{}

% Color citations for slides
\newenvironment{citecolor}
    {\footnotesize\begin{color}{accent2}}
    {\end{color}}

\newcommand{\citetcolor}[1]{{\footnotesize\textcolor{gray}{\citet{#1}}}}
\newcommand{\citepcolor}[1]{{\footnotesize\textcolor{gray}{\citep{#1}}}}

% Tables -------------------------------------------
% Tables too big
% \begin{adjustbox}{width = 1.2\textwidth, center}
\usepackage{adjustbox}
\usepackage{array}
\usepackage{threeparttable, booktabs, adjustbox}
    
% Fix \input with tables
% \input fails when \\ is at end of external .tex file

\makeatletter
\let\input\@@input
\makeatother

% Tables too narrow
% \begin{tabularx}{\linewidth}{cols}
% col-types: X - center, L - left, R -right
% Relative scale: >{\hsize=.8\hsize}X/L/R
\usepackage{tabularx}
\newcolumntype{L}{>{\raggedright\arraybackslash}X}
\newcolumntype{R}{>{\raggedleft\arraybackslash}X}
\newcolumntype{C}{>{\centering\arraybackslash}X}

% Figures

% \imageframe{img_name} -----------------------------
% from https://github.com/mattjetwell/cousteau
\newcommand{\imageframe}[1]{%
    \begin{frame}[plain]
        \begin{tikzpicture}[remember picture, overlay]
            \node[at = (current page.center), xshift = 0cm] (cover) {%
                \includegraphics[keepaspectratio, width=\paperwidth, height=\paperheight]{#1}
            };
        \end{tikzpicture}
    \end{frame}%
}

% subfigures
\usepackage{subfigure}




% Highlight slide -----------------------------------
% \begin{transitionframe} Text \end{transitionframe}
% from paulgp's beamer tips
\newenvironment{transitionframe}{
    \setbeamercolor{background canvas}{bg=accent!60!black}
    \begin{frame}\color{accent!10!white}\LARGE\centering
}{
    \end{frame}
}


% Table Highlighting --------------------------------
% Create top-left and bottom-right markets in tabular cells with a unique matching id and these commands will outline those cells

\usepackage[beamer,customcolors]{hf-tikz}
\usetikzlibrary{calc}
\usetikzlibrary{fit,shapes.misc}

\usepackage{pgfplots}
\pgfplotsset{compat=newest}
\usepgfplotslibrary{groupplots}
\usepgfplotslibrary{polar}
\usepgfplotslibrary{smithchart}
\usepgfplotslibrary{statistics}
\usepgfplotslibrary{dateplot}
\usepgfplotslibrary{ternary}
\usetikzlibrary{arrows.meta}
\usetikzlibrary{backgrounds}
\usepgfplotslibrary{patchplots}
\usepgfplotslibrary{fillbetween}
\pgfplotsset{%
    layers/standard/.define layer set={%
        background,axis background,axis grid,axis ticks,axis lines,axis tick labels,pre main,main,axis descriptions,axis foreground%
    }{
        grid style={/pgfplots/on layer=axis grid},%
        tick style={/pgfplots/on layer=axis ticks},%
        axis line style={/pgfplots/on layer=axis lines},%
        label style={/pgfplots/on layer=axis descriptions},%
        legend style={/pgfplots/on layer=axis descriptions},%
        title style={/pgfplots/on layer=axis descriptions},%
        colorbar style={/pgfplots/on layer=axis descriptions},%
        ticklabel style={/pgfplots/on layer=axis tick labels},%
        axis background@ style={/pgfplots/on layer=axis background},%
        3d box foreground style={/pgfplots/on layer=axis foreground},%
    },
}

% To set the hypothesis highlighting boxes red.
\newcommand\marktopleft[1]{%
    \tikz[overlay,remember picture] 
        \node (marker-#1-a) at (0,1.5ex) {};%
}
\newcommand\markbottomright[1]{%
    \tikz[overlay,remember picture] 
        \node (marker-#1-b) at (0,0) {};%
    \tikz[accent!80!jet, ultra thick, overlay, remember picture, inner sep=4pt]
        \node[draw, rectangle, fit=(marker-#1-a.center) (marker-#1-b.center)] {};%
}


% Custom Math Definitions ------------------------------------------------------

\global\long\def\expec#1{\mathbb{E}\left[#1\right]}%
\newcommand{\condexpec}[2]{\mathbb{E}\left[#1 \ \vert \ #2\right]}
\global\long\def\prob#1{\mathbb{P}\left[#1\right]}%
\global\long\def\var#1{\mathrm{Var}\left[#1\right]}%
\global\long\def\cov#1{\mathrm{Cov}\left[#1\right]}%
\global\long\def\one{\mathbf{1}}%

% \graphicspath{../../figures/}

% Set-up Bibliography ------------------------------
\addbibresource{references.bib}
\usepackage{bm}
\usepackage{adjustbox}

\begin{document}

% ------------------------------------------------------------------------------
\begin{frame}
  \maketitle

  % \vspace{2.5mm}
  % {}
\end{frame}
% ------------------------------------------------------------------------------

\begin{frame}{Introduction}
  We are interested in effects of an intervention.

  \vspace{.5cm}

  \textbf{Notation:}

  Observed outcomes have two potential states:
  \begin{itemize}
    \item Treated potential outcomes $y_{it}(1)$.
    \item Untreated potential outcomes $y_{it}(0)$.
  \end{itemize}


  \bigskip
  The \textbf{treatment effect}, at time $t$ for unit $i$ is
  \begin{equation}
    \tau_{it} = y_{it}(1) - \coral{y_{it}(0)},
  \end{equation}
  where $\coral{y_{it}(0)}$ is unobserved.
\end{frame}

\begin{frame}{TWFE and Parallel Trends}
  In panel settings, researchers often assume that a parallel-trends type assumption holds for outcomes
  $$
    \coral{y_{it}(0)} = \mu_i + \lambda_t + u_{it},
  $$
  with $\condexpec{u_{it}}{D_i = 1} = \condexpec{u_{it}}{D_i = 0} = 0$ for all $t$.

  \bigskip
  Under this assumption, treated and control units are (on average) on the same trend: $\lambda_t$

  \pause
  \bigskip
  This is often undesirable, units select into treatment based on their economic trends all the time!
\end{frame}

\begin{frame}{Example:}
  Walmart choses where to open new stores in the 90s.
  \begin{itemize}
    \item Interested on the labor market impacts
  \end{itemize}

  \bigskip \pause

  \textbf{Untreated Model 1:} $\text{employment}_{it}(0) = \text{macro}_t + \text{county}_i + u_{it}$.
  \begin{itemize}
    \item We would need to assume that treated counties and control counties are equally exposed to macroeconomic trends
  \end{itemize}


\end{frame}

\begin{frame}{}
  \begin{figure}
    \caption{TWFE Estimatied Effects of Walmart Entry on $\log$ Employment}

    \input{../../figures/did2s_retail_bootstrap_1000_slides.tex}
  \end{figure}
\end{frame}

\begin{frame}{Example:}
  Walmart choses where to open new stores in the 90s.
  \begin{itemize}
    \item Interested on the labor market impacts
  \end{itemize}

  \bigskip
  \textbf{Untreated Model 2:} $\text{employment}_{it}(0) = \text{macro}_t * \text{county}_i + u_{it}$.
  \begin{itemize}
    \item Under this model, we can allow treated counties to have differential exposure to the macro shocks (e.g. manufacturing share)
  \end{itemize}
\end{frame}

\begin{frame}{Intuition of Factor Model}
  The intuition is very similar to that of a shift-share variable:
  $$
    z_{it} = \gamma_i f_t
  $$
  \vspace*{-5mm}
  \begin{itemize}
    \item $f_t$ is the set of \emph{`macroeconomic'} shocks (shifts) that all units experience
    \item $\gamma_i$ is an individuals \emph{exposure} to the shocks (shares)
  \end{itemize}

  \bigskip
  The difference being that \textbf{we do not observe} the variables $\gamma_i$ and $f_t$ (like we don't observe fixed effects)
\end{frame}

\section{Theory}

\begin{frame}{Model}
  $N$ individuals observed for $T$ times periods.
  \begin{itemize}
    \item Treatment begins \textbf{after} period $T_0$ (ignoring staggered treatment timing for this presentation).
    \item $N_1$ treated individuals, $N_0$ untreated individuals.
  \end{itemize}

  \pause
  \bigskip
  Untreated potential outcomes are given by a \textbf{factor model}:
  \begin{equation}
    y_{it}(0) = \mu_i + \lambda_t + \bm f_t' \bm \gamma_i + u_{it}
  \end{equation}

  \vspace{-3mm}
  \begin{itemize}
    \item $\bm f_t$: $p \times 1$ vector of unobserved common factors.
    \item $\bm \gamma_i$: $p \times 1$ vector of unobserved individual factor loadings.
    \item Nests the common TWFE model ($\bm \gamma_i = 0$).
  \end{itemize}
\end{frame}

\begin{frame}{Assumptions}
  \textbf{\color{alice} Assumption:} {\color{asher} Arbitrary Treatment Effects}
  \begin{equation}
    y_{it}(1) = y_{it}(0) + \tau_{it}
  \end{equation}


  \bigskip\pause
  \textbf{\color{alice} Assumption:} {\color{asher} No Anticipation}
  $$
    y_{it}(0) = y_{it} \text{ when } d_{it} = 0
  $$

  \begin{itemize}
    \item Treated units do not change their behavior before treatment.
    \item Can estimate and test for limited anticipation effects in our framework.
  \end{itemize}
\end{frame}

\begin{frame}{Assumptions}{`Non-Parallel Trends'}
  \textbf{\color{alice} Assumption:} {\color{asher} Selection into Treatment}
  $$
    y_{it}(0) = \mu_i + \lambda_t + \bm f_t' \bm \gamma_i + u_{it},
  $$
  with
  $$
    \condexpec{u_{it}}{\mu_i, \bm \gamma_i, D_i} = 0
  $$

  \bigskip
  \begin{itemize}
    \item Relaxes parallel trends by allowing units to enter treatment based on exposure to macroeconomic shocks
    
    \item Does not let units enter into treatment based on unit specific shocks $u_{it}$ 
  \end{itemize}
\end{frame}

\begin{frame}{Selection into Treatment and Parallel Trends}
  In the two period case ($t = 1, 0$) consider the difference-in-differences estimand with parallel trends on the error term:
  \begin{align*}
      & \mathbb{E}_{i} \left[ y_{i1}(1) - y_{i0}(0) \ \vert \ D_i = 1 \right] - \mathbb{E}_{i} \left[ y_{i1}(1) - y_{i0}(0) \ \vert \ D_i = 0 \right] \\ 
      \only<2> {
        &= \mathbb{E}_{i} \left[ \tau_{i1} \ \vert \ D_i = 1 \right] + \bm f_t' \big( \mathbb{E}_{i} \left[ \bm{\gamma}_i \ \vert \ D_i = 1 \right] - \mathbb{E}_{i} \left[ \bm{\gamma}_i \ \vert \ D_i = 0 \right] \big)
      }
  \end{align*}

  \bigskip
  \only<2>{This last term makes parallel trends not hold. That is, differential exposure to macroeconomic shocks violates parallel trends!}
\end{frame}

\begin{frame}{Identification under a factor model}
  There are many estimators for treatment effects under factor models:
  \begin{enumerate}
    \item Synthetic control \begin{citecolor}\citep{Abadie_2021}\end{citecolor}
    \item Matrix Completion \begin{citecolor}\citep{Athey_et_al_2021}\end{citecolor}
    \item Imputation Estimators \begin{citecolor}\citep{Gobillon_Magnac_2016, Xu_2017}\end{citecolor}
  \end{enumerate}

  \bigskip
  \textbf{None of these are valid in short-$T$ settings.} Our paper introduces a general method that is valid in short-$T$ settings. 

  \begin{itemize}
    \item Unlocks a large econometric literature on factor model estimation and incorporates it into causal inference methods
  \end{itemize}
\end{frame}

\begin{frame}{ATT Identification}
  $$
    \text{ATT}_t \equiv \mathbb{E}_{i} \left[ y_{it}(1) \ \vert \ D_i = 1 \right] - \left[ \coral{y_{it}(0)} \ \vert \ D_i = 1 \right]
  $$
  
  \bigskip
  For a given $t$, the average outcome for the treated sample:
  
  \begin{equation}
    \mathbb{E}_{i} \left[ \coral{y_{it}(0)} \ \vert \ D_i = 1 \right] = \lambda_t + \mathbb{E}_{i} \left[ \mu_i \ \vert \ D_i = 1 \right] + \bm f_t' \mathbb{E}_{i} \left[ \bm{\gamma}_i \ \vert \ D_i = 1 \right]
  \end{equation}
  \begin{itemize}
    \item \textbf{Insight:} Do not need to know each $\bm \gamma_i$ which would require large-$T$
    \begin{itemize}
      \item Only need to estimate $\left[ \bm{\gamma}_i \ \vert \ D_i = 1 \right]$.
    \end{itemize}
  \end{itemize}

\end{frame}

\begin{frame}{ATT Identification}
  For now, ignore the additive fixed effects, so that 
  $$
    y_{it}(0) = \bm{f}_t' \bm{\gamma}_i + u_{it}
  $$

  \begin{itemize}
    \item Later, we remove the fixed effects with a within-transformation on $y$
  \end{itemize}
\end{frame}


\begin{frame}{ATT Identification}
  Let `$\text{pre}$' denote the time periods before treatment $t \leq T_0$. If we observed the factors, $\bm{F}$, then for $t > T_0$,

  \begin{equation}
    \text{ATT}_t = \mathbb{E}_i \left[ y_{it} - f_t (\bm{F}_{\text{pre}}' \bm{F}_{\text{pre}})^{-1} \bm{F}_{\text{pre}}' \bm y_{i, \text{pre}}) \ \vert \ D_i = 1 \right]
  \end{equation}

  \smallskip\pause
  What's happening here:
  $$
    f_t \underbrace{(\bm{F}_{\text{pre}}' \bm{F}_{\text{pre}})^{-1} \bm{F}_{\text{pre}}' \bm y_{i, \text{pre}})}_{\to^p \ \mathbb{E}_{i} \left[ \bm{\gamma}_i \ \vert \ D_i = 1 \right]} \to^p 
  $$

  \pause
  This is a general imputation procedure that only requires $\sqrt{n}$-consistent estimation of the factors $\bm{F}$. This brings in a large literature on factor model estimation to causal-inference methods.
\end{frame}

\begin{frame}{Removing additive effects}
  We consider the residuals after within-transforming
  $$
    \tilde{y}_{it} = y_{it} - \overline{y}_{0,t} - \overline{y}_{i,pre} + \overline{y}_{0,pre}
  $$

  \begin{itemize}
    \item $\overline{y}_{0,t}$: never-treated cross-sectional averages.
    \item $\overline{y}_{i,pre}$: pre-treated time averages.
    \item $\overline{y}_{0,pre}$: overall never-treated pre-treated average.
  \end{itemize}

  % \hyperlink{within_transformation}{\beamergotobutton{Details}}
\end{frame}

\begin{frame}{Removing additive effects}
  $$
    \tilde{y}_{it} = y_{it} - \overline{y}_{0,t} - \overline{y}_{i,pre} + \overline{y}_{0,pre}
  $$

  After performing our transformation, we have:
  $$
    \condexpec{\tilde{y}_{it}}{D_i = 1} = \condexpec{d_{it} \tau_{it} + \tilde{\bm f}_t' \tilde{\bm \gamma}_i }{D_i = 1}
  $$
  where $\tilde{\bm f}_t$ are the pre-treatment demeaned factors and $\tilde{\bm \gamma}_i$ are the never-treated demeaned loadings.

  \begin{itemize}
    \item Our transformation removes $(\mu_i, \lambda_t)$ but preserves a common factor structure.
  \end{itemize}
\end{frame}

\begin{frame}{When is TWFE sufficient?}
  If $\condexpec{\bm \gamma_i}{D_i} = \expec{\bm \gamma_i}$, the ATTs are identified by the modified TWFE transformation.

  \begin{equation}
    \condexpec{\tilde{y}_{it}}{D_i = 1} = \condexpec{\tau_{it}}{D_i = 1} = \tau_t
  \end{equation}
  for $t > T_0$.

  \pause
  \begin{itemize}
    \item Says TWFE is sufficient even if there are factors, so long as exposure to these factors are the same between treated and control group.
    \item In the paper, we provide tests for this condition.
  \end{itemize}
\end{frame}

\begin{frame}{Factor Identification}
  We consider instrumental-variables based identification of \citet{Ahn_Lee_Schmidt_2013}.
  \begin{itemize}
    \item Allows fixed-$T$ identification of $\bm{F}$.
    \item A GMM estimator $\implies$ inference is standard
  \end{itemize}

\end{frame}

\begin{frame}{Factor Identification}
  Intuitively, we need a set of instruments that we think:

  \begin{enumerate}
    \item Are correlated with the factor-loadings $\gamma_i$.

    \item Satisfy an exclusion restriction on $u_{it}$. We can't pick up on $(i,t)$ shocks that are correlated with treatment
  \end{enumerate}

  \bigskip
  We think the best IV strategy would entail using time-invariant characteristics that we think are correlated with $\gamma_i$ as instruments.
\end{frame}

\section{Empirical Example}

\begin{frame}{Example}
  For example, consider our Walmart example. Since Walmart is likely targeting growing economies, we think that parallel trends would fail.

  \begin{enumerate}
    \item Plausibly, Walmart is not targetting a specific location based on local shocks, that is based on $u_{it}$.

    \item More so, targetting places that are doing well due to national economic conditions, that is based on $\bm{f}_t \bm{\gamma}_i$.
  \end{enumerate}
\end{frame}

\begin{frame}{Data}
  We construct a dataset following the description in \citet{basker2005job}.

  \begin{itemize}
    \item In particular, we use the County Business Patterns dataset from 1964 and 1977-1999
    \item Subset to counties that (i) had more than 1500 employees overall in 1964 and (ii) had non-negative aggregate employment growth between 1964 and 1977
  \end{itemize}

  \smallskip\pause
  We use a geocoded dataset of Walmart openings from \citet{arcidiacono2020competitive}

  \begin{itemize}
    \item Treatment dummy is equal to one if the county has any Walmart in that year and our group variable denotes the year of entrance for the \emph{first} Walmart in the county.
  \end{itemize}
\end{frame}

\begin{frame}
  \begin{figure}
    \caption{Effect of Walmart on County $\log$ Retail Employment (TWFE Estimate)}
    
    \input{../../figures/did2s_retail_bootstrap_1000_slides.tex}
  \end{figure}
\end{frame}

\begin{frame}{Factor Model}
  Turning to our factor model estimator, we use the following variables at their 1980 baseline values as instruments:
  \begin{itemize}
    \item share of population employed in manufacturing
    \item shares of population below and above the poverty line
    \item shares of population employed in the private-sector and by the government
    \item shares of population with high-school and college degrees
  \end{itemize}

  \smallskip\pause
  Think that these are predictive of the kinds of economic trends Walmart may be targeting
  \begin{itemize}
    \item Using baseline values helps us avoid picking up on concurrent shocks that are correlated with walmart opening
  \end{itemize}
\end{frame}

\begin{frame}{}
  \begin{figure}
    \caption{Effect of Walmart on County $\log$ Retail Employment (Factor Model)}
    \adjustbox{width = \textwidth}{
      \input{../../figures/factor_retail_p_2_bootstrap_1000_slides.tex}
    }
  \end{figure}
\end{frame}

\begin{frame}{Conclusion}
  \begin{itemize}
    \item Present a fixed-T imputation procedure to identify treatment effects under a factor-model 
    
    \item Allows for differential trends between treated and control groups based on differential exposure to macroeconomic trends
    
    \item Proposed instrument-based identification of factors by using baseline characteristics that correlate with the factor-loadings
  \end{itemize}
\end{frame}

\appendix
\begin{frame}{Removing additive effects}
  We consider the residuals after within-transforming
  $$
    \tilde{y}_{it} = y_{it} - \overline{y}_{0,t} - \overline{y}_{i,pre} + \overline{y}_{0,pre},
  $$
  \begin{gather*}
    \overline{y}_{i, pre} = \frac{1}{T_0} \sum_{t = 1}^{T_0} y_{it} \\
    \overline{y}_{0, t} = \frac{1}{N_{0}} \sum_{i = 1}^N (1 - D_i) y_{it} \\
    \overline{y}_{0, pre} = \frac{1}{T_0} \sum_{t = 1}^{T_0} y_{0, t}
  \end{gather*}

  % \hyperlink{Back}{\beamergotobutton{back-within_transformation}}
\end{frame}

\end{document}
