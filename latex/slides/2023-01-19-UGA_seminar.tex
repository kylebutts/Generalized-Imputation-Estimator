\documentclass{beamer}

\mode<presentation>
{
  \usetheme{Madrid}
  \usecolortheme{default}
  \usefonttheme{default} 
  \setbeamertemplate{navigation symbols}{}
  \setbeamertemplate{caption}[numbered]
} 

\usepackage[utf8]{inputenc}
\usepackage[english]{babel}
\usepackage{bm}
\usepackage{mathabx}
\usepackage{pdfpages}

\usepackage{threeparttable}
\usepackage{tabularx}
\usepackage{booktabs}
\usepackage{adjustbox}
\usepackage{graphicx}

\newcommand{\R}{\mathbb{R}} %real numbers
\newcommand{\N}{\mathbb{N}} %natural number
\newcommand{\indep}{\rotatebox[origin=c]{90}{$\models$}} %Independence symbol
\newcommand{\1}{\mathds{1}} %indicator function
\newcommand{\del}{\partial}
\newcommand{\norm}[1]{\left\lVert#1\right\rVert} %norm operator
\newcommand{\supnorm}[1]{\norm{#1}_\infty} %essential supremum
\newcommand{\convd}{\stackrel{d}{\rightarrow}} %convergence in distribution symbol
\newcommand{\plim}{\stackrel{p}{\rightarrow}} %convergence in probability symbol
\global\long\def\expec#1{\mathbb{E}\left[#1\right]}%
\newcommand{\condexpec}[2]{\mathbb{E}\left[#1 \ \vert \ #2\right]}
\global\long\def\prob#1{\mathbb{P}\left[#1\right]}%
\global\long\def\var#1{\mathrm{Var}\left[#1\right]}%
\global\long\def\cov#1{\mathrm{Cov}\left[#1\right]}%
\global\long\def\one{\mathbf{1}}%


\newcommand{\trueb}{\bm \beta_0}
\newcommand{\truet}{\bm \theta_0}
\newcommand{\trueT}{\bm \Theta_0}
\newcommand{\xbar}{\bar{\bm x}_i}
\newcommand{\Xbar}{\overline{\bm X}}
\newcommand{\ybar}{\overline{\bm y}}
\newcommand{\bcce}{\widehat{\bm \beta}_{CCE}}
\newcommand{\qldmg}{\widehat{\bm \beta}_{QLDMG}} %quasi-diff MG
\newcommand{\qldp}{\widehat{\bm \beta}_{QLDP}}
\newcommand{\ccep}{\widehat{\bm \beta}_{CCEP}}

%summands indexed by i and t
\newcommand{\isum}{\sum_{i = 1}^N}
\newcommand{\tsum}{\sum_{t = 1}^T}
\newcommand{\fep}{\widehat{\bm \beta}_{FEP}}
\newcommand{\gfep}{\widehat{\bm \beta}_{GFEP}}
\newcommand{\bu}{\widehat{\bm \beta}_u}
\newcommand{\br}{\widehat{\bm \beta}_r}
\newcommand{\ba}{\widehat{\bm \beta}_A}

%'sketch of proof' environment
\newenvironment{sproof}{%
  \renewcommand{\proofname}{Sketch of Proof}\proof}{\endproof}

  % Table and Figure labelling ---------------------------------------------------

\usepackage{caption}

\DeclareCaptionLabelSeparator{threedash}{\,---\,}
\DeclareCaptionFont{navy}{\color{navy}}
\DeclareCaptionFont{jet}{\color{jet}}
\captionsetup[table]{format=plain, labelsep=threedash, font={navy, bf}}
\captionsetup[figure]{format=plain, labelsep=threedash, font={navy, bf}}

% Alternative: Left align captions
% \captionsetup[table]{labelfont=it, textfont={navy, bf}, labelsep=newline, justification=raggedright, singlelinecheck=off}
% \captionsetup[figure]{labelfont=it, textfont={navy, bf}, labelsep=newline, justification=raggedright, singlelinecheck=off}

% multifigure with \caption
% \begin{subfigure}\caption{} \end{subfigure}
\usepackage{subcaption}
\captionsetup[subfigure]{format=plain, font={jet, footnotesize, bf}}


% Tables -----------------------------------------------------------------------

% Fix \input with tables
% \input fails when \\ is at end of external .tex file
\makeatletter
\let\input\@@input
\makeatother

% Make tables/figures wider than \textwidth using:
% \begin{adjustbox}{width = 1.2\textwidth, center}
% \end{adjustbox}
\usepackage{adjustbox}

% Slighty more spacing between rows
\usepackage{array}
\renewcommand\arraystretch{1.25}

% Table with easy to use footnotes
% \begin{threeparttable}
%    \begin{tabular} ... \end{tabular}
%    \begin{tablenotes}
%        \item \textit{Notes.}
%    \end{tablenotes}  
% \end{threeparttable}
\usepackage[flushleft]{threeparttable}
\setlength\labelsep{0pt}

% \toprule, \cmidrule, \bottomrule
\usepackage{booktabs}

% If tables are too narrow, fill columns using:
% \begin{tabularx}{\linewidth}{cols}
% col-types: X - center, L - left, R -right
% If you want relative scale for columns: 
% >{\hsize=.8\hsize}X/L/R
\usepackage{tabularx}
\newcolumntype{L}{>{\raggedright\arraybackslash}X}
\newcolumntype{R}{>{\raggedleft\arraybackslash}X}
\newcolumntype{C}{>{\centering\arraybackslash}X}

% Landscape table 
% \begin{landscape} \pagestyle{lscaped} table... \end{landscsape}
% \usepackage{pdflscape} - rotates page left-side up in pdf
% \usepackage{lscape} - does not rotate page, only figure/table

\usepackage{pdflscape}

% For landscape, fix page number location
\usepackage{fancyhdr}
\fancypagestyle{lscaped}{%
    \fancyhf{}
    \renewcommand{\headrulewidth}{0pt}
    \textnormal
    \fancyfoot{%
        \tikz[remember picture,overlay]
        \node[outer sep=2.5cm,above,rotate=90] at (current page.east) {\thepage};
    }
}
  

% ------------------------------------------------------------------------------

\usepackage{pgfplots}
\pgfplotsset{compat=newest}
\usepgfplotslibrary{groupplots}
\usepgfplotslibrary{polar}
\usepgfplotslibrary{smithchart}
\usepgfplotslibrary{statistics}
\usepgfplotslibrary{dateplot}
\usepgfplotslibrary{ternary}
\usetikzlibrary{arrows.meta}
\usetikzlibrary{backgrounds}
\usepgfplotslibrary{patchplots}
\usepgfplotslibrary{fillbetween}
\pgfplotsset{%
    layers/standard/.define layer set={%
        background,axis background,axis grid,axis ticks,axis lines,axis tick labels,pre main,main,axis descriptions,axis foreground%
    }{
        grid style={/pgfplots/on layer=axis grid},%
        tick style={/pgfplots/on layer=axis ticks},%
        axis line style={/pgfplots/on layer=axis lines},%
        label style={/pgfplots/on layer=axis descriptions},%
        legend style={/pgfplots/on layer=axis descriptions},%
        title style={/pgfplots/on layer=axis descriptions},%
        colorbar style={/pgfplots/on layer=axis descriptions},%
        ticklabel style={/pgfplots/on layer=axis tick labels},%
        axis background@ style={/pgfplots/on layer=axis background},%
        3d box foreground style={/pgfplots/on layer=axis foreground},%
    },
}


\setbeamertemplate{enumerate items}[default]
\setbeamertemplate{itemize item}[circle]
\setbeamertemplate{itemize subitem}[square]
\setbeamertemplate{itemize subsubitem}[triangle]
% \setbeamertemplate{page number in head/foot}{}


%Title page info
%%% The [] is what appears on the footline of each slide; default is what is in the brackets {}
\title[]{A Unified Framework for Dynamic Treatment Effect Estimation in Interactive Fixed Effect Models}
\author[Brown and Butts (2022)]{
\begin{tabular}{c @{\extracolsep{16pt}} c}
  Nicholas Brown & 
  Kyle Butts  \\
  {\small\textit{Queen's University}} & 
  {\small\textit{University of Colorado Boulder}} \\
\end{tabular}
}

\date[]{
\begin{tabular}{c @{\extracolsep{16pt}} c}
January 19, 2023
\end{tabular}
}

%puts table of contents before each section
\AtBeginSection[]
{
  \begin{frame}
    \frametitle{Table of Contents}
    \tableofcontents[currentsection]
  \end{frame}
}

\begin{document}

\frame{\titlepage}

% \begin{frame}{Table of Contents}
% \tableofcontents
% \end{frame}

%%%%%%%%%%%%%%%%%%%%%%%%%%%%%%%%%%%%%%%%%%%%%%%%%%%%%%%%%%%%%%%%%%%%%%%

\begin{frame}{Introduction}
    We are interested in effects of a treatment/intervention. 
    \begin{itemize}
        \item Observed outcomes have two potential states:
        \begin{itemize}
            \item Treated: $y_{it}(1)$.
            \item Untreated: $y_{it}(0)$.
        \end{itemize}
        \end{itemize}

            \begin{block}{Definition}
        \textbf{Average Treatment Effect on the Treated:}
        \begin{equation}
            \text{ATT}_t = \condexpec{y_{it}(1) - y_{it}(0)}{D_i = 1}
        \end{equation}
    \end{block}
    \begin{itemize}
        \item $D_i = 1$ if unit is eventually treated; $0$ otherwise.
        \begin{itemize}
            \item See paper for staggered intervention case. 
        \end{itemize}
    \end{itemize}
\end{frame}

%%%%%%%%%%%%%%%%%%%%%%%%%%%%%%%%%%%%%%%%%%%%%%%%%%%%%%%%%%%%%%%%%%%%%%%

\begin{frame}{Introduction}
    Standard parallel trends assumption: 
    \begin{equation}
        \condexpec{y_{it}(0) - y_{i,t-1}(0)}{D_i = 0} = \condexpec{y_{it}(0) - y_{i,t-1}(0)}{D_i = 1}
    \end{equation}

    \vspace{.5 cm}

    Strength of assumption depends on $y_{it}(0)$. We assume a \textbf{common factor model}:
    \begin{equation}
        y_{it}(0) = \mu_i + \theta_t + \sum_{r = 1}^p f_{t,r} * \gamma_{r, i} + u_{it}
    \end{equation}
    \begin{itemize}
        \item \textbf{Factors} are common, time-varying effects.
        \item Time effects are constant. 
    \end{itemize}

\end{frame}%%%%%%%%%%%%%%%%%%%%%%%%%%%%%%%%%%%%%%%%%%%%%%%%%%%%%%%%%%%%%%%%%%%%%%%

\begin{frame}{Example: Job Training Program}
    Workers are eligible for a job training program.
    \begin{itemize}
        \item Interested in workers' income $y_{it}$.
    \end{itemize}

    \vspace{.5cm} \pause

    \textbf{Untreated Model 1:} $y_{it}(0) = TFP_t + skill_i + u_{it}$.
    \begin{itemize}
        \item Two-way fixed effects (TWFE).
    \end{itemize}

    \vspace{.5cm} \pause
    
    \textbf{Untreated Model 2:} $y_{it}(0) = TFP_t * skill_i + u_{it}$.
    \begin{itemize}
        \item Technology adoption is tied to skill. 
    \end{itemize}
\end{frame}

%%%%%%%%%%%%%%%%%%%%%%%%%%%%%%%%%%%%%%%%%%%%%%%%%%%%%%%%%%%%%%%%%%%%%%%

\begin{frame}{Example: Local Effects of Walmart Openings}
    New Walmart may affect local labor market outcomes.
    \begin{itemize}
        \item Interested in employment in retail and non-retail sectors. 
        \item Walmart does place stores randomly (unfortunate).
    \end{itemize}

    \vspace{.5cm} \pause

    \textbf{Untreated Model 1:} $y_{it}(0) = macro_t + local_i + u_{it}$.
    \begin{itemize}
        \item Selection on macro trends and micro conditions.
    \end{itemize}

    \vspace{.5cm} \pause
    
    \textbf{Untreated Model 2:} $y_{it}(0) = macro_t * local_i + u_{it}$.
    \begin{itemize}
        \item Selection on local exposure to macro trends. 
    \end{itemize}
\end{frame}

%%%%%%%%%%%%%%%%%%%%%%%%%%%%%%%%%%%%%%%%%%%%%%%%%%%%%%%%%%%%%%%%%%%%%%%

\begin{frame}{Contribution}
        Current methods for factor models require
        \begin{enumerate}
            \item \textbf{Large time series:} Gobillon and Magnac (2016), Xu (2017), Abadie (2021), Athey et al. (2021).
            \item \textbf{White noise errors:} Imbens et al. (2021).
            \item \textbf{Time-constant effects:} Chan and Kwok (2022).
        \end{enumerate}

         \vspace{.5 cm}

         Our identification approach: estimate factor model, impute $y_{it}(0)$.
         \begin{itemize}
             \item Long or short time series. 
             \item No restrictions on time series dependence. 
             \item Obtain plotted $y_{it}(0)$ for visual inspection.
             \item Allow $\text{ATT}_t$ to vary over time. 
         \end{itemize}

         
\end{frame}

%%%%%%%%%%%%%%%%%%%%%%%%%%%%%%%%%%%%%%%%%%%%%%%%%%%%%%%%%%%%%%%%%%%%%%%

\begin{frame}{Contribution}

    \textbf{Imputation solution:} ``estimate" time and unit fixed effects ($\mu_i + \theta_t$) with untreated sample, then impute $\condexpec{y_{it}(0)}{D_i = 1}$.
    \begin{itemize}
        \item Borusyak et al. (2022), Gardner (2021), Wooldridge (2021).
        \item We are the first to consider factor model imputation.
        \begin{itemize}
            \item Explicitly nest TWFE.
        \end{itemize}
    \end{itemize}

    \vspace{.5cm}

         \textbf{Callaway and Karami (2022):} Fixed-$T$, differencing strategy. 
         \begin{itemize}
             \item Their instruments are a subset of ours. 
             \item We give general identification result.
         \end{itemize}
\end{frame}

%%%%%%%%%%%%%%%%%%%%%%%%%%%%%%%%%%%%%%%%%%%%%%%%%%%%%%%%%%%%%%%%%%%%%%%

\begin{frame}{Model: Panel Structure}
    $N$ individuals observed for $T$ times periods.
\begin{itemize}
    \item Treatment begins \textbf{after} period $T_0$.
    \item $N_1$ treated individuals, $N_0$ untreated individuals.
\end{itemize}

\vspace{.5cm}

Stack outcomes over time:
\begin{itemize}
    \item $\bm y_i = (y_{i1},...,y_{iT})'$.
    \item $\bm F = (\bm f_1,...,\bm f_T)'$ where $\bm f_t = (f_{1t},...,f_{pt})'$.
    \item $\bm \gamma_i = (\gamma_{i1},...,\gamma_{ip})'$.
\end{itemize}
\end{frame}

%%%%%%%%%%%%%%%%%%%%%%%%%%%%%%%%%%%%%%%%%%%%%%%%%%%%%%%%%%%%%%%%%%%%%%%

\begin{frame}{Model: Potential Outcomes}
We assume random sampling in the cross-section.

    \begin{block}{Assumption 1 (Untreated Potential Outcomes)}
        \begin{equation}
            y_{it}(0) = \mu_i + \theta_t + \bm f_t' \bm \gamma_i + u_{it}
        \end{equation}
    \end{block}
        \begin{itemize}
            \item $\bm f_t$: $p \times 1$ vector of common factors.
            \item $\bm \gamma_i$: $p \times 1$ vector of individual factor loadings.
            \item Nests the common TWFE model ($\bm \gamma_i = \bm 0$). 
        \end{itemize}

        \vspace{.5 cm}

        \textbf{Important:} Allows arbitrary treatment effects.
        \begin{equation}
            y_{it}(1) = y_{it}(0) + \tau_{it}
        \end{equation}
\end{frame}

%%%%%%%%%%%%%%%%%%%%%%%%%%%%%%%%%%%%%%%%%%%%%%%%%%%%%%%%%%%%%%%%%%%%%%%

\begin{frame}{Model: Anticipation}
    \begin{block}{Assumption 2 (No Anticipation)}
        $y_{it} = y_{it}(0)$ if unit $i$ is untreated at time $t$.
    \end{block}
    \begin{itemize}
        \item Treated units do not change their behavior before treatment.
        \item Can be relaxed to a conditional mean assumption. 
        \item Define first treatment date to capture anticipatory effectcs.  
    \end{itemize}
\end{frame}

%%%%%%%%%%%%%%%%%%%%%%%%%%%%%%%%%%%%%%%%%%%%%%%%%%%%%%%%%%%%%%%%%%%%%%%

\begin{frame}{Model: Selection}
    \begin{block}{Assumption 3 (Selection into Treatment)}
        \begin{equation*}
            \condexpec{u_{it}}{\mu_i, \bm \gamma_i, D_i} = \theta_t
        \end{equation*}
        for a sequence of constants $\theta_1,...,\theta_T$.
    \end{block}
    \begin{itemize}
        \item \textbf{Mean Model:} $\condexpec{y_{it}(0)}{\mu_i, \bm \gamma_i, D_i} = \mu_i + \theta_t + \bm f_t' \bm \gamma_i$.
        \item Allows selection into treatment based on level and interactive fixed effects (e.g. skill and innovation).
        \item Can include covariates.
    \end{itemize}
\end{frame}

%%%%%%%%%%%%%%%%%%%%%%%%%%%%%%%%%%%%%%%%%%%%%%%%%%%%%%%%%%%%%%%%%%%%%%%

\begin{frame}{ATT Identification}

Treated sample:
\begin{equation}
    \condexpec{y_{it}(0)}{D_i = 1} = \theta_t + \condexpec{\mu_i}{D_i = 1} + \bm f_t' \condexpec{\bm \gamma_i}{D_i = 1}
\end{equation}
\begin{itemize} 
    \item \textbf{Insight:} Do not need to know $\bm \gamma_i$; only $\condexpec{\bm \gamma_i}{D_i = 1}$.
\end{itemize}

\vspace{.5cm}

\textbf{First:} eliminate additive effects. 
\begin{itemize}
    \item Must be careful to maintain common factors. 
\end{itemize}
    
\end{frame}

%%%%%%%%%%%%%%%%%%%%%%%%%%%%%%%%%%%%%%%%%%%%%%%%%%%%%%%%%%%%%%%%%%%%%%%

\begin{frame}{ATT Identification}

\begin{block}{Definition}

\vspace{-.25 cm}

    \begin{gather*}
    \overline{y}_{0 , t} = \frac{1}{N_{0}} \sum_{i = 1}^N (1 - D_i) y_{it} \\
    \overline{y}_{i,t\leq T_0} = \frac{1}{T_0} \sum_{t = 1}^{T_0} y_{it} \\
    \overline{y}_{0, t < T_0} = \frac{1}{N_{0} T_0} \sum_{i = 1}^N \sum_{t = 1}^{T_0} (1 - D_i) y_{it}
\end{gather*}
\end{block}

\begin{itemize}
    \item $\overline{y}_{0,t}$: never-treated cross-sectional averages.
    \item $\overline{y}_{i,t \leq T_0}$: pre-treated time averages.
    \item $\overline{y}_{0, t \leq T_0}$: overall never-treated pre-treated average. 
\end{itemize}

\vspace{.25 cm}

\textbf{Note:} none of the averages are contaminated by treatment. 


    
\end{frame}

%%%%%%%%%%%%%%%%%%%%%%%%%%%%%%%%%%%%%%%%%%%%%%%%%%%%%%%%%%%%%%%%%%%%%%%

%%%%%%%%%%%%%%%%%%%%%%%%%%%%%%%%%%%%%%%%%%%%%%%%%%%%%%%%%%%%%%%%%%%%%%%

\begin{frame}{ATT Identification}
    We only consider residuals $\tilde{y}_{it} = y_{it} - \overline{y}_{0,t} - \overline{y}_{i,t \leq T_0} + \overline{y}_{0, t \leq T_0}$.
    \begin{itemize}
        \item Similar to double-demeaned TWFE residuals.
    \end{itemize}

    \vspace{.5 cm}

    \textbf{Note:} not the residuals from TWFE regression on untreated sample.
    \begin{itemize}
        \item Borusyak et al. (2022), Gardner (2021), Wooldridge (2021).
        \item Will remove $\mu_i + \theta_t$ but ruins common factor structure.
        \begin{itemize}
            \item Some units have more time observations than others.
        \end{itemize}
        \item Restrict regression to use only pre-treatment variation.
    \end{itemize}

    
\end{frame}

%%%%%%%%%%%%%%%%%%%%%%%%%%%%%%%%%%%%%%%%%%%%%%%%%%%%%%%%%%%%%%%%%%%%%%%

\begin{frame}{ATT Identification}

    \begin{block}{Lemma 2.1}
        \begin{align*}
            \condexpec{\tilde{y}_{it}}{D_i = 1} 
            &= \condexpec{\tau_{it} + \tilde{\bm f}_t' \tilde{\bm \gamma}_i }{D_i = 1}\\
            &= \text{ATT}_t + \tilde{\bm f}'_t \condexpec{\Tilde{\bm \gamma}_i}{D_i = 1}
        \end{align*}
        where $\tilde{\bm f}_t$ are the pre-treatment demeaned factors and $\tilde{\bm \gamma}_i$ are the never-treated demeaned loadings.
    \end{block}
    \begin{itemize}
        \item \textbf{General result:} Our transformation removes $(\mu_i, \theta_t)$ but preserves a common factor structure.
        \item Every factor imputation approach needs a result like Lemma 2.1 to nest TWFE structure.
    \end{itemize}

    
\end{frame}

%%%%%%%%%%%%%%%%%%%%%%%%%%%%%%%%%%%%%%%%%%%%%%%%%%%%%%%%%%%%%%%%%%%%%%%

\begin{frame}{ATT Identification}
Post-treatment treated outcomes:
\begin{equation}
    \condexpec{\Tilde{y}_{it}}{D_i = 1} = \text{ATT}_t + \Tilde{\bm f}_t' \condexpec{\tilde{\bm \gamma}_i}{D_i = 1}
\end{equation}

    \begin{block}{Corollary 2.1}
        If
        \begin{equation*}
            \condexpec{\bm \gamma_i}{D_i} = \expec{\bm \gamma_i}
        \end{equation*}, the ATTs are identified by the modified TWFE transformation.
    \end{block}
    \begin{equation}
        \condexpec{\tilde{y}_{it}}{D_i = 1} = \condexpec{\tau_{it}}{D_i = 1} = \text{ATT}_t
    \end{equation}
    for $t > T_0$.

    \begin{itemize}
        \item Says TWFE is sufficient even if there are factors. 
        \item \textbf{Later:} we can test for this condition.
    \end{itemize}
\end{frame}

%%%%%%%%%%%%%%%%%%%%%%%%%%%%%%%%%%%%%%%%%%%%%%%%%%%%%%%%%%%%%%%%%%%%%%%

%%%%%%%%%%%%%%%%%%%%%%%%%%%%%%%%%%%%%%%%%%%%%%%%%%%%%%%%%%%%%%%%%%%%%%%

\begin{frame}{ATT Identification}
\label{Theorem 2.1 proof back}
    \begin{block}{Definition (Imputation Matrix)}
        Let $\bm X$, $\bm W$ be $n \times k$ and $m \times k$ respectively. If $\text{Rank}(\bm X) = k$, the \textbf{imputation matrix} is 
        \begin{equation}
            \bm P(\bm W, \bm X) = \bm W (\bm X' \bm X)^{-1} \bm X'
        \end{equation}
    \end{block}

    \vspace{.5 cm}

    \textbf{Notation:} Stack the factors $\tilde{\bm F} = (\tilde{\bm f_1},...,\tilde{\bm f}_T)'$.
    \begin{itemize}
        \item Let $\tilde{\bm F}_{t \leq T_0}$ and $\tilde{\bm F}_{t > T_0}$ be the first $T_0$ and last $T - T_0$ rows of $\tilde{\bm F}$.
    \end{itemize}

    \vspace{.5 cm}

    \begin{block}{Theorem 2.1}
        Suppose $\tilde{\bm F}$ is known and $\text{Rank}(\tilde{\bm F}_{t \leq T_0}) = p$. Then for $t > T_0$,
        \begin{equation}
            \text{ATT}_t = \condexpec{\tilde{y}_{it} - \bm P(\tilde{\bm f}_t, \tilde{\bm F}_{t \leq T_0}) \bm y_{i, t \leq T_0} }{D_i = 1}
        \end{equation}
    \end{block}

    \hyperlink{Theorem 2.1 proof}{\beamerbutton{Proof.}}
        
    
\end{frame}

%%%%%%%%%%%%%%%%%%%%%%%%%%%%%%%%%%%%%%%%%%%%%%%%%%%%%%%%%%%%%%%%%%%%%%%

\begin{frame}{Factor Identification}

We consider instrument-based identification of Ahn, Lee, and Schmidt (2013).
\begin{itemize}
    \item Allows fixed-$T$ analysis.
    \item Provides moment conditions; inference is easy.
\end{itemize}

\begin{block}{Assumption 4}
    The following rank assumptions for the untreated units, where $\bm w_i$ is a $L \times 1$ vector of instruments: 
\begin{enumerate}[(i)]
    \item $\text{Rank}( Var(\bm \gamma_i \ \vert \ D_i = 0 ) ) = \text{Rank}(\tilde{\bm F}_{t < T_0}) = p < T_0$.
    \item The matrix $\expec{\bm I_{(T-p)} \otimes \bm w_i \tilde{\bm \gamma}_i' \ \vert \ D_i = 0}$ has full column rank.
    \item $\condexpec{\bm u_i}{\bm w_i, D_i = 0} = \bm 0$.
    \item $p$ is known.
\end{enumerate}
\end{block}
    
\end{frame}

%%%%%%%%%%%%%%%%%%%%%%%%%%%%%%%%%%%%%%%%%%%%%%%%%%%%%%%%%%%%%%%%%%%%%%%

\begin{frame}{Factor Identification}
    \textbf{Fact:} $\tilde{\bm F} \tilde{\bm \gamma}_i$ are never separately identifiable (\textbf{rotation problem}):
    \begin{equation*}
        \tilde{\bm F} \tilde{\bm \gamma}_i = (\tilde{\bm F}\bm A) (\bm A^{-1} \tilde{\bm \gamma}_i)
    \end{equation*} 

    \vspace{.25cm}

    Impose the $p^2$ normalizations
    \begin{equation*}
        \tilde{\bm F}(\bm \theta) = 
    \begin{pmatrix}
        \bm \Theta\\
        - \bm I_p
    \end{pmatrix}
    \end{equation*}

    where $\bm \theta = \text{vec}(\bm \Theta)$. The \textbf{quasi-long-differencing} (QLD) matrix is
    \begin{equation*}
        \bm H(\bm \theta)' = 
    \begin{pmatrix}
        \bm I_{(T-p)} & \bm \Theta
    \end{pmatrix}
    \end{equation*}
    \begin{itemize}
        \item $\bm H(\bm \theta)' \bm F(\bm \theta) = \bm 0$ for any $\bm \theta$. 
    \end{itemize}


    
\end{frame}

%%%%%%%%%%%%%%%%%%%%%%%%%%%%%%%%%%%%%%%%%%%%%%%%%%%%%%%%%%%%%%%%%%%%%%%

\begin{frame}{Factor Identification}
\label{Lemma 2.3 proof back}
    \begin{block}{Lemma 2.2}
        $\bm \theta$ is identified by
        \begin{equation}
            \condexpec{\bm H(\bm \theta)' \tilde{\bm y}_i \otimes \bm w_i }{D_i = 0} = \bm 0
        \end{equation}
    \end{block}
    \begin{itemize}
        \item Use control sample to exploit entire time series. 
    \end{itemize}

    \vspace{.5cm}
    
    $\tilde{\bm F}(\bm \theta) \neq \tilde{\bm F}$, but irrelevant for estimation. 
    \begin{block}{Lemma 2.3}
        $\bm P(\tilde{\bm F}(\bm \theta)_{t > T_0}, \tilde{\bm F}(\bm \theta)_{t \leq T_0}) = \bm P(\tilde{\bm F}_{t > T_0}, \tilde{\bm F}_{t \leq T_0})$
    \end{block}

    \hyperlink{Lemma 2.3 proof}{\beamerbutton{Proof.}}
    
    \begin{itemize}
        \item Knowing $\bm \theta$ is just as good as knowing $\tilde{\bm F}$.
        \item Result holds for \emph{any} rotation.
        \begin{itemize}
            \item E.g. principal components, CCE, etc.
        \end{itemize}
    \end{itemize}
\end{frame}

%%%%%%%%%%%%%%%%%%%%%%%%%%%%%%%%%%%%%%%%%%%%%%%%%%%%%%%%%%%%%%%%%%%%%%%

\begin{frame}{Estimation}
    We jointly estimate the QLD parameters and ATTs:
    \begin{gather*}
        \expec{ \bm g_{i0}(\bm \theta)} = \expec{\frac{(1 - D_i)}{\mathbb{P}(D_i = 0)} \bm H(\bm \theta)' \tilde{\bm y}_i \otimes \bm w_i } = \bm 0\\
        \expec{\bm g_{i1}(\bm \theta, \bm \tau)} = \expec{\frac{D_i}{\mathbb{P}(D_i = 1)} \left( \tilde{y}_{i,t > T_0} - \bm P(\tilde{\bm F}_{t > T_0}, \tilde{\bm F}_{t \leq T_0}) \bm y_{i, t \leq T_0} - \bm \tau \right) } = \bm 0
    \end{gather*}
    where $\bm \tau = (\tau_{T_0 + 1},...,\tau_T)'$ denote the ATTs. 

    \vspace{.5cm}

    \begin{equation}
        \bm \Delta = 
        \begin{pmatrix}
            \expec{\bm g_{i0}(\bm \theta)} & \bm 0\\
            \bm 0 & \expec{\bm g_{i1}(\bm \theta, \bm \tau)}
        \end{pmatrix}
    \end{equation}
\end{frame}

%%%%%%%%%%%%%%%%%%%%%%%%%%%%%%%%%%%%%%%%%%%%%%%%%%%%%%%%%%%%%%%%%%%%%%%

\begin{frame}{Estimation}
    Collect the moment functions as $\bm g_i(\bm \theta, \bm \tau) = (\bm g_{i0}(\bm \theta)', \bm g_{i1}(\bm \theta, \bm \tau))'$ and let $(\widehat{\bm \theta}', \widehat{\bm \tau}')'$ solve
    \begin{equation*}
        \min_{\bm \theta, \bm \tau} \left( \sum_{i = 1}^N \bm g_i(\bm \theta, \bm \tau) \right)' \widehat{\bm \Delta}^{-1} \left( \sum_{i = 1}^N \bm g_i(\bm \theta, \bm \tau) \right)
    \end{equation*}

    \begin{block}{Theorem 3.1}
        Suppose $\widehat{\bm \Delta} \plim \bm \Delta$. Then for $T$ fixed and $N \rightarrow \infty$, 
        \begin{equation}
            \sqrt{N}(\widehat{\bm \tau} - \bm \tau) \convd N \left( \bm 0, \bm \Delta_1 + \bm D_{21}' \text{Avar}(\sqrt{N}(\widehat{\bm \theta} - \bm \theta)) \bm D_{21} \right)
        \end{equation}
    \end{block}
    \begin{itemize}
        \item Inference is simple to carry out.
        \begin{itemize}
            \item Joint and two-step estimators are asymptotically equivalent. 
        \end{itemize}
        \item Can estimate aggregate effects (Callaway and Sant'Anna 2021).
    \end{itemize}
\end{frame}

%%%%%%%%%%%%%%%%%%%%%%%%%%%%%%%%%%%%%%%%%%%%%%%%%%%%%%%%%%%%%%%%%%%%%%%

\begin{frame}{Asymptotic Variance}
    \begin{equation*}
        Avar(\sqrt{N}(\widehat{\bm \tau} - \bm \tau)) = \underbrace{\bm \Delta_1}_\text{Uncertainty for ATT, given $\bm \theta$} + \underbrace{\bm D_{21} Avar(\sqrt{N}(\widehat{\bm \theta} - \bm \theta)) \bm D_{21}'}_\text{Uncertainty for $\bm \theta$}
    \end{equation*}

    \begin{block}{Theorem 3.2}
        Define 
        \begin{equation}
            \widehat{\bm \Delta}_1 = \frac{1}{N_1} \isum D_i \left( \widehat{\bm \Delta}_{i1} - \widehat{\bm \tau} \right) \left( \widehat{\bm \Delta}_{i1} - \widehat{\bm \tau} \right)'
        \end{equation}
        Under the conditions of Theorem 3.1, 
        \begin{equation}
            \widehat{\bm \Delta}_1 \plim \bm \Delta_1
        \end{equation}
    \end{block}
    \begin{itemize}
        \item Gives valid standard errors when $\bm D_{21} = \bm 0$. 
    \end{itemize}
\end{frame}

%%%%%%%%%%%%%%%%%%%%%%%%%%%%%%%%%%%%%%%%%%%%%%%%%%%%%%%%%%%%%%%%%%%%%%%

% \begin{frame}{Anticipating Treatment}
%     We may believe treated units anticipate treatment at period $Q + 1 \leq T_0$.
%     \begin{itemize}
%         \item Redefine last pre-treatment period as $Q$:
%     \end{itemize}
%     \begin{equation*}
%         \expec{ \frac{D_i}{\mathbb{P}(D_i = 1)} \Big( \tilde{\bm y}_{i,t >Q} - \bm P\big(\bm F(\bm \theta)_{t > Q}, \bm F(\bm \theta)_{t \leq Q}\big) \tilde{\bm y}_{i,t \leq Q} - \bm \tau \Big) } = \bm 0 
%     \end{equation*}
%     where $\bm \tau = (\tau_{Q + 1},...,\tau_{T_0},...,\tau_T)$.

%     \begin{block}{Testing for Anticipation}
%         Under Assumption 2 (No Anticipation), 
%         \begin{equation*}
%             \tau_{Q+1} = ... = \tau_{T_0} = 0
%         \end{equation*}
%     \end{block}
%     \begin{itemize}
%         \item Testable hypothesis by Theorem 3.1.
%         \item Need $Q > p$.
%     \end{itemize}
% \end{frame}

%%%%%%%%%%%%%%%%%%%%%%%%%%%%%%%%%%%%%%%%%%%%%%%%%%%%%%%%%%%%%%%%%%%%%%%

% \begin{frame}{Adding Covariates}
%     Weaken the parallel trends assumption with a rich set of covariates:
%     \begin{equation}
%         \condexpec{y_{it}(0)}{\bm X_i, \mu_i, \bm \gamma_i, D_i} = \bm X_i \bm \beta_t + \theta_t + \mu_i + \bm F \bm \gamma_i
%     \end{equation}

%     Jointly estimate $(\bm \theta', \bm \beta')'$:
%     \begin{equation}
%         \condexpec{\bm H(\bm \theta)' (\tilde{\bm y_i} - \tilde{\bm X_i} \bm \beta_t) \otimes \bm w_i}{D_i = 0} = \bm 0
%     \end{equation}
%     \begin{itemize}
%         \item Include $\bm X_i \subset \bm w_i$.
%     \end{itemize}

%     Identifying the ATTs:
%     \begin{equation*}
%         \condexpec{(\tilde{\bm y}_{i, t > T_0} - \tilde{\bm X}_{i,t > T_0} \bm \beta_{t > T_0}) - \bm P(\tilde{\bm F}_{t > T_0}, \tilde{\bm F}_{t \leq T_0})( \tilde{\bm y}_{i,t \leq T_0} - \tilde{\bm X}_{i, t \leq T_0} \bm \beta_{t \leq T_0}) }{D_i = 1}
%     \end{equation*}
% \end{frame}

% %%%%%%%%%%%%%%%%%%%%%%%%%%%%%%%%%%%%%%%%%%%%%%%%%%%%%%%%%%%%%%%%%%%%%%%

% \begin{frame}{Adding Covariates}
%     We can allow slopes to vary by treatment status and timing:
%     \begin{gather*}
%         \condexpec{\bm H(\bm \theta)' (\tilde{\bm y_i} - \tilde{\bm X_i} \bm \beta_0) \otimes \bm w_i}{D_i = 0}\\
%         \condexpec{ \bm z_i \otimes \left( \tilde{\bm e}_{i, t > T_0}- \bm P(\tilde{\bm F}_{t > T_0}, \tilde{\bm F}_{t \leq T_0}) \tilde{\bm e}_{i, t \leq T_0} - \bm \tau \right) }{D_i = 1}
%     \end{gather*}
%     \begin{itemize}
%         \item $\tilde{\bm e}_{i, t > T_0} = \tilde{\bm y}_{i,t > T_0} - \tilde{\bm X}_{i, t > T_0} \bm \beta_{1, \text{post}}$.
%         \item $\tilde{\bm e}_{i, t \leq T_0} = \tilde{\bm y}_{i, t \leq T_0} - \tilde{\bm X}_{i, t \leq T_0} \bm \beta_{1, \text{pre}}$.
%     \end{itemize}

%     \vspace{.5 cm}

%     $\bm z_i$ is a vector of instruments to identify the slopes.
%     \begin{itemize}
%         \item More model complexity means more instruments necessary.
%     \end{itemize}
    
% \end{frame}

%%%%%%%%%%%%%%%%%%%%%%%%%%%%%%%%%%%%%%%%%%%%%%%%%%%%%%%%%%%%%%%%%%%%%%%

%%%%%%%%%%%%%%%%%%%%%%%%%%%%%%%%%%%%%%%%%%%%%%%%%%%%%%%%%%%%%%%%%%%%%%%

\begin{frame}{Testing the Factor Model}
    In general, the TWFE model is easier to estimate than a general factor model.
    \begin{itemize}
        \item Computationally simple. 
        \item Can be done via routine statistical packages. 
    \end{itemize}

    \vspace{.5 cm}

    Could compare our estimator to TWFE via generalized Hausman test.
    \begin{itemize}
        \item Requires structural assumptions on the model's errors.
        \item Possibly poor finite-sample performance. 
    \end{itemize}

    \vspace{.5 cm}

    We provide some simple tests for sufficiency of TWFE
\end{frame}

%%%%%%%%%%%%%%%%%%%%%%%%%%%%%%%%%%%%%%%%%%%%%%%%%%%%%%%%%%%%%%%%%%%%%%%

\begin{frame}{Testing the Factor Model: Number of Factors}
    Ahn, Lee, and Schmidt (2013) provide tests for $p$ based off of moments for estimating $\bm \theta$. 

    \begin{align*}
        H_0&: y_{it}(0) = \mu_i + \lambda_t + u_{it} \\
        H_A&: y_{it}(0) = \mu_i + \lambda_t + \bm f_t' \bm \gamma_i + u_{it} 
    \end{align*}

    \begin{block}{Theorem 4.1}
        Under the null hypothesis, $p = 0$. 
    \end{block}
\end{frame}

%%%%%%%%%%%%%%%%%%%%%%%%%%%%%%%%%%%%%%%%%%%%%%%%%%%%%%%%%%%%%%%%%%%%%%%

\begin{frame}{Testing the Factor Model: Heterogeneity Variation}
    Even if factors are present, Corollary 2.1 says TWFE is consistent if $\condexpec{\bm \gamma_i}{D_i} = \expec{\bm \gamma_i}$.

    \vspace{.25 cm}
    
    \begin{itemize}
        \item No systemic variation in heterogeneity.
    \end{itemize}

    \vspace{.5 cm}

    Impute the pre-treated observations onto an identity matrix:

    \begin{equation*}
        \bm P(\bm I_{p}, \bm F(\bm \theta)_{t \leq T_0}) \condexpec{\bm y_{i, t \leq T_0}}{D_i} = \bm A^* \condexpec{\bm \gamma_i}{D_i} 
    \end{equation*}
    \begin{itemize}
        \item $\bm A^*$ is a nonsingular rotation.
    \end{itemize}

    \begin{equation*}
        \bm A^* \condexpec{\bm \gamma_i}{D_i} = \bm A^* \expec{\bm \gamma_i} \iff \condexpec{\bm \gamma_i}{D_i } = \expec{\bm \gamma_i}
    \end{equation*}
\end{frame}

%%%%%%%%%%%%%%%%%%%%%%%%%%%%%%%%%%%%%%%%%%%%%%%%%%%%%%%%%%%%%%%%%%%%%%%

\begin{frame}{Testing the Factor Model: Heterogeneity Variation}
    \begin{gather*}
        \expec{\frac{(1 - D_i)}{\mathbb{P}(D_i = 0)} \bm H(\bm \theta)' \tilde{\bm y}_i \otimes \bm w_i} = \bm 0\\
        \expec{\frac{(1 - D_i)}{\mathbb{P}(D_i = 0)} \left( \bm P(\bm I_{p}, \bm F(\bm \theta)_{t \leq T_0}) \tilde{\bm y}_{i, t \leq T_0} - \bm \gamma_0 \right) } = \bm 0\\
        \expec{\frac{D_i}{\mathbb{P}(D_i = 1)} \left( \bm P(\bm I_{p}, \bm F(\bm \theta)_{t \leq T_0}) \tilde{\bm y}_{i, t \leq T_0} - \bm \gamma_1 \right) } = \bm 0\\
        \expec{\bm P(\bm I_{p}, \bm F(\bm \theta)_{t \leq T_0}) \tilde{\bm y}_{i, t \leq T_0} - \bm \gamma} = \bm 0
    \end{gather*}

    \begin{block}{Theorem 4.2}
        If $\bm \gamma_0 = \bm \gamma_1 = \bm \gamma$, TWFE is sufficient for estimating $\bm \tau$.
    \end{block}
\end{frame}

%%%%%%%%%%%%%%%%%%%%%%%%%%%%%%%%%%%%%%%%%%%%%%%%%%%%%%%%%%%%%%%%%%%%%%%

\begin{frame}{Testing the Factor Structure: Common Factors}
    \textbf{Factor model:} The factors in the untreated sample are the same as those in the treated sample ($\bm F_0 \neq \bm F_1$).
    \begin{itemize}
        \item \textbf{Idea:} Estimate the factors using both samples and compare.
    \end{itemize}

    \begin{gather*}
    \expec{\bm g^0_i(\bm \theta_0)} = \expec{\frac{(1-D_i)}{\mathbb{P}(D_i = 0)} \bm H^*(\bm \theta_0)' \bm y_{i,t < T_0} \otimes \bm w_i } = \bm 0_{T_0 \times 1} \\
    \expec{\bm g^1_i(\bm \theta_1)} = \expec{\frac{D_i}{\mathbb{P}(D_i = 1)} \bm H^*(\bm \theta_1)' \bm y_{i,t < T_0} \otimes \bm w_i } = \bm 0_{T_0 \times 1}
\end{gather*}
\begin{itemize}
    \item Just like loadings: $\bm \theta_0 = \bm \theta_1 \iff \bm F_0 = \bm F_1$.
\end{itemize}{}
\end{frame}

%%%%%%%%%%%%%%%%%%%%%%%%%%%%%%%%%%%%%%%%%%%%%%%%%%%%%%%%%%%%%%%%%%%%%%%

\begin{frame}{Testing the Factor Structure: Common Factors}
    Can test for structural breaks in factors.
    \begin{itemize}
        \item Break accross the sample, not over time. 
        \item First, test $p_0 = p_1$.
        \begin{itemize}
            \item If $p_0 > p_1$, might still be fine. 
        \end{itemize}
    \end{itemize}

    \vspace{1cm}

    Define:
    \begin{gather*}
    \bm S_0(\bm \theta_0) = Var(\bm g^0_i)\\
    \bm S_1(\bm \theta_1) = Var(\bm g^1_i)
\end{gather*}
with consistent estimators $\widehat{\bm S}_0$ and $\widehat{\bm S}_1$.
\end{frame}

%%%%%%%%%%%%%%%%%%%%%%%%%%%%%%%%%%%%%%%%%%%%%%%%%%%%%%%%%%%%%%%%%%%%%%%

\begin{frame}{Testing the Factor Structure: Common Factors}
    Test statistic:
    \begin{equation}
    \bm J(\bm \theta_0, \bm \theta_1) = \frac{N_0}{N} g^0 (\bm \theta_0)' \widehat{\bm S}_0^{-1} g^0 (\bm \theta_0) + \frac{N_1}{N} g^1 (\bm \theta_1)' \widehat{\bm S}_1^{-1} \bm g^1(\bm \theta_1)
\end{equation}

\begin{block}{Theorem 4.3}
    Let $\widehat{\bm \theta}$ use both sets of moments, and $(\widehat{\bm \theta}_0, \widehat{\bm \theta}_1$ use each set of moments. If $\bm \theta_0 = \bm \theta_1$, then
    \begin{equation*}
        N * \left( J( \widehat{\bm \theta}, \widehat{\bm \theta}) - J(\widehat{\bm \theta}_0, \widehat{\bm \theta}_1)  \right) \stackrel{d}{\rightarrow} \chi^2_{((T_0-p)p)}
    \end{equation*}
\end{block}
\end{frame}

%%%%%%%%%%%%%%%%%%%%%%%%%%%%%%%%%%%%%%%%%%%%%%%%%%%%%%%%%%%%%%%%%%%%%%%

% \begin{frame}{Testing the Factor Model}
%     In general, the TWFE model is easier to estimate than a general factor model.
%     \begin{itemize}
%         \item Computationally simple. 
%         \item Can be done via routine statistical packages. 
%     \end{itemize}

%     \vspace{.5 cm}

%     Could compare our estimator to TWFE via generalized Hausman test.
%     \begin{itemize}
%         \item May require structural assumptions on the model's errors.
%         \item Possibly poor finite-sample performance. 
%     \end{itemize}

%     \vspace{.5 cm}

%     We provide some simple tests for sufficiency of TWFE.
% \end{frame}

% %%%%%%%%%%%%%%%%%%%%%%%%%%%%%%%%%%%%%%%%%%%%%%%%%%%%%%%%%%%%%%%%%%%%%%%

% \begin{frame}{Testing the Factor Model: Number of Factors}
%     Ahn, Lee, and Schmidt (2013) provide tests for $p$ based off of moments for estimating $\bm \theta$. 

%     \begin{align*}
%         H_0&: y_{it}(0) = \mu_i + \lambda_t + u_{it} \\
%         H_A&: y_{it}(0) = \mu_i + \lambda_t + \bm f_t' \bm \gamma_i + u_{it} 
%     \end{align*}

%     \begin{block}{Theorem 4.1}
%         Under the null hypothesis, $p = 0$. 
%     \end{block}
% \end{frame}

% %%%%%%%%%%%%%%%%%%%%%%%%%%%%%%%%%%%%%%%%%%%%%%%%%%%%%%%%%%%%%%%%%%%%%%%

% \begin{frame}{Testing the Factor Model: Heterogeneity Variation}
%     Even if factors are present, Corollary 2.1 says TWFE is consistent if $\condexpec{\bm \gamma_i}{D_i} = \expec{\bm \gamma_i}$.

%     \vspace{.25 cm}
    
%     \begin{itemize}
%         \item No systemic variation in heterogeneity.
%     \end{itemize}

%     \vspace{.5 cm}

%     Impute the pre-treated observations onto an identity matrix:

%     \begin{equation*}
%         \bm P(\bm I_{p}, \bm F(\bm \theta)_{t \leq T_0}) \condexpec{\bm y_{i, t \leq T_0}}{D_i} = \bm A^* \condexpec{\bm \gamma_i}{D_i} 
%     \end{equation*}
%     \begin{itemize}
%         \item $\bm A^*$ is a nonsingular rotation.
%     \end{itemize}

%     \begin{equation*}
%         \bm A^* \condexpec{\bm \gamma_i}{D_i} = \bm A^* \expec{\bm \gamma_i} \iff \condexpec{\bm \gamma_i}{D_i } = \expec{\bm \gamma_i}
%     \end{equation*}
% \end{frame}

% %%%%%%%%%%%%%%%%%%%%%%%%%%%%%%%%%%%%%%%%%%%%%%%%%%%%%%%%%%%%%%%%%%%%%%%

% \begin{frame}{Testing the Factor Model: Heterogeneity Variation}
%     \begin{gather*}
%         \expec{\frac{(1 - D_i)}{\mathbb{P}(D_i = 0)} \bm H(\bm \theta)' \tilde{\bm y}_i \otimes \bm w_i} = \bm 0\\
%         \expec{\frac{(1 - D_i)}{\mathbb{P}(D_i = 0)} \left( \bm P(\bm I_{p}, \bm F(\bm \theta)_{t \leq T_0}) \tilde{\bm y}_{i, t \leq T_0} - \bm \gamma_0 \right) } = \bm 0\\
%         \expec{\frac{D_i}{\mathbb{P}(D_i = 1)} \left( \bm P(\bm I_{p}, \bm F(\bm \theta)_{t \leq T_0}) \tilde{\bm y}_{i, t \leq T_0} - \bm \gamma_1 \right) } = \bm 0\\
%         \expec{\bm P(\bm I_{p}, \bm F(\bm \theta)_{t \leq T_0}) \tilde{\bm y}_{i, t \leq T_0} - \bm \gamma} = \bm 0
%     \end{gather*}

%     \begin{block}{Theorem 4.2}
%         If $\bm \gamma_0 = \bm \gamma_1 = \bm \gamma$, TWFE is sufficient for estimating $\bm \tau$.
%     \end{block}
% \end{frame}

%%%%%%%%%%%%%%%%%%%%%%%%%%%%%%%%%%%%%%%%%%%%%%%%%%%%%%%%%%%%%%%%%%%%%%%

\begin{frame}{Simulations}
    $y_{it} = \tau_{it}*d_{it} + \theta_t + \mu_i + t* \gamma_i +u_{it}$
\begin{itemize}
    \item $N = 200$, $T = 8$, $T_0 = 5$.
    \item $\theta_t, u_{it} \sim AR(1)$, $\rho = 0.75$.
    \item $\mu_i \sim N(0,4)$.
    \item $\gamma_i \sim N(\mu_i, 1)$.
    \item $\tau_{it} = \mu_i * \tau_t / 4$.
    \begin{itemize}
        \item $\tau_{T_0 + s} = s$. 
    \end{itemize}
    \item Treated probability: $\pi_i = 0.5 + \gamma_i / ( \max_i \gamma_i - \min_i \gamma_i )$.
    \item External instrument: $w_i = \gamma_i + N(0,1)$.
    \begin{itemize}
        \item See paper for varried signal-to-noise ratio.
    \end{itemize}
\end{itemize}
\end{frame}

%%%%%%%%%%%%%%%%%%%%%%%%%%%%%%%%%%%%%%%%%%%%%%%%%%%%%%%%%%%%%%%%%%%%%%%

\begin{frame}{Simulations: TWFE Model Holds}

\begin{table}
\def\arraystretch{1.25}
\label{tab:monte_results_pt}

\begin{adjustbox}{width=\textwidth}
\begin{threeparttable}
\begin{tabular}{@{} >{\RaggedRight}p{3.2cm} @{\extracolsep{4pt}}cccccc @{}} 
    % Head
    \toprule \addlinespace[3mm]
  
    \multicolumn{7}{@{}l}{
        \textbf{Panel A:} TWFE Model.
    } \\
    \midrule \addlinespace[3mm]
    
    & $\text{Bias}\ \big(\hat{\tau}_6 \big)$ & $\text{MSE}\ \big(\hat{\tau}_6\big)$
    & $\text{Bias}\ \big(\hat{\tau}_7 \big)$ & $\text{MSE}\ \big(\hat{\tau}_7\big)$
    & $\text{Bias}\ \big(\hat{\tau}_8 \big)$ & $\text{MSE}\ \big(\hat{\tau}_8\big)$ 
    \\
    \cmidrule{2-7}
    
    TWFE & 0.00 & 0.01 & -0.00 & 0.02 & 0.00 & 0.02 \\ 
    TWFE Imputation & 0.01 & 0.01 & 0.00 & 0.02 & 0.01 & 0.02 \\ 
    Factor Imputation & -0.00 & 0.04 & -0.01 & 0.11 & -0.01 & 0.24 \\ 


    
    
    
    \bottomrule
\end{tabular}
    
% Notes 
\begin{tablenotes}[flushleft] \footnotesize
    \item \textit{Notes.} This table presents a set of simulations with 10000 simulations. Each panel contains one of three data-generating processes described in the text. Each row in a panel consists of one of the four treatment effect estimators as described in the text. The columns report average bias and  mean-squared error for the three post-treatment treatment effects. 
\end{tablenotes}
\end{threeparttable}
\end{adjustbox}

\end{table}
    
\end{frame}

%%%%%%%%%%%%%%%%%%%%%%%%%%%%%%%%%%%%%%%%%%%%%%%%%%%%%%%%%%%%%%%%%%%%%%%

\begin{frame}{Simulations: Factor Model, TWFE is Consistent}

\begin{table}
\def\arraystretch{1.25}
\label{tab:monte_results_pt}

\begin{adjustbox}{width=\textwidth}
\begin{threeparttable}
\begin{tabular}{@{} >{\RaggedRight}p{3.2cm} @{\extracolsep{4pt}}cccccc @{}} 
    % Head


    \midrule \addlinespace[3mm]
    \multicolumn{7}{@{}l}{
        \textbf{Panel B:} Factor Model. Parallel Trends Hold
    } \\
    \midrule \addlinespace[3mm]
    
    & $\text{Bias}\ \big(\hat{\tau}_6 \big)$ & $\text{MSE}\ \big(\hat{\tau}_6\big)$
    & $\text{Bias}\ \big(\hat{\tau}_7 \big)$ & $\text{MSE}\ \big(\hat{\tau}_7\big)$ 
    & $\text{Bias}\ \big(\hat{\tau}_8 \big)$ & $\text{MSE}\ \big(\hat{\tau}_8\big)$ 
    \\
    \cmidrule{2-7}
    
    TWFE & 0.00 & 0.11 & 0.00 & 0.43 & 0.01 & 0.95 \\ 
    TWFE Imputation & 0.00 & 0.94 & 0.00 & 1.67 & 0.01 & 2.60 \\ 
    Factor Imputation & -0.00 & 0.02 & -0.00 & 0.03 & 0.00 & 0.05 \\ 

    
    
    
    \bottomrule
\end{tabular}
    
% Notes 
\begin{tablenotes}[flushleft] \footnotesize
    \item \textit{Notes.} This table presents a set of simulations with 10000 simulations. Each panel contains one of three data-generating processes described in the text. Each row in a panel consists of one of the four treatment effect estimators as described in the text. The columns report average bias and  mean-squared error for the three post-treatment treatment effects. 
\end{tablenotes}
\end{threeparttable}
\end{adjustbox}

\end{table}
    
\end{frame}

%%%%%%%%%%%%%%%%%%%%%%%%%%%%%%%%%%%%%%%%%%%%%%%%%%%%%%%%%%%%%%%%%%%%%%%

\begin{frame}{Simulations: Factor Model, TWFE is Inconsistent}

\begin{table}
\def\arraystretch{1.25}
\label{tab:monte_results_pt}

\begin{adjustbox}{width=\textwidth}
\begin{threeparttable}
\begin{tabular}{@{} >{\RaggedRight}p{3.2cm} @{\extracolsep{4pt}}cccccc @{}} 
    % Head

    \midrule \addlinespace[3mm]
    \multicolumn{7}{@{}l}{
        \textbf{Panel C:} Factor Model. Parallel Trends Do Not Hold
    } \\
    \midrule \addlinespace[3mm]
    
    & $\text{Bias}\ \big(\hat{\tau}_6 \big)$ & $\text{MSE}\ \big(\hat{\tau}_6\big)$
    & $\text{Bias}\ \big(\hat{\tau}_7 \big)$ & $\text{MSE}\ \big(\hat{\tau}_7\big)$ 
    & $\text{Bias}\ \big(\hat{\tau}_8 \big)$ & $\text{MSE}\ \big(\hat{\tau}_8\big)$ 
    \\
    \cmidrule{2-7}
    
    TWFE & -1.63 & 2.77 & -3.27 & 11.05 & -4.90 & 24.84 \\ 
    TWFE Imputation & -4.90 & 24.81 & -6.53 & 44.12 & -8.16 & 68.93 \\  
    Factor Imputation & 0.01 & 0.03 & 0.01 & 0.05 & 0.02 & 0.09 \\ 
    
    
    \bottomrule
\end{tabular}
    
% Notes 
\begin{tablenotes}[flushleft] \footnotesize
    \item \textit{Notes.} This table presents a set of simulations with 10000 simulations. Each panel contains one of three data-generating processes described in the text. Each row in a panel consists of one of the four treatment effect estimators as described in the text. The columns report average bias and  mean-squared error for the three post-treatment treatment effects. 
\end{tablenotes}
\end{threeparttable}
\end{adjustbox}

\end{table}
    
\end{frame}

%%%%%%%%%%%%%%%%%%%%%%%%%%%%%%%%%%%%%%%%%%%%%%%%%%%%%%%%%%%%%%%%%%%%%%%

\begin{frame}{Application Overview}
    We reevaluate Walmart openings on local labor markets. 

    \vspace{.5cm}

    \textbf{Volpe and Boland (2022):} Mixed results in empirical literature.
    \begin{itemize}
        \item \textbf{Basker (2005):} Small positive effect on retail employment.
        \item \textbf{Neumark et al. (2008):} Small negative effect on employment. 
    \end{itemize}

    \vspace{.5cm}

    Newmark et al. (2008) suggests Walmart opens stores based on local economic trajectories.
    \begin{itemize}
        \item Our assumption: trajectories are decomposed as local exposures to macro shocks (i.e. factor model).
    \end{itemize}
\end{frame}

%%%%%%%%%%%%%%%%%%%%%%%%%%%%%%%%%%%%%%%%%%%%%%%%%%%%%%%%%%%%%%%%%%%%%%%

\begin{frame}{Sample}
    County Business Patterns data set from 1977-1999. 
    \begin{itemize}
        \item We drop units treated at or before 1985.
        \begin{itemize}
            \item $T_0 = 9$.
        \end{itemize}
        \item Keep counties with $\geq 1500$ employees in 1964 and positive employment growth between 1964 and 1977.
        \item $d_{it} = 1$ if county $i$ has a Walmart in year $t$.
        \begin{itemize}
            \item Treatment group $g$ is first year county received a Walmart. 
            \item 82.4\% of counties receive $\leq 1$ Walmarts. 10.4\% receive $2$ Walmarts. 
        \end{itemize}
        \item $N = 1274$.
        \begin{itemize}
            \item Smaller than Basker (2005) by $500$ counties.
        \end{itemize}
        \item Interested in effects on retail and non-retail employment.
    \end{itemize}
\end{frame}

%%%%%%%%%%%%%%%%%%%%%%%%%%%%%%%%%%%%%%%%%%%%%%%%%%%%%%%%%%%%%%%%%%%%%%%

\begin{frame}{Estimation}
    TWFE imputation:
    \begin{equation}
        \log(y_{it}) = \mu_i + \theta_t + \sum_{\ell = -22}^{13} \tau^\ell d_{it}^\ell + e_{it}
    \end{equation}
    \begin{itemize}
        \item Wooldridge (2021), Borusyak et al. (2022). 
    \end{itemize}

    \vspace{.5cm}

    Factor model:
    \begin{itemize}
        \item Instruments: we use 1980 baseline shares of the population.
        \begin{itemize}
            \item Manufacturing employment, above/below poverty line, government employment, private sector employment. 
        \end{itemize}
        \item Set $p = 2$. 
        \begin{itemize}
            \item Justified via overidentifying restrictions test. 
        \end{itemize}
    \end{itemize}
\end{frame}

%%%%%%%%%%%%%%%%%%%%%%%%%%%%%%%%%%%%%%%%%%%%%%%%%%%%%%%%%%%%%%%%%%%%%%%

\begin{frame}{Results}
    \begin{figure}
\caption*{Effect of Walmart on County $\log$ Retail Employment}
\label{fig:walmart_retail}

\begin{subfigure}[b]{0.45\textwidth}
    \caption{TWFE Imputation Estimator}
    \begin{adjustbox}{width=\textwidth, center}
        % Recommended preamble:
% \usetikzlibrary{arrows.meta}
% \usetikzlibrary{backgrounds}
% \usepgfplotslibrary{patchplots}
% \usepgfplotslibrary{fillbetween}
% \pgfplotsset{%
%     layers/standard/.define layer set={%
%         background,axis background,axis grid,axis ticks,axis lines,axis tick labels,pre main,main,axis descriptions,axis foreground%
%     }{
%         grid style={/pgfplots/on layer=axis grid},%
%         tick style={/pgfplots/on layer=axis ticks},%
%         axis line style={/pgfplots/on layer=axis lines},%
%         label style={/pgfplots/on layer=axis descriptions},%
%         legend style={/pgfplots/on layer=axis descriptions},%
%         title style={/pgfplots/on layer=axis descriptions},%
%         colorbar style={/pgfplots/on layer=axis descriptions},%
%         ticklabel style={/pgfplots/on layer=axis tick labels},%
%         axis background@ style={/pgfplots/on layer=axis background},%
%         3d box foreground style={/pgfplots/on layer=axis foreground},%
%     },
% }

\begin{tikzpicture}[/tikz/background rectangle/.style={fill={rgb,1:red,1.0;green,1.0;blue,1.0}, draw opacity={1.0}}, show background rectangle]
\begin{axis}[point meta max={nan}, point meta min={nan}, legend cell align={left}, legend columns={1}, title={}, title style={at={{(0.5,1)}}, anchor={south}, font={{\fontsize{14 pt}{18.2 pt}\selectfont}}, color={rgb,1:red,0.0;green,0.0;blue,0.0}, draw opacity={1.0}, rotate={0.0}, align={center}}, legend style={color={rgb,1:red,0.0;green,0.0;blue,0.0}, draw opacity={1.0}, line width={1}, solid, fill={rgb,1:red,1.0;green,1.0;blue,1.0}, fill opacity={1.0}, text opacity={1.0}, font={{\fontsize{8 pt}{10.4 pt}\selectfont}}, text={rgb,1:red,0.0;green,0.0;blue,0.0}, cells={anchor={center}}, at={(1.02, 1)}, anchor={north west}}, axis background/.style={fill={rgb,1:red,1.0;green,1.0;blue,1.0}, opacity={1.0}}, anchor={north west}, xshift={1.0mm}, yshift={-1.0mm}, width={97.06mm}, height={64.04mm}, scaled x ticks={false}, xlabel={Event Time}, x tick style={color={rgb,1:red,0.0;green,0.0;blue,0.0}, opacity={1.0}}, x tick label style={color={rgb,1:red,0.0;green,0.0;blue,0.0}, opacity={1.0}, rotate={0}}, xlabel style={at={(ticklabel cs:0.5)}, anchor=near ticklabel, at={{(ticklabel cs:0.5)}}, anchor={near ticklabel}, font={{\fontsize{11 pt}{14.3 pt}\selectfont}}, color={rgb,1:red,0.0;green,0.0;blue,0.0}, draw opacity={1.0}, rotate={0.0}}, xmajorgrids={true}, xmin={-23.05}, xmax={14.05}, xticklabels={{$-20$,$-10$,$0$,$10$}}, xtick={{-20.0,-10.0,0.0,10.0}}, xtick align={inside}, xticklabel style={font={{\fontsize{8 pt}{10.4 pt}\selectfont}}, color={rgb,1:red,0.0;green,0.0;blue,0.0}, draw opacity={1.0}, rotate={0.0}}, x grid style={color={rgb,1:red,0.0;green,0.0;blue,0.0}, draw opacity={0.1}, line width={0.5}, solid}, axis x line*={left}, x axis line style={color={rgb,1:red,0.0;green,0.0;blue,0.0}, draw opacity={1.0}, line width={1}, solid}, scaled y ticks={false}, ylabel={Coefficient}, y tick style={color={rgb,1:red,0.0;green,0.0;blue,0.0}, opacity={1.0}}, y tick label style={color={rgb,1:red,0.0;green,0.0;blue,0.0}, opacity={1.0}, rotate={0}}, ylabel style={at={(ticklabel cs:0.5)}, anchor=near ticklabel, at={{(ticklabel cs:0.5)}}, anchor={near ticklabel}, font={{\fontsize{11 pt}{14.3 pt}\selectfont}}, color={rgb,1:red,0.0;green,0.0;blue,0.0}, draw opacity={1.0}, rotate={0.0}}, ymajorgrids={true}, ymin={-0.175}, ymax={0.3}, yticklabels={{$-0.1$,$0.0$,$0.1$,$0.2$}}, ytick={{-0.1,0.0,0.1,0.2}}, ytick align={inside}, yticklabel style={font={{\fontsize{8 pt}{10.4 pt}\selectfont}}, color={rgb,1:red,0.0;green,0.0;blue,0.0}, draw opacity={1.0}, rotate={0.0}}, y grid style={color={rgb,1:red,0.0;green,0.0;blue,0.0}, draw opacity={0.1}, line width={0.5}, solid}, axis y line*={left}, y axis line style={color={rgb,1:red,0.0;green,0.0;blue,0.0}, draw opacity={1.0}, line width={1}, solid}, colorbar={false}]
    \addplot[color={rgb,1:red,0.0;green,0.0;blue,0.0}, name path={51a38ef9-15e3-410f-9f6f-f87f9cf500db}, draw opacity={1.0}, line width={1}, dashed]
        table[row sep={\\}]
        {
            \\
            -60.150000000000006  0.0  \\
            51.150000000000006  0.0  \\
        }
        ;
    \addplot[color={rgb,1:red,0.6039;green,0.1412;blue,0.0824}, name path={c53d3c17-8c64-48f5-bd38-344e5ccac28d}, draw opacity={1.0}, line width={1}, solid]
        table[row sep={\\}]
        {
            \\
            -60.150000000000006  -0.17415708872706115  \\
            51.150000000000006  0.19940463725234964  \\
        }
        ;
    \addplot[color={rgb,1:red,0.6039;green,0.1412;blue,0.0824}, name path={42110b88-5d74-446e-b257-a8115795fad9}, draw opacity={1.0}, line width={1}, solid, mark={-}, mark size={1.5 pt}, mark repeat={1}, mark options={color={rgb,1:red,0.6039;green,0.1412;blue,0.0824}, draw opacity={1.0}, fill={rgb,1:red,0.6039;green,0.1412;blue,0.0824}, fill opacity={1.0}, line width={0.75}, rotate={0}, solid}]
        table[row sep={\\}]
        {
            \\
            -22.0  -0.158110234903843  \\
            -22.0  -0.019211167974375096  \\
        }
        ;
    \addplot[color={rgb,1:red,0.6039;green,0.1412;blue,0.0824}, name path={42110b88-5d74-446e-b257-a8115795fad9}, draw opacity={1.0}, line width={1}, solid, mark={-}, mark size={1.5 pt}, mark repeat={1}, mark options={color={rgb,1:red,0.6039;green,0.1412;blue,0.0824}, draw opacity={1.0}, fill={rgb,1:red,0.6039;green,0.1412;blue,0.0824}, fill opacity={1.0}, line width={0.75}, rotate={0}, solid}]
        table[row sep={\\}]
        {
            \\
            -21.0  -0.114389239792449  \\
            -21.0  -0.0209940132992332  \\
        }
        ;
    \addplot[color={rgb,1:red,0.6039;green,0.1412;blue,0.0824}, name path={42110b88-5d74-446e-b257-a8115795fad9}, draw opacity={1.0}, line width={1}, solid, mark={-}, mark size={1.5 pt}, mark repeat={1}, mark options={color={rgb,1:red,0.6039;green,0.1412;blue,0.0824}, draw opacity={1.0}, fill={rgb,1:red,0.6039;green,0.1412;blue,0.0824}, fill opacity={1.0}, line width={0.75}, rotate={0}, solid}]
        table[row sep={\\}]
        {
            \\
            -20.0  -0.10783083675818  \\
            -20.0  -0.0389028858470394  \\
        }
        ;
    \addplot[color={rgb,1:red,0.6039;green,0.1412;blue,0.0824}, name path={42110b88-5d74-446e-b257-a8115795fad9}, draw opacity={1.0}, line width={1}, solid, mark={-}, mark size={1.5 pt}, mark repeat={1}, mark options={color={rgb,1:red,0.6039;green,0.1412;blue,0.0824}, draw opacity={1.0}, fill={rgb,1:red,0.6039;green,0.1412;blue,0.0824}, fill opacity={1.0}, line width={0.75}, rotate={0}, solid}]
        table[row sep={\\}]
        {
            \\
            -19.0  -0.0906383093355555  \\
            -19.0  -0.0354708993518486  \\
        }
        ;
    \addplot[color={rgb,1:red,0.6039;green,0.1412;blue,0.0824}, name path={42110b88-5d74-446e-b257-a8115795fad9}, draw opacity={1.0}, line width={1}, solid, mark={-}, mark size={1.5 pt}, mark repeat={1}, mark options={color={rgb,1:red,0.6039;green,0.1412;blue,0.0824}, draw opacity={1.0}, fill={rgb,1:red,0.6039;green,0.1412;blue,0.0824}, fill opacity={1.0}, line width={0.75}, rotate={0}, solid}]
        table[row sep={\\}]
        {
            \\
            -18.0  -0.065134928944624  \\
            -18.0  -0.0244713332608457  \\
        }
        ;
    \addplot[color={rgb,1:red,0.6039;green,0.1412;blue,0.0824}, name path={42110b88-5d74-446e-b257-a8115795fad9}, draw opacity={1.0}, line width={1}, solid, mark={-}, mark size={1.5 pt}, mark repeat={1}, mark options={color={rgb,1:red,0.6039;green,0.1412;blue,0.0824}, draw opacity={1.0}, fill={rgb,1:red,0.6039;green,0.1412;blue,0.0824}, fill opacity={1.0}, line width={0.75}, rotate={0}, solid}]
        table[row sep={\\}]
        {
            \\
            -17.0  -0.0585971616101195  \\
            -17.0  -0.0275075044506591  \\
        }
        ;
    \addplot[color={rgb,1:red,0.6039;green,0.1412;blue,0.0824}, name path={42110b88-5d74-446e-b257-a8115795fad9}, draw opacity={1.0}, line width={1}, solid, mark={-}, mark size={1.5 pt}, mark repeat={1}, mark options={color={rgb,1:red,0.6039;green,0.1412;blue,0.0824}, draw opacity={1.0}, fill={rgb,1:red,0.6039;green,0.1412;blue,0.0824}, fill opacity={1.0}, line width={0.75}, rotate={0}, solid}]
        table[row sep={\\}]
        {
            \\
            -16.0  -0.0423406783276412  \\
            -16.0  -0.0192118562599878  \\
        }
        ;
    \addplot[color={rgb,1:red,0.6039;green,0.1412;blue,0.0824}, name path={42110b88-5d74-446e-b257-a8115795fad9}, draw opacity={1.0}, line width={1}, solid, mark={-}, mark size={1.5 pt}, mark repeat={1}, mark options={color={rgb,1:red,0.6039;green,0.1412;blue,0.0824}, draw opacity={1.0}, fill={rgb,1:red,0.6039;green,0.1412;blue,0.0824}, fill opacity={1.0}, line width={0.75}, rotate={0}, solid}]
        table[row sep={\\}]
        {
            \\
            -15.0  -0.0322874320263894  \\
            -15.0  -0.0128146726355074  \\
        }
        ;
    \addplot[color={rgb,1:red,0.6039;green,0.1412;blue,0.0824}, name path={42110b88-5d74-446e-b257-a8115795fad9}, draw opacity={1.0}, line width={1}, solid, mark={-}, mark size={1.5 pt}, mark repeat={1}, mark options={color={rgb,1:red,0.6039;green,0.1412;blue,0.0824}, draw opacity={1.0}, fill={rgb,1:red,0.6039;green,0.1412;blue,0.0824}, fill opacity={1.0}, line width={0.75}, rotate={0}, solid}]
        table[row sep={\\}]
        {
            \\
            -14.0  -0.0260061672680235  \\
            -14.0  -0.00873613324588016  \\
        }
        ;
    \addplot[color={rgb,1:red,0.6039;green,0.1412;blue,0.0824}, name path={42110b88-5d74-446e-b257-a8115795fad9}, draw opacity={1.0}, line width={1}, solid, mark={-}, mark size={1.5 pt}, mark repeat={1}, mark options={color={rgb,1:red,0.6039;green,0.1412;blue,0.0824}, draw opacity={1.0}, fill={rgb,1:red,0.6039;green,0.1412;blue,0.0824}, fill opacity={1.0}, line width={0.75}, rotate={0}, solid}]
        table[row sep={\\}]
        {
            \\
            -13.0  -0.0241938692763972  \\
            -13.0  -0.00786523662571666  \\
        }
        ;
    \addplot[color={rgb,1:red,0.6039;green,0.1412;blue,0.0824}, name path={42110b88-5d74-446e-b257-a8115795fad9}, draw opacity={1.0}, line width={1}, solid, mark={-}, mark size={1.5 pt}, mark repeat={1}, mark options={color={rgb,1:red,0.6039;green,0.1412;blue,0.0824}, draw opacity={1.0}, fill={rgb,1:red,0.6039;green,0.1412;blue,0.0824}, fill opacity={1.0}, line width={0.75}, rotate={0}, solid}]
        table[row sep={\\}]
        {
            \\
            -12.0  -0.0202579112441287  \\
            -12.0  -0.00771619484042468  \\
        }
        ;
    \addplot[color={rgb,1:red,0.6039;green,0.1412;blue,0.0824}, name path={42110b88-5d74-446e-b257-a8115795fad9}, draw opacity={1.0}, line width={1}, solid, mark={-}, mark size={1.5 pt}, mark repeat={1}, mark options={color={rgb,1:red,0.6039;green,0.1412;blue,0.0824}, draw opacity={1.0}, fill={rgb,1:red,0.6039;green,0.1412;blue,0.0824}, fill opacity={1.0}, line width={0.75}, rotate={0}, solid}]
        table[row sep={\\}]
        {
            \\
            -11.0  -0.0173743966162851  \\
            -11.0  -0.00593903136988103  \\
        }
        ;
    \addplot[color={rgb,1:red,0.6039;green,0.1412;blue,0.0824}, name path={42110b88-5d74-446e-b257-a8115795fad9}, draw opacity={1.0}, line width={1}, solid, mark={-}, mark size={1.5 pt}, mark repeat={1}, mark options={color={rgb,1:red,0.6039;green,0.1412;blue,0.0824}, draw opacity={1.0}, fill={rgb,1:red,0.6039;green,0.1412;blue,0.0824}, fill opacity={1.0}, line width={0.75}, rotate={0}, solid}]
        table[row sep={\\}]
        {
            \\
            -10.0  -0.0129207673268064  \\
            -10.0  -0.0008315899863417034  \\
        }
        ;
    \addplot[color={rgb,1:red,0.6039;green,0.1412;blue,0.0824}, name path={42110b88-5d74-446e-b257-a8115795fad9}, draw opacity={1.0}, line width={1}, solid, mark={-}, mark size={1.5 pt}, mark repeat={1}, mark options={color={rgb,1:red,0.6039;green,0.1412;blue,0.0824}, draw opacity={1.0}, fill={rgb,1:red,0.6039;green,0.1412;blue,0.0824}, fill opacity={1.0}, line width={0.75}, rotate={0}, solid}]
        table[row sep={\\}]
        {
            \\
            -9.0  -0.0109174199413242  \\
            -9.0  0.0005081179321002761  \\
        }
        ;
    \addplot[color={rgb,1:red,0.6039;green,0.1412;blue,0.0824}, name path={42110b88-5d74-446e-b257-a8115795fad9}, draw opacity={1.0}, line width={1}, solid, mark={-}, mark size={1.5 pt}, mark repeat={1}, mark options={color={rgb,1:red,0.6039;green,0.1412;blue,0.0824}, draw opacity={1.0}, fill={rgb,1:red,0.6039;green,0.1412;blue,0.0824}, fill opacity={1.0}, line width={0.75}, rotate={0}, solid}]
        table[row sep={\\}]
        {
            \\
            -8.0  -0.00478049093318894  \\
            -8.0  0.00585854133089775  \\
        }
        ;
    \addplot[color={rgb,1:red,0.6039;green,0.1412;blue,0.0824}, name path={42110b88-5d74-446e-b257-a8115795fad9}, draw opacity={1.0}, line width={1}, solid, mark={-}, mark size={1.5 pt}, mark repeat={1}, mark options={color={rgb,1:red,0.6039;green,0.1412;blue,0.0824}, draw opacity={1.0}, fill={rgb,1:red,0.6039;green,0.1412;blue,0.0824}, fill opacity={1.0}, line width={0.75}, rotate={0}, solid}]
        table[row sep={\\}]
        {
            \\
            -7.0  0.0007732806752649276  \\
            -7.0  0.0106542598505913  \\
        }
        ;
    \addplot[color={rgb,1:red,0.6039;green,0.1412;blue,0.0824}, name path={42110b88-5d74-446e-b257-a8115795fad9}, draw opacity={1.0}, line width={1}, solid, mark={-}, mark size={1.5 pt}, mark repeat={1}, mark options={color={rgb,1:red,0.6039;green,0.1412;blue,0.0824}, draw opacity={1.0}, fill={rgb,1:red,0.6039;green,0.1412;blue,0.0824}, fill opacity={1.0}, line width={0.75}, rotate={0}, solid}]
        table[row sep={\\}]
        {
            \\
            -6.0  0.00283525583785008  \\
            -6.0  0.0123764017049996  \\
        }
        ;
    \addplot[color={rgb,1:red,0.6039;green,0.1412;blue,0.0824}, name path={42110b88-5d74-446e-b257-a8115795fad9}, draw opacity={1.0}, line width={1}, solid, mark={-}, mark size={1.5 pt}, mark repeat={1}, mark options={color={rgb,1:red,0.6039;green,0.1412;blue,0.0824}, draw opacity={1.0}, fill={rgb,1:red,0.6039;green,0.1412;blue,0.0824}, fill opacity={1.0}, line width={0.75}, rotate={0}, solid}]
        table[row sep={\\}]
        {
            \\
            -5.0  0.00801203699657077  \\
            -5.0  0.0179528333993491  \\
        }
        ;
    \addplot[color={rgb,1:red,0.6039;green,0.1412;blue,0.0824}, name path={42110b88-5d74-446e-b257-a8115795fad9}, draw opacity={1.0}, line width={1}, solid, mark={-}, mark size={1.5 pt}, mark repeat={1}, mark options={color={rgb,1:red,0.6039;green,0.1412;blue,0.0824}, draw opacity={1.0}, fill={rgb,1:red,0.6039;green,0.1412;blue,0.0824}, fill opacity={1.0}, line width={0.75}, rotate={0}, solid}]
        table[row sep={\\}]
        {
            \\
            -4.0  0.0134782496678425  \\
            -4.0  0.0236031144623834  \\
        }
        ;
    \addplot[color={rgb,1:red,0.6039;green,0.1412;blue,0.0824}, name path={42110b88-5d74-446e-b257-a8115795fad9}, draw opacity={1.0}, line width={1}, solid, mark={-}, mark size={1.5 pt}, mark repeat={1}, mark options={color={rgb,1:red,0.6039;green,0.1412;blue,0.0824}, draw opacity={1.0}, fill={rgb,1:red,0.6039;green,0.1412;blue,0.0824}, fill opacity={1.0}, line width={0.75}, rotate={0}, solid}]
        table[row sep={\\}]
        {
            \\
            -3.0  0.013045876474248  \\
            -3.0  0.0244617162360145  \\
        }
        ;
    \addplot[color={rgb,1:red,0.6039;green,0.1412;blue,0.0824}, name path={42110b88-5d74-446e-b257-a8115795fad9}, draw opacity={1.0}, line width={1}, solid, mark={-}, mark size={1.5 pt}, mark repeat={1}, mark options={color={rgb,1:red,0.6039;green,0.1412;blue,0.0824}, draw opacity={1.0}, fill={rgb,1:red,0.6039;green,0.1412;blue,0.0824}, fill opacity={1.0}, line width={0.75}, rotate={0}, solid}]
        table[row sep={\\}]
        {
            \\
            -2.0  0.0130402512262549  \\
            -2.0  0.0270265756420808  \\
        }
        ;
    \addplot[color={rgb,1:red,0.6039;green,0.1412;blue,0.0824}, name path={42110b88-5d74-446e-b257-a8115795fad9}, draw opacity={1.0}, line width={1}, solid, mark={-}, mark size={1.5 pt}, mark repeat={1}, mark options={color={rgb,1:red,0.6039;green,0.1412;blue,0.0824}, draw opacity={1.0}, fill={rgb,1:red,0.6039;green,0.1412;blue,0.0824}, fill opacity={1.0}, line width={0.75}, rotate={0}, solid}]
        table[row sep={\\}]
        {
            \\
            -1.0  0.0125100079152404  \\
            -1.0  0.0296850281158563  \\
        }
        ;
    \addplot[color={rgb,1:red,0.2422;green,0.6433;blue,0.3044}, name path={486f81bd-43a2-42f6-8303-98bd9e811cb6}, only marks, draw opacity={1.0}, line width={0}, solid, mark={*}, mark size={1.5 pt}, mark repeat={1}, mark options={color={rgb,1:red,0.6039;green,0.1412;blue,0.0824}, draw opacity={1.0}, fill={rgb,1:red,0.6039;green,0.1412;blue,0.0824}, fill opacity={1.0}, line width={0.75}, rotate={0}, solid}]
        table[row sep={\\}]
        {
            \\
            -22.0  -0.0902893946864491  \\
            -21.0  -0.0672351572717881  \\
            -20.0  -0.071878150059426  \\
            -19.0  -0.0626020621487759  \\
            -18.0  -0.0445772040023762  \\
            -17.0  -0.0421962837394736  \\
            -16.0  -0.0306272140213823  \\
            -15.0  -0.0225078023435874  \\
            -14.0  -0.0169924485706699  \\
            -13.0  -0.0160995021618369  \\
            -12.0  -0.0136641060991802  \\
            -11.0  -0.0114701867317889  \\
            -10.0  -0.00645183127639015  \\
            -9.0  -0.00490453668791194  \\
            -8.0  0.000625032852838539  \\
            -7.0  0.00589125064296933  \\
            -6.0  0.00784283846774344  \\
            -5.0  0.0131498027309412  \\
            -4.0  0.0184004515725384  \\
            -3.0  0.0185964028165947  \\
            -2.0  0.0198444435377969  \\
            -1.0  0.0208884438785636  \\
        }
        ;
    \addplot[color={rgb,1:red,0.0627;green,0.4706;blue,0.5843}, name path={448c04a9-31ec-4ecc-9ba9-3b69cd8fc29b}, draw opacity={1.0}, line width={1}, solid, mark={-}, mark size={1.5 pt}, mark repeat={1}, mark options={color={rgb,1:red,0.0627;green,0.4706;blue,0.5843}, draw opacity={1.0}, fill={rgb,1:red,0.0627;green,0.4706;blue,0.5843}, fill opacity={1.0}, line width={0.75}, rotate={0}, solid}]
        table[row sep={\\}]
        {
            \\
            0.0  0.0349640398444361  \\
            0.0  0.0565667443749788  \\
        }
        ;
    \addplot[color={rgb,1:red,0.0627;green,0.4706;blue,0.5843}, name path={448c04a9-31ec-4ecc-9ba9-3b69cd8fc29b}, draw opacity={1.0}, line width={1}, solid, mark={-}, mark size={1.5 pt}, mark repeat={1}, mark options={color={rgb,1:red,0.0627;green,0.4706;blue,0.5843}, draw opacity={1.0}, fill={rgb,1:red,0.0627;green,0.4706;blue,0.5843}, fill opacity={1.0}, line width={0.75}, rotate={0}, solid}]
        table[row sep={\\}]
        {
            \\
            1.0  0.0652232220069012  \\
            1.0  0.089539603941482  \\
        }
        ;
    \addplot[color={rgb,1:red,0.0627;green,0.4706;blue,0.5843}, name path={448c04a9-31ec-4ecc-9ba9-3b69cd8fc29b}, draw opacity={1.0}, line width={1}, solid, mark={-}, mark size={1.5 pt}, mark repeat={1}, mark options={color={rgb,1:red,0.0627;green,0.4706;blue,0.5843}, draw opacity={1.0}, fill={rgb,1:red,0.0627;green,0.4706;blue,0.5843}, fill opacity={1.0}, line width={0.75}, rotate={0}, solid}]
        table[row sep={\\}]
        {
            \\
            2.0  0.0636713767635303  \\
            2.0  0.0923625187831656  \\
        }
        ;
    \addplot[color={rgb,1:red,0.0627;green,0.4706;blue,0.5843}, name path={448c04a9-31ec-4ecc-9ba9-3b69cd8fc29b}, draw opacity={1.0}, line width={1}, solid, mark={-}, mark size={1.5 pt}, mark repeat={1}, mark options={color={rgb,1:red,0.0627;green,0.4706;blue,0.5843}, draw opacity={1.0}, fill={rgb,1:red,0.0627;green,0.4706;blue,0.5843}, fill opacity={1.0}, line width={0.75}, rotate={0}, solid}]
        table[row sep={\\}]
        {
            \\
            3.0  0.0630429413143631  \\
            3.0  0.0967074361109923  \\
        }
        ;
    \addplot[color={rgb,1:red,0.0627;green,0.4706;blue,0.5843}, name path={448c04a9-31ec-4ecc-9ba9-3b69cd8fc29b}, draw opacity={1.0}, line width={1}, solid, mark={-}, mark size={1.5 pt}, mark repeat={1}, mark options={color={rgb,1:red,0.0627;green,0.4706;blue,0.5843}, draw opacity={1.0}, fill={rgb,1:red,0.0627;green,0.4706;blue,0.5843}, fill opacity={1.0}, line width={0.75}, rotate={0}, solid}]
        table[row sep={\\}]
        {
            \\
            4.0  0.0661334337225134  \\
            4.0  0.104887251059072  \\
        }
        ;
    \addplot[color={rgb,1:red,0.0627;green,0.4706;blue,0.5843}, name path={448c04a9-31ec-4ecc-9ba9-3b69cd8fc29b}, draw opacity={1.0}, line width={1}, solid, mark={-}, mark size={1.5 pt}, mark repeat={1}, mark options={color={rgb,1:red,0.0627;green,0.4706;blue,0.5843}, draw opacity={1.0}, fill={rgb,1:red,0.0627;green,0.4706;blue,0.5843}, fill opacity={1.0}, line width={0.75}, rotate={0}, solid}]
        table[row sep={\\}]
        {
            \\
            5.0  0.0757383292012583  \\
            5.0  0.118096101888863  \\
        }
        ;
    \addplot[color={rgb,1:red,0.0627;green,0.4706;blue,0.5843}, name path={448c04a9-31ec-4ecc-9ba9-3b69cd8fc29b}, draw opacity={1.0}, line width={1}, solid, mark={-}, mark size={1.5 pt}, mark repeat={1}, mark options={color={rgb,1:red,0.0627;green,0.4706;blue,0.5843}, draw opacity={1.0}, fill={rgb,1:red,0.0627;green,0.4706;blue,0.5843}, fill opacity={1.0}, line width={0.75}, rotate={0}, solid}]
        table[row sep={\\}]
        {
            \\
            6.0  0.0851504702956824  \\
            6.0  0.131678165211638  \\
        }
        ;
    \addplot[color={rgb,1:red,0.0627;green,0.4706;blue,0.5843}, name path={448c04a9-31ec-4ecc-9ba9-3b69cd8fc29b}, draw opacity={1.0}, line width={1}, solid, mark={-}, mark size={1.5 pt}, mark repeat={1}, mark options={color={rgb,1:red,0.0627;green,0.4706;blue,0.5843}, draw opacity={1.0}, fill={rgb,1:red,0.0627;green,0.4706;blue,0.5843}, fill opacity={1.0}, line width={0.75}, rotate={0}, solid}]
        table[row sep={\\}]
        {
            \\
            7.0  0.100059390457104  \\
            7.0  0.151735629440466  \\
        }
        ;
    \addplot[color={rgb,1:red,0.0627;green,0.4706;blue,0.5843}, name path={448c04a9-31ec-4ecc-9ba9-3b69cd8fc29b}, draw opacity={1.0}, line width={1}, solid, mark={-}, mark size={1.5 pt}, mark repeat={1}, mark options={color={rgb,1:red,0.0627;green,0.4706;blue,0.5843}, draw opacity={1.0}, fill={rgb,1:red,0.0627;green,0.4706;blue,0.5843}, fill opacity={1.0}, line width={0.75}, rotate={0}, solid}]
        table[row sep={\\}]
        {
            \\
            8.0  0.108076012181693  \\
            8.0  0.165310703644478  \\
        }
        ;
    \addplot[color={rgb,1:red,0.0627;green,0.4706;blue,0.5843}, name path={448c04a9-31ec-4ecc-9ba9-3b69cd8fc29b}, draw opacity={1.0}, line width={1}, solid, mark={-}, mark size={1.5 pt}, mark repeat={1}, mark options={color={rgb,1:red,0.0627;green,0.4706;blue,0.5843}, draw opacity={1.0}, fill={rgb,1:red,0.0627;green,0.4706;blue,0.5843}, fill opacity={1.0}, line width={0.75}, rotate={0}, solid}]
        table[row sep={\\}]
        {
            \\
            9.0  0.106880004731766  \\
            9.0  0.16881461451368  \\
        }
        ;
    \addplot[color={rgb,1:red,0.0627;green,0.4706;blue,0.5843}, name path={448c04a9-31ec-4ecc-9ba9-3b69cd8fc29b}, draw opacity={1.0}, line width={1}, solid, mark={-}, mark size={1.5 pt}, mark repeat={1}, mark options={color={rgb,1:red,0.0627;green,0.4706;blue,0.5843}, draw opacity={1.0}, fill={rgb,1:red,0.0627;green,0.4706;blue,0.5843}, fill opacity={1.0}, line width={0.75}, rotate={0}, solid}]
        table[row sep={\\}]
        {
            \\
            10.0  0.113815939423889  \\
            10.0  0.191326845237213  \\
        }
        ;
    \addplot[color={rgb,1:red,0.0627;green,0.4706;blue,0.5843}, name path={448c04a9-31ec-4ecc-9ba9-3b69cd8fc29b}, draw opacity={1.0}, line width={1}, solid, mark={-}, mark size={1.5 pt}, mark repeat={1}, mark options={color={rgb,1:red,0.0627;green,0.4706;blue,0.5843}, draw opacity={1.0}, fill={rgb,1:red,0.0627;green,0.4706;blue,0.5843}, fill opacity={1.0}, line width={0.75}, rotate={0}, solid}]
        table[row sep={\\}]
        {
            \\
            11.0  0.107560865892189  \\
            11.0  0.201952242196239  \\
        }
        ;
    \addplot[color={rgb,1:red,0.0627;green,0.4706;blue,0.5843}, name path={448c04a9-31ec-4ecc-9ba9-3b69cd8fc29b}, draw opacity={1.0}, line width={1}, solid, mark={-}, mark size={1.5 pt}, mark repeat={1}, mark options={color={rgb,1:red,0.0627;green,0.4706;blue,0.5843}, draw opacity={1.0}, fill={rgb,1:red,0.0627;green,0.4706;blue,0.5843}, fill opacity={1.0}, line width={0.75}, rotate={0}, solid}]
        table[row sep={\\}]
        {
            \\
            12.0  0.0957206816203397  \\
            12.0  0.216277050269159  \\
        }
        ;
    \addplot[color={rgb,1:red,0.0627;green,0.4706;blue,0.5843}, name path={448c04a9-31ec-4ecc-9ba9-3b69cd8fc29b}, draw opacity={1.0}, line width={1}, solid, mark={-}, mark size={1.5 pt}, mark repeat={1}, mark options={color={rgb,1:red,0.0627;green,0.4706;blue,0.5843}, draw opacity={1.0}, fill={rgb,1:red,0.0627;green,0.4706;blue,0.5843}, fill opacity={1.0}, line width={0.75}, rotate={0}, solid}]
        table[row sep={\\}]
        {
            \\
            13.0  0.0962577351427004  \\
            13.0  0.275745177396092  \\
        }
        ;
    \addplot[color={rgb,1:red,0.7644;green,0.4441;blue,0.8243}, name path={c7a5c632-7838-4cf4-a2af-a56ff0e3e32f}, only marks, draw opacity={1.0}, line width={0}, solid, mark={*}, mark size={1.5 pt}, mark repeat={1}, mark options={color={rgb,1:red,0.0627;green,0.4706;blue,0.5843}, draw opacity={1.0}, fill={rgb,1:red,0.0627;green,0.4706;blue,0.5843}, fill opacity={1.0}, line width={0.75}, rotate={0}, solid}]
        table[row sep={\\}]
        {
            \\
            0.0  0.0451289007737432  \\
            1.0  0.0767471264240398  \\
            2.0  0.0773130953713352  \\
            3.0  0.0787876309916187  \\
            4.0  0.0843090404991735  \\
            5.0  0.0960169283977269  \\
            6.0  0.108178132452275  \\
            7.0  0.125682348932943  \\
            8.0  0.136469790119065  \\
            9.0  0.137882401069387  \\
            10.0  0.152978263537319  \\
            11.0  0.156125762740464  \\
            12.0  0.155102418154466  \\
            13.0  0.18721958280233  \\
        }
        ;
\end{axis}
\end{tikzpicture}

    \end{adjustbox}
\end{subfigure}
\hfill
\begin{subfigure}[b]{0.45\textwidth}
    \caption{Factor Imputation Estimator}
    \begin{adjustbox}{width=\textwidth, center}
        % Recommended preamble:
% \usetikzlibrary{arrows.meta}
% \usetikzlibrary{backgrounds}
% \usepgfplotslibrary{patchplots}
% \usepgfplotslibrary{fillbetween}
% \pgfplotsset{%
%     layers/standard/.define layer set={%
%         background,axis background,axis grid,axis ticks,axis lines,axis tick labels,pre main,main,axis descriptions,axis foreground%
%     }{
%         grid style={/pgfplots/on layer=axis grid},%
%         tick style={/pgfplots/on layer=axis ticks},%
%         axis line style={/pgfplots/on layer=axis lines},%
%         label style={/pgfplots/on layer=axis descriptions},%
%         legend style={/pgfplots/on layer=axis descriptions},%
%         title style={/pgfplots/on layer=axis descriptions},%
%         colorbar style={/pgfplots/on layer=axis descriptions},%
%         ticklabel style={/pgfplots/on layer=axis tick labels},%
%         axis background@ style={/pgfplots/on layer=axis background},%
%         3d box foreground style={/pgfplots/on layer=axis foreground},%
%     },
% }

\begin{tikzpicture}[/tikz/background rectangle/.style={fill={rgb,1:red,1.0;green,1.0;blue,1.0}, draw opacity={1.0}}, show background rectangle]
\begin{axis}[point meta max={nan}, point meta min={nan}, legend cell align={left}, legend columns={1}, title={}, title style={at={{(0.5,1)}}, anchor={south}, font={{\fontsize{14 pt}{18.2 pt}\selectfont}}, color={rgb,1:red,0.0;green,0.0;blue,0.0}, draw opacity={1.0}, rotate={0.0}, align={center}}, legend style={color={rgb,1:red,0.0;green,0.0;blue,0.0}, draw opacity={1.0}, line width={1}, solid, fill={rgb,1:red,1.0;green,1.0;blue,1.0}, fill opacity={1.0}, text opacity={1.0}, font={{\fontsize{8 pt}{10.4 pt}\selectfont}}, text={rgb,1:red,0.0;green,0.0;blue,0.0}, cells={anchor={center}}, at={(1.02, 1)}, anchor={north west}}, axis background/.style={fill={rgb,1:red,1.0;green,1.0;blue,1.0}, opacity={1.0}}, anchor={north west}, xshift={1.0mm}, yshift={-1.0mm}, width={97.06mm}, height={64.04mm}, scaled x ticks={false}, xlabel={Event Time}, x tick style={color={rgb,1:red,0.0;green,0.0;blue,0.0}, opacity={1.0}}, x tick label style={color={rgb,1:red,0.0;green,0.0;blue,0.0}, opacity={1.0}, rotate={0}}, xlabel style={at={(ticklabel cs:0.5)}, anchor=near ticklabel, at={{(ticklabel cs:0.5)}}, anchor={near ticklabel}, font={{\fontsize{11 pt}{14.3 pt}\selectfont}}, color={rgb,1:red,0.0;green,0.0;blue,0.0}, draw opacity={1.0}, rotate={0.0}}, xmajorgrids={true}, xmin={-23.05}, xmax={14.05}, xticklabels={{$-20$,$-10$,$0$,$10$}}, xtick={{-20.0,-10.0,0.0,10.0}}, xtick align={inside}, xticklabel style={font={{\fontsize{8 pt}{10.4 pt}\selectfont}}, color={rgb,1:red,0.0;green,0.0;blue,0.0}, draw opacity={1.0}, rotate={0.0}}, x grid style={color={rgb,1:red,0.0;green,0.0;blue,0.0}, draw opacity={0.1}, line width={0.5}, solid}, axis x line*={left}, x axis line style={color={rgb,1:red,0.0;green,0.0;blue,0.0}, draw opacity={1.0}, line width={1}, solid}, scaled y ticks={false}, ylabel={Coefficient}, y tick style={color={rgb,1:red,0.0;green,0.0;blue,0.0}, opacity={1.0}}, y tick label style={color={rgb,1:red,0.0;green,0.0;blue,0.0}, opacity={1.0}, rotate={0}}, ylabel style={at={(ticklabel cs:0.5)}, anchor=near ticklabel, at={{(ticklabel cs:0.5)}}, anchor={near ticklabel}, font={{\fontsize{11 pt}{14.3 pt}\selectfont}}, color={rgb,1:red,0.0;green,0.0;blue,0.0}, draw opacity={1.0}, rotate={0.0}}, ymajorgrids={true}, ymin={-0.175}, ymax={0.3}, yticklabels={{$-0.1$,$0.0$,$0.1$,$0.2$}}, ytick={{-0.1,0.0,0.1,0.2}}, ytick align={inside}, yticklabel style={font={{\fontsize{8 pt}{10.4 pt}\selectfont}}, color={rgb,1:red,0.0;green,0.0;blue,0.0}, draw opacity={1.0}, rotate={0.0}}, y grid style={color={rgb,1:red,0.0;green,0.0;blue,0.0}, draw opacity={0.1}, line width={0.5}, solid}, axis y line*={left}, y axis line style={color={rgb,1:red,0.0;green,0.0;blue,0.0}, draw opacity={1.0}, line width={1}, solid}, colorbar={false}]
    \addplot[color={rgb,1:red,0.0;green,0.0;blue,0.0}, name path={f06dcf78-f7b9-455a-a128-631b98bdd965}, draw opacity={1.0}, line width={1}, dashed]
        table[row sep={\\}]
        {
            \\
            -60.150000000000006  0.0  \\
            51.150000000000006  0.0  \\
        }
        ;
    \addplot[color={rgb,1:red,0.6039;green,0.1412;blue,0.0824}, name path={65e72695-3ca4-4097-8c4a-59ee32d9c666}, draw opacity={1.0}, line width={1}, solid]
        table[row sep={\\}]
        {
            \\
            51.150000000000006  -0.0038190758630880775  \\
            -60.150000000000006  0.004194391706156057  \\
        }
        ;
    \addplot[color={rgb,1:red,0.6039;green,0.1412;blue,0.0824}, name path={9b664ad0-e3fd-4b70-ac25-73aff5962437}, draw opacity={1.0}, line width={1}, solid, mark={-}, mark size={1.5 pt}, mark repeat={1}, mark options={color={rgb,1:red,0.6039;green,0.1412;blue,0.0824}, draw opacity={1.0}, fill={rgb,1:red,0.6039;green,0.1412;blue,0.0824}, fill opacity={1.0}, line width={0.75}, rotate={0}, solid}]
        table[row sep={\\}]
        {
            \\
            -22.0  -0.0686188595102608  \\
            -22.0  0.011569682459961289  \\
        }
        ;
    \addplot[color={rgb,1:red,0.6039;green,0.1412;blue,0.0824}, name path={9b664ad0-e3fd-4b70-ac25-73aff5962437}, draw opacity={1.0}, line width={1}, solid, mark={-}, mark size={1.5 pt}, mark repeat={1}, mark options={color={rgb,1:red,0.6039;green,0.1412;blue,0.0824}, draw opacity={1.0}, fill={rgb,1:red,0.6039;green,0.1412;blue,0.0824}, fill opacity={1.0}, line width={0.75}, rotate={0}, solid}]
        table[row sep={\\}]
        {
            \\
            -21.0  -0.05673736835699916  \\
            -21.0  0.0030807315805880774  \\
        }
        ;
    \addplot[color={rgb,1:red,0.6039;green,0.1412;blue,0.0824}, name path={9b664ad0-e3fd-4b70-ac25-73aff5962437}, draw opacity={1.0}, line width={1}, solid, mark={-}, mark size={1.5 pt}, mark repeat={1}, mark options={color={rgb,1:red,0.6039;green,0.1412;blue,0.0824}, draw opacity={1.0}, fill={rgb,1:red,0.6039;green,0.1412;blue,0.0824}, fill opacity={1.0}, line width={0.75}, rotate={0}, solid}]
        table[row sep={\\}]
        {
            \\
            -20.0  -0.032148996832228806  \\
            -20.0  0.01629453040202141  \\
        }
        ;
    \addplot[color={rgb,1:red,0.6039;green,0.1412;blue,0.0824}, name path={9b664ad0-e3fd-4b70-ac25-73aff5962437}, draw opacity={1.0}, line width={1}, solid, mark={-}, mark size={1.5 pt}, mark repeat={1}, mark options={color={rgb,1:red,0.6039;green,0.1412;blue,0.0824}, draw opacity={1.0}, fill={rgb,1:red,0.6039;green,0.1412;blue,0.0824}, fill opacity={1.0}, line width={0.75}, rotate={0}, solid}]
        table[row sep={\\}]
        {
            \\
            -19.0  -0.025926717899848375  \\
            -19.0  0.01600246266708315  \\
        }
        ;
    \addplot[color={rgb,1:red,0.6039;green,0.1412;blue,0.0824}, name path={9b664ad0-e3fd-4b70-ac25-73aff5962437}, draw opacity={1.0}, line width={1}, solid, mark={-}, mark size={1.5 pt}, mark repeat={1}, mark options={color={rgb,1:red,0.6039;green,0.1412;blue,0.0824}, draw opacity={1.0}, fill={rgb,1:red,0.6039;green,0.1412;blue,0.0824}, fill opacity={1.0}, line width={0.75}, rotate={0}, solid}]
        table[row sep={\\}]
        {
            \\
            -18.0  -0.016047566298418418  \\
            -18.0  0.015315541563718258  \\
        }
        ;
    \addplot[color={rgb,1:red,0.6039;green,0.1412;blue,0.0824}, name path={9b664ad0-e3fd-4b70-ac25-73aff5962437}, draw opacity={1.0}, line width={1}, solid, mark={-}, mark size={1.5 pt}, mark repeat={1}, mark options={color={rgb,1:red,0.6039;green,0.1412;blue,0.0824}, draw opacity={1.0}, fill={rgb,1:red,0.6039;green,0.1412;blue,0.0824}, fill opacity={1.0}, line width={0.75}, rotate={0}, solid}]
        table[row sep={\\}]
        {
            \\
            -17.0  -0.01519099930368875  \\
            -17.0  0.00893732520193499  \\
        }
        ;
    \addplot[color={rgb,1:red,0.6039;green,0.1412;blue,0.0824}, name path={9b664ad0-e3fd-4b70-ac25-73aff5962437}, draw opacity={1.0}, line width={1}, solid, mark={-}, mark size={1.5 pt}, mark repeat={1}, mark options={color={rgb,1:red,0.6039;green,0.1412;blue,0.0824}, draw opacity={1.0}, fill={rgb,1:red,0.6039;green,0.1412;blue,0.0824}, fill opacity={1.0}, line width={0.75}, rotate={0}, solid}]
        table[row sep={\\}]
        {
            \\
            -16.0  -0.0026609065884977507  \\
            -16.0  0.018039515753349788  \\
        }
        ;
    \addplot[color={rgb,1:red,0.6039;green,0.1412;blue,0.0824}, name path={9b664ad0-e3fd-4b70-ac25-73aff5962437}, draw opacity={1.0}, line width={1}, solid, mark={-}, mark size={1.5 pt}, mark repeat={1}, mark options={color={rgb,1:red,0.6039;green,0.1412;blue,0.0824}, draw opacity={1.0}, fill={rgb,1:red,0.6039;green,0.1412;blue,0.0824}, fill opacity={1.0}, line width={0.75}, rotate={0}, solid}]
        table[row sep={\\}]
        {
            \\
            -15.0  -0.0033135195362712488  \\
            -15.0  0.01355826706791583  \\
        }
        ;
    \addplot[color={rgb,1:red,0.6039;green,0.1412;blue,0.0824}, name path={9b664ad0-e3fd-4b70-ac25-73aff5962437}, draw opacity={1.0}, line width={1}, solid, mark={-}, mark size={1.5 pt}, mark repeat={1}, mark options={color={rgb,1:red,0.6039;green,0.1412;blue,0.0824}, draw opacity={1.0}, fill={rgb,1:red,0.6039;green,0.1412;blue,0.0824}, fill opacity={1.0}, line width={0.75}, rotate={0}, solid}]
        table[row sep={\\}]
        {
            \\
            -14.0  0.0003290366319177885  \\
            -14.0  0.013779027621483645  \\
        }
        ;
    \addplot[color={rgb,1:red,0.6039;green,0.1412;blue,0.0824}, name path={9b664ad0-e3fd-4b70-ac25-73aff5962437}, draw opacity={1.0}, line width={1}, solid, mark={-}, mark size={1.5 pt}, mark repeat={1}, mark options={color={rgb,1:red,0.6039;green,0.1412;blue,0.0824}, draw opacity={1.0}, fill={rgb,1:red,0.6039;green,0.1412;blue,0.0824}, fill opacity={1.0}, line width={0.75}, rotate={0}, solid}]
        table[row sep={\\}]
        {
            \\
            -13.0  -0.009773868555526633  \\
            -13.0  0.0032413450489575117  \\
        }
        ;
    \addplot[color={rgb,1:red,0.6039;green,0.1412;blue,0.0824}, name path={9b664ad0-e3fd-4b70-ac25-73aff5962437}, draw opacity={1.0}, line width={1}, solid, mark={-}, mark size={1.5 pt}, mark repeat={1}, mark options={color={rgb,1:red,0.6039;green,0.1412;blue,0.0824}, draw opacity={1.0}, fill={rgb,1:red,0.6039;green,0.1412;blue,0.0824}, fill opacity={1.0}, line width={0.75}, rotate={0}, solid}]
        table[row sep={\\}]
        {
            \\
            -12.0  -0.011048855776364178  \\
            -12.0  0.00019953001223847223  \\
        }
        ;
    \addplot[color={rgb,1:red,0.6039;green,0.1412;blue,0.0824}, name path={9b664ad0-e3fd-4b70-ac25-73aff5962437}, draw opacity={1.0}, line width={1}, solid, mark={-}, mark size={1.5 pt}, mark repeat={1}, mark options={color={rgb,1:red,0.6039;green,0.1412;blue,0.0824}, draw opacity={1.0}, fill={rgb,1:red,0.6039;green,0.1412;blue,0.0824}, fill opacity={1.0}, line width={0.75}, rotate={0}, solid}]
        table[row sep={\\}]
        {
            \\
            -11.0  -0.007078077338773719  \\
            -11.0  0.003615969030912147  \\
        }
        ;
    \addplot[color={rgb,1:red,0.6039;green,0.1412;blue,0.0824}, name path={9b664ad0-e3fd-4b70-ac25-73aff5962437}, draw opacity={1.0}, line width={1}, solid, mark={-}, mark size={1.5 pt}, mark repeat={1}, mark options={color={rgb,1:red,0.6039;green,0.1412;blue,0.0824}, draw opacity={1.0}, fill={rgb,1:red,0.6039;green,0.1412;blue,0.0824}, fill opacity={1.0}, line width={0.75}, rotate={0}, solid}]
        table[row sep={\\}]
        {
            \\
            -10.0  -0.006626688736279389  \\
            -10.0  0.004000094855340491  \\
        }
        ;
    \addplot[color={rgb,1:red,0.6039;green,0.1412;blue,0.0824}, name path={9b664ad0-e3fd-4b70-ac25-73aff5962437}, draw opacity={1.0}, line width={1}, solid, mark={-}, mark size={1.5 pt}, mark repeat={1}, mark options={color={rgb,1:red,0.6039;green,0.1412;blue,0.0824}, draw opacity={1.0}, fill={rgb,1:red,0.6039;green,0.1412;blue,0.0824}, fill opacity={1.0}, line width={0.75}, rotate={0}, solid}]
        table[row sep={\\}]
        {
            \\
            -9.0  -0.0076595948408875545  \\
            -9.0  0.0032631849331220906  \\
        }
        ;
    \addplot[color={rgb,1:red,0.6039;green,0.1412;blue,0.0824}, name path={9b664ad0-e3fd-4b70-ac25-73aff5962437}, draw opacity={1.0}, line width={1}, solid, mark={-}, mark size={1.5 pt}, mark repeat={1}, mark options={color={rgb,1:red,0.6039;green,0.1412;blue,0.0824}, draw opacity={1.0}, fill={rgb,1:red,0.6039;green,0.1412;blue,0.0824}, fill opacity={1.0}, line width={0.75}, rotate={0}, solid}]
        table[row sep={\\}]
        {
            \\
            -8.0  -0.006437986432305206  \\
            -8.0  0.004117933724438953  \\
        }
        ;
    \addplot[color={rgb,1:red,0.6039;green,0.1412;blue,0.0824}, name path={9b664ad0-e3fd-4b70-ac25-73aff5962437}, draw opacity={1.0}, line width={1}, solid, mark={-}, mark size={1.5 pt}, mark repeat={1}, mark options={color={rgb,1:red,0.6039;green,0.1412;blue,0.0824}, draw opacity={1.0}, fill={rgb,1:red,0.6039;green,0.1412;blue,0.0824}, fill opacity={1.0}, line width={0.75}, rotate={0}, solid}]
        table[row sep={\\}]
        {
            \\
            -7.0  -0.0015273015097126775  \\
            -7.0  0.009632165028455084  \\
        }
        ;
    \addplot[color={rgb,1:red,0.6039;green,0.1412;blue,0.0824}, name path={9b664ad0-e3fd-4b70-ac25-73aff5962437}, draw opacity={1.0}, line width={1}, solid, mark={-}, mark size={1.5 pt}, mark repeat={1}, mark options={color={rgb,1:red,0.6039;green,0.1412;blue,0.0824}, draw opacity={1.0}, fill={rgb,1:red,0.6039;green,0.1412;blue,0.0824}, fill opacity={1.0}, line width={0.75}, rotate={0}, solid}]
        table[row sep={\\}]
        {
            \\
            -6.0  -0.004689660785538569  \\
            -6.0  0.006038183024068414  \\
        }
        ;
    \addplot[color={rgb,1:red,0.6039;green,0.1412;blue,0.0824}, name path={9b664ad0-e3fd-4b70-ac25-73aff5962437}, draw opacity={1.0}, line width={1}, solid, mark={-}, mark size={1.5 pt}, mark repeat={1}, mark options={color={rgb,1:red,0.6039;green,0.1412;blue,0.0824}, draw opacity={1.0}, fill={rgb,1:red,0.6039;green,0.1412;blue,0.0824}, fill opacity={1.0}, line width={0.75}, rotate={0}, solid}]
        table[row sep={\\}]
        {
            \\
            -5.0  -0.0026091173425210404  \\
            -5.0  0.008521276545220837  \\
        }
        ;
    \addplot[color={rgb,1:red,0.6039;green,0.1412;blue,0.0824}, name path={9b664ad0-e3fd-4b70-ac25-73aff5962437}, draw opacity={1.0}, line width={1}, solid, mark={-}, mark size={1.5 pt}, mark repeat={1}, mark options={color={rgb,1:red,0.6039;green,0.1412;blue,0.0824}, draw opacity={1.0}, fill={rgb,1:red,0.6039;green,0.1412;blue,0.0824}, fill opacity={1.0}, line width={0.75}, rotate={0}, solid}]
        table[row sep={\\}]
        {
            \\
            -4.0  -6.248668196676133e-5  \\
            -4.0  0.009782537817281313  \\
        }
        ;
    \addplot[color={rgb,1:red,0.6039;green,0.1412;blue,0.0824}, name path={9b664ad0-e3fd-4b70-ac25-73aff5962437}, draw opacity={1.0}, line width={1}, solid, mark={-}, mark size={1.5 pt}, mark repeat={1}, mark options={color={rgb,1:red,0.6039;green,0.1412;blue,0.0824}, draw opacity={1.0}, fill={rgb,1:red,0.6039;green,0.1412;blue,0.0824}, fill opacity={1.0}, line width={0.75}, rotate={0}, solid}]
        table[row sep={\\}]
        {
            \\
            -3.0  -0.002895154253979358  \\
            -3.0  0.006201609598981246  \\
        }
        ;
    \addplot[color={rgb,1:red,0.6039;green,0.1412;blue,0.0824}, name path={9b664ad0-e3fd-4b70-ac25-73aff5962437}, draw opacity={1.0}, line width={1}, solid, mark={-}, mark size={1.5 pt}, mark repeat={1}, mark options={color={rgb,1:red,0.6039;green,0.1412;blue,0.0824}, draw opacity={1.0}, fill={rgb,1:red,0.6039;green,0.1412;blue,0.0824}, fill opacity={1.0}, line width={0.75}, rotate={0}, solid}]
        table[row sep={\\}]
        {
            \\
            -2.0  -0.007100303008334544  \\
            -2.0  0.005527073428848861  \\
        }
        ;
    \addplot[color={rgb,1:red,0.6039;green,0.1412;blue,0.0824}, name path={9b664ad0-e3fd-4b70-ac25-73aff5962437}, draw opacity={1.0}, line width={1}, solid, mark={-}, mark size={1.5 pt}, mark repeat={1}, mark options={color={rgb,1:red,0.6039;green,0.1412;blue,0.0824}, draw opacity={1.0}, fill={rgb,1:red,0.6039;green,0.1412;blue,0.0824}, fill opacity={1.0}, line width={0.75}, rotate={0}, solid}]
        table[row sep={\\}]
        {
            \\
            -1.0  -0.012869185577831736  \\
            -1.0  0.005074178718454277  \\
        }
        ;
    \addplot[color={rgb,1:red,0.2422;green,0.6433;blue,0.3044}, name path={5b58b6af-43c7-4dea-9140-250bc88f1c3b}, only marks, draw opacity={1.0}, line width={0}, solid, mark={*}, mark size={1.5 pt}, mark repeat={1}, mark options={color={rgb,1:red,0.6039;green,0.1412;blue,0.0824}, draw opacity={1.0}, fill={rgb,1:red,0.6039;green,0.1412;blue,0.0824}, fill opacity={1.0}, line width={0.75}, rotate={0}, solid}]
        table[row sep={\\}]
        {
            \\
            -22.0  -0.028524588525149754  \\
            -21.0  -0.026828318388205544  \\
            -20.0  -0.007927233215103696  \\
            -19.0  -0.004962127616382612  \\
            -18.0  -0.00036601236735008035  \\
            -17.0  -0.00312683705087688  \\
            -16.0  0.007689304582426019  \\
            -15.0  0.0051223737658222905  \\
            -14.0  0.007054032126700717  \\
            -13.0  -0.0032662617532845605  \\
            -12.0  -0.005424662882062853  \\
            -11.0  -0.001731054153930786  \\
            -10.0  -0.0013132969404694489  \\
            -9.0  -0.002198204953882732  \\
            -8.0  -0.0011600263539331267  \\
            -7.0  0.0040524317593712035  \\
            -6.0  0.0006742611192649226  \\
            -5.0  0.002956079601349898  \\
            -4.0  0.004860025567657276  \\
            -3.0  0.0016532276725009442  \\
            -2.0  -0.0007866147897428416  \\
            -1.0  -0.003897503429688729  \\
        }
        ;
    \addplot[color={rgb,1:red,0.0627;green,0.4706;blue,0.5843}, name path={9a97c198-846a-4101-aa25-158ab9a15d8e}, draw opacity={1.0}, line width={1}, solid, mark={-}, mark size={1.5 pt}, mark repeat={1}, mark options={color={rgb,1:red,0.0627;green,0.4706;blue,0.5843}, draw opacity={1.0}, fill={rgb,1:red,0.0627;green,0.4706;blue,0.5843}, fill opacity={1.0}, line width={0.75}, rotate={0}, solid}]
        table[row sep={\\}]
        {
            \\
            0.0  -0.0018692561266267227  \\
            0.0  0.028239619741945953  \\
        }
        ;
    \addplot[color={rgb,1:red,0.0627;green,0.4706;blue,0.5843}, name path={9a97c198-846a-4101-aa25-158ab9a15d8e}, draw opacity={1.0}, line width={1}, solid, mark={-}, mark size={1.5 pt}, mark repeat={1}, mark options={color={rgb,1:red,0.0627;green,0.4706;blue,0.5843}, draw opacity={1.0}, fill={rgb,1:red,0.0627;green,0.4706;blue,0.5843}, fill opacity={1.0}, line width={0.75}, rotate={0}, solid}]
        table[row sep={\\}]
        {
            \\
            1.0  0.0205668693114555  \\
            1.0  0.06272015027559127  \\
        }
        ;
    \addplot[color={rgb,1:red,0.0627;green,0.4706;blue,0.5843}, name path={9a97c198-846a-4101-aa25-158ab9a15d8e}, draw opacity={1.0}, line width={1}, solid, mark={-}, mark size={1.5 pt}, mark repeat={1}, mark options={color={rgb,1:red,0.0627;green,0.4706;blue,0.5843}, draw opacity={1.0}, fill={rgb,1:red,0.0627;green,0.4706;blue,0.5843}, fill opacity={1.0}, line width={0.75}, rotate={0}, solid}]
        table[row sep={\\}]
        {
            \\
            2.0  0.014363381764220515  \\
            2.0  0.06711811575546653  \\
        }
        ;
    \addplot[color={rgb,1:red,0.0627;green,0.4706;blue,0.5843}, name path={9a97c198-846a-4101-aa25-158ab9a15d8e}, draw opacity={1.0}, line width={1}, solid, mark={-}, mark size={1.5 pt}, mark repeat={1}, mark options={color={rgb,1:red,0.0627;green,0.4706;blue,0.5843}, draw opacity={1.0}, fill={rgb,1:red,0.0627;green,0.4706;blue,0.5843}, fill opacity={1.0}, line width={0.75}, rotate={0}, solid}]
        table[row sep={\\}]
        {
            \\
            3.0  0.005803734363782384  \\
            3.0  0.06893359272131452  \\
        }
        ;
    \addplot[color={rgb,1:red,0.0627;green,0.4706;blue,0.5843}, name path={9a97c198-846a-4101-aa25-158ab9a15d8e}, draw opacity={1.0}, line width={1}, solid, mark={-}, mark size={1.5 pt}, mark repeat={1}, mark options={color={rgb,1:red,0.0627;green,0.4706;blue,0.5843}, draw opacity={1.0}, fill={rgb,1:red,0.0627;green,0.4706;blue,0.5843}, fill opacity={1.0}, line width={0.75}, rotate={0}, solid}]
        table[row sep={\\}]
        {
            \\
            4.0  0.005091965369339244  \\
            4.0  0.07747074596796331  \\
        }
        ;
    \addplot[color={rgb,1:red,0.0627;green,0.4706;blue,0.5843}, name path={9a97c198-846a-4101-aa25-158ab9a15d8e}, draw opacity={1.0}, line width={1}, solid, mark={-}, mark size={1.5 pt}, mark repeat={1}, mark options={color={rgb,1:red,0.0627;green,0.4706;blue,0.5843}, draw opacity={1.0}, fill={rgb,1:red,0.0627;green,0.4706;blue,0.5843}, fill opacity={1.0}, line width={0.75}, rotate={0}, solid}]
        table[row sep={\\}]
        {
            \\
            5.0  0.006720029522983395  \\
            5.0  0.09068001066325225  \\
        }
        ;
    \addplot[color={rgb,1:red,0.0627;green,0.4706;blue,0.5843}, name path={9a97c198-846a-4101-aa25-158ab9a15d8e}, draw opacity={1.0}, line width={1}, solid, mark={-}, mark size={1.5 pt}, mark repeat={1}, mark options={color={rgb,1:red,0.0627;green,0.4706;blue,0.5843}, draw opacity={1.0}, fill={rgb,1:red,0.0627;green,0.4706;blue,0.5843}, fill opacity={1.0}, line width={0.75}, rotate={0}, solid}]
        table[row sep={\\}]
        {
            \\
            6.0  0.015114690248520843  \\
            6.0  0.1042829798836477  \\
        }
        ;
    \addplot[color={rgb,1:red,0.0627;green,0.4706;blue,0.5843}, name path={9a97c198-846a-4101-aa25-158ab9a15d8e}, draw opacity={1.0}, line width={1}, solid, mark={-}, mark size={1.5 pt}, mark repeat={1}, mark options={color={rgb,1:red,0.0627;green,0.4706;blue,0.5843}, draw opacity={1.0}, fill={rgb,1:red,0.0627;green,0.4706;blue,0.5843}, fill opacity={1.0}, line width={0.75}, rotate={0}, solid}]
        table[row sep={\\}]
        {
            \\
            7.0  0.023581403932003676  \\
            7.0  0.12379326628055878  \\
        }
        ;
    \addplot[color={rgb,1:red,0.0627;green,0.4706;blue,0.5843}, name path={9a97c198-846a-4101-aa25-158ab9a15d8e}, draw opacity={1.0}, line width={1}, solid, mark={-}, mark size={1.5 pt}, mark repeat={1}, mark options={color={rgb,1:red,0.0627;green,0.4706;blue,0.5843}, draw opacity={1.0}, fill={rgb,1:red,0.0627;green,0.4706;blue,0.5843}, fill opacity={1.0}, line width={0.75}, rotate={0}, solid}]
        table[row sep={\\}]
        {
            \\
            8.0  0.02571392221646169  \\
            8.0  0.14134669441514772  \\
        }
        ;
    \addplot[color={rgb,1:red,0.0627;green,0.4706;blue,0.5843}, name path={9a97c198-846a-4101-aa25-158ab9a15d8e}, draw opacity={1.0}, line width={1}, solid, mark={-}, mark size={1.5 pt}, mark repeat={1}, mark options={color={rgb,1:red,0.0627;green,0.4706;blue,0.5843}, draw opacity={1.0}, fill={rgb,1:red,0.0627;green,0.4706;blue,0.5843}, fill opacity={1.0}, line width={0.75}, rotate={0}, solid}]
        table[row sep={\\}]
        {
            \\
            9.0  0.01017622359514496  \\
            9.0  0.15951214710477746  \\
        }
        ;
    \addplot[color={rgb,1:red,0.0627;green,0.4706;blue,0.5843}, name path={9a97c198-846a-4101-aa25-158ab9a15d8e}, draw opacity={1.0}, line width={1}, solid, mark={-}, mark size={1.5 pt}, mark repeat={1}, mark options={color={rgb,1:red,0.0627;green,0.4706;blue,0.5843}, draw opacity={1.0}, fill={rgb,1:red,0.0627;green,0.4706;blue,0.5843}, fill opacity={1.0}, line width={0.75}, rotate={0}, solid}]
        table[row sep={\\}]
        {
            \\
            10.0  -0.01936122210637914  \\
            10.0  0.16476796839155314  \\
        }
        ;
    \addplot[color={rgb,1:red,0.0627;green,0.4706;blue,0.5843}, name path={9a97c198-846a-4101-aa25-158ab9a15d8e}, draw opacity={1.0}, line width={1}, solid, mark={-}, mark size={1.5 pt}, mark repeat={1}, mark options={color={rgb,1:red,0.0627;green,0.4706;blue,0.5843}, draw opacity={1.0}, fill={rgb,1:red,0.0627;green,0.4706;blue,0.5843}, fill opacity={1.0}, line width={0.75}, rotate={0}, solid}]
        table[row sep={\\}]
        {
            \\
            11.0  -0.058890297357561606  \\
            11.0  0.1782343924296466  \\
        }
        ;
    \addplot[color={rgb,1:red,0.0627;green,0.4706;blue,0.5843}, name path={9a97c198-846a-4101-aa25-158ab9a15d8e}, draw opacity={1.0}, line width={1}, solid, mark={-}, mark size={1.5 pt}, mark repeat={1}, mark options={color={rgb,1:red,0.0627;green,0.4706;blue,0.5843}, draw opacity={1.0}, fill={rgb,1:red,0.0627;green,0.4706;blue,0.5843}, fill opacity={1.0}, line width={0.75}, rotate={0}, solid}]
        table[row sep={\\}]
        {
            \\
            12.0  -0.08469628150504467  \\
            12.0  0.19956902824779937  \\
        }
        ;
    \addplot[color={rgb,1:red,0.0627;green,0.4706;blue,0.5843}, name path={9a97c198-846a-4101-aa25-158ab9a15d8e}, draw opacity={1.0}, line width={1}, solid, mark={-}, mark size={1.5 pt}, mark repeat={1}, mark options={color={rgb,1:red,0.0627;green,0.4706;blue,0.5843}, draw opacity={1.0}, fill={rgb,1:red,0.0627;green,0.4706;blue,0.5843}, fill opacity={1.0}, line width={0.75}, rotate={0}, solid}]
        table[row sep={\\}]
        {
            \\
            13.0  -0.05006488160287552  \\
            13.0  0.25038141195328645  \\
        }
        ;
    \addplot[color={rgb,1:red,0.7644;green,0.4441;blue,0.8243}, name path={ba960072-e362-44a1-b12f-1cff3e3ccfc3}, only marks, draw opacity={1.0}, line width={0}, solid, mark={*}, mark size={1.5 pt}, mark repeat={1}, mark options={color={rgb,1:red,0.0627;green,0.4706;blue,0.5843}, draw opacity={1.0}, fill={rgb,1:red,0.0627;green,0.4706;blue,0.5843}, fill opacity={1.0}, line width={0.75}, rotate={0}, solid}]
        table[row sep={\\}]
        {
            \\
            0.0  0.013185181807659615  \\
            1.0  0.04164350979352339  \\
            2.0  0.04074074875984352  \\
            3.0  0.037368663542548454  \\
            4.0  0.04128135566865128  \\
            5.0  0.04870002009311782  \\
            6.0  0.05969883506608427  \\
            7.0  0.07368733510628123  \\
            8.0  0.0835303083158047  \\
            9.0  0.08484418534996122  \\
            10.0  0.072703373142587  \\
            11.0  0.05967204753604249  \\
            12.0  0.057436373371377344  \\
            13.0  0.10015826517520547  \\
        }
        ;
\end{axis}
\end{tikzpicture}

    \end{adjustbox}
\end{subfigure}
\end{figure}
\begin{itemize}
    \item Average coefficient estimate of $6\%$ in post-treatment periods. 
    \begin{itemize}
        \item Consistent with Basker (2005) and Stapp (2014). 
    \end{itemize}
\end{itemize}
\end{frame}

%%%%%%%%%%%%%%%%%%%%%%%%%%%%%%%%%%%%%%%%%%%%%%%%%%%%%%%%%%%%%%%%%%%%%%%

\begin{frame}{Results}
    \begin{figure}
\caption*{Effect of Walmart on County $\log$ Wholesale Retail Employment}
\label{fig:walmart_wholesale}

\begin{subfigure}[b]{0.45\textwidth}
    \caption{TWFE Imputation Estimator}
    \begin{adjustbox}{width=\textwidth, center}
        % Recommended preamble:
% \usetikzlibrary{arrows.meta}
% \usetikzlibrary{backgrounds}
% \usepgfplotslibrary{patchplots}
% \usepgfplotslibrary{fillbetween}
% \pgfplotsset{%
%     layers/standard/.define layer set={%
%         background,axis background,axis grid,axis ticks,axis lines,axis tick labels,pre main,main,axis descriptions,axis foreground%
%     }{
%         grid style={/pgfplots/on layer=axis grid},%
%         tick style={/pgfplots/on layer=axis ticks},%
%         axis line style={/pgfplots/on layer=axis lines},%
%         label style={/pgfplots/on layer=axis descriptions},%
%         legend style={/pgfplots/on layer=axis descriptions},%
%         title style={/pgfplots/on layer=axis descriptions},%
%         colorbar style={/pgfplots/on layer=axis descriptions},%
%         ticklabel style={/pgfplots/on layer=axis tick labels},%
%         axis background@ style={/pgfplots/on layer=axis background},%
%         3d box foreground style={/pgfplots/on layer=axis foreground},%
%     },
% }

\begin{tikzpicture}[/tikz/background rectangle/.style={fill={rgb,1:red,1.0;green,1.0;blue,1.0}, draw opacity={1.0}}, show background rectangle]
\begin{axis}[point meta max={nan}, point meta min={nan}, legend cell align={left}, legend columns={1}, title={}, title style={at={{(0.5,1)}}, anchor={south}, font={{\fontsize{14 pt}{18.2 pt}\selectfont}}, color={rgb,1:red,0.0;green,0.0;blue,0.0}, draw opacity={1.0}, rotate={0.0}, align={center}}, legend style={color={rgb,1:red,0.0;green,0.0;blue,0.0}, draw opacity={1.0}, line width={1}, solid, fill={rgb,1:red,1.0;green,1.0;blue,1.0}, fill opacity={1.0}, text opacity={1.0}, font={{\fontsize{8 pt}{10.4 pt}\selectfont}}, text={rgb,1:red,0.0;green,0.0;blue,0.0}, cells={anchor={center}}, at={(1.02, 1)}, anchor={north west}}, axis background/.style={fill={rgb,1:red,1.0;green,1.0;blue,1.0}, opacity={1.0}}, anchor={north west}, xshift={1.0mm}, yshift={-1.0mm}, width={97.06mm}, height={64.04mm}, scaled x ticks={false}, xlabel={Event Time}, x tick style={color={rgb,1:red,0.0;green,0.0;blue,0.0}, opacity={1.0}}, x tick label style={color={rgb,1:red,0.0;green,0.0;blue,0.0}, opacity={1.0}, rotate={0}}, xlabel style={at={(ticklabel cs:0.5)}, anchor=near ticklabel, at={{(ticklabel cs:0.5)}}, anchor={near ticklabel}, font={{\fontsize{11 pt}{14.3 pt}\selectfont}}, color={rgb,1:red,0.0;green,0.0;blue,0.0}, draw opacity={1.0}, rotate={0.0}}, xmajorgrids={true}, xmin={-23.05}, xmax={14.05}, xticklabels={{$-20$,$-10$,$0$,$10$}}, xtick={{-20.0,-10.0,0.0,10.0}}, xtick align={inside}, xticklabel style={font={{\fontsize{8 pt}{10.4 pt}\selectfont}}, color={rgb,1:red,0.0;green,0.0;blue,0.0}, draw opacity={1.0}, rotate={0.0}}, x grid style={color={rgb,1:red,0.0;green,0.0;blue,0.0}, draw opacity={0.1}, line width={0.5}, solid}, axis x line*={left}, x axis line style={color={rgb,1:red,0.0;green,0.0;blue,0.0}, draw opacity={1.0}, line width={1}, solid}, scaled y ticks={false}, ylabel={Coefficient}, y tick style={color={rgb,1:red,0.0;green,0.0;blue,0.0}, opacity={1.0}}, y tick label style={color={rgb,1:red,0.0;green,0.0;blue,0.0}, opacity={1.0}, rotate={0}}, ylabel style={at={(ticklabel cs:0.5)}, anchor=near ticklabel, at={{(ticklabel cs:0.5)}}, anchor={near ticklabel}, font={{\fontsize{11 pt}{14.3 pt}\selectfont}}, color={rgb,1:red,0.0;green,0.0;blue,0.0}, draw opacity={1.0}, rotate={0.0}}, ymajorgrids={true}, ymin={-0.4}, ymax={0.2}, yticklabels={{$-0.4$,$-0.3$,$-0.2$,$-0.1$,$0.0$,$0.1$,$0.2$}}, ytick={{-0.4,-0.30000000000000004,-0.2,-0.1,0.0,0.1,0.2}}, ytick align={inside}, yticklabel style={font={{\fontsize{8 pt}{10.4 pt}\selectfont}}, color={rgb,1:red,0.0;green,0.0;blue,0.0}, draw opacity={1.0}, rotate={0.0}}, y grid style={color={rgb,1:red,0.0;green,0.0;blue,0.0}, draw opacity={0.1}, line width={0.5}, solid}, axis y line*={left}, y axis line style={color={rgb,1:red,0.0;green,0.0;blue,0.0}, draw opacity={1.0}, line width={1}, solid}, colorbar={false}]
    \addplot[color={rgb,1:red,0.0;green,0.0;blue,0.0}, name path={e4b3c3a7-bd9c-43dd-8459-834ec3268279}, draw opacity={1.0}, line width={1}, dashed]
        table[row sep={\\}]
        {
            \\
            -60.150000000000006  0.0  \\
            51.150000000000006  0.0  \\
        }
        ;
    \addplot[color={rgb,1:red,0.6039;green,0.1412;blue,0.0824}, name path={9864498f-37ef-47ba-bd3a-875bb8102c9a}, draw opacity={1.0}, line width={1}, solid]
        table[row sep={\\}]
        {
            \\
            -60.150000000000006  -0.2704746278144035  \\
            51.150000000000006  0.305802234514702  \\
        }
        ;
    \addplot[color={rgb,1:red,0.6039;green,0.1412;blue,0.0824}, name path={7b09ae57-16ee-4e77-8338-1cf18ace8480}, draw opacity={1.0}, line width={1}, solid, mark={-}, mark size={1.5 pt}, mark repeat={1}, mark options={color={rgb,1:red,0.6039;green,0.1412;blue,0.0824}, draw opacity={1.0}, fill={rgb,1:red,0.6039;green,0.1412;blue,0.0824}, fill opacity={1.0}, line width={0.75}, rotate={0}, solid}]
        table[row sep={\\}]
        {
            \\
            -22.0  -0.212037738463464  \\
            -22.0  0.0275181412346501  \\
        }
        ;
    \addplot[color={rgb,1:red,0.6039;green,0.1412;blue,0.0824}, name path={7b09ae57-16ee-4e77-8338-1cf18ace8480}, draw opacity={1.0}, line width={1}, solid, mark={-}, mark size={1.5 pt}, mark repeat={1}, mark options={color={rgb,1:red,0.6039;green,0.1412;blue,0.0824}, draw opacity={1.0}, fill={rgb,1:red,0.6039;green,0.1412;blue,0.0824}, fill opacity={1.0}, line width={0.75}, rotate={0}, solid}]
        table[row sep={\\}]
        {
            \\
            -21.0  -0.153570543005722  \\
            -21.0  0.017295846321045805  \\
        }
        ;
    \addplot[color={rgb,1:red,0.6039;green,0.1412;blue,0.0824}, name path={7b09ae57-16ee-4e77-8338-1cf18ace8480}, draw opacity={1.0}, line width={1}, solid, mark={-}, mark size={1.5 pt}, mark repeat={1}, mark options={color={rgb,1:red,0.6039;green,0.1412;blue,0.0824}, draw opacity={1.0}, fill={rgb,1:red,0.6039;green,0.1412;blue,0.0824}, fill opacity={1.0}, line width={0.75}, rotate={0}, solid}]
        table[row sep={\\}]
        {
            \\
            -20.0  -0.105414655889321  \\
            -20.0  0.021629824097528796  \\
        }
        ;
    \addplot[color={rgb,1:red,0.6039;green,0.1412;blue,0.0824}, name path={7b09ae57-16ee-4e77-8338-1cf18ace8480}, draw opacity={1.0}, line width={1}, solid, mark={-}, mark size={1.5 pt}, mark repeat={1}, mark options={color={rgb,1:red,0.6039;green,0.1412;blue,0.0824}, draw opacity={1.0}, fill={rgb,1:red,0.6039;green,0.1412;blue,0.0824}, fill opacity={1.0}, line width={0.75}, rotate={0}, solid}]
        table[row sep={\\}]
        {
            \\
            -19.0  -0.104305739549016  \\
            -19.0  -0.004974830882164827  \\
        }
        ;
    \addplot[color={rgb,1:red,0.6039;green,0.1412;blue,0.0824}, name path={7b09ae57-16ee-4e77-8338-1cf18ace8480}, draw opacity={1.0}, line width={1}, solid, mark={-}, mark size={1.5 pt}, mark repeat={1}, mark options={color={rgb,1:red,0.6039;green,0.1412;blue,0.0824}, draw opacity={1.0}, fill={rgb,1:red,0.6039;green,0.1412;blue,0.0824}, fill opacity={1.0}, line width={0.75}, rotate={0}, solid}]
        table[row sep={\\}]
        {
            \\
            -18.0  -0.0782595135504461  \\
            -18.0  -0.0074891678097325914  \\
        }
        ;
    \addplot[color={rgb,1:red,0.6039;green,0.1412;blue,0.0824}, name path={7b09ae57-16ee-4e77-8338-1cf18ace8480}, draw opacity={1.0}, line width={1}, solid, mark={-}, mark size={1.5 pt}, mark repeat={1}, mark options={color={rgb,1:red,0.6039;green,0.1412;blue,0.0824}, draw opacity={1.0}, fill={rgb,1:red,0.6039;green,0.1412;blue,0.0824}, fill opacity={1.0}, line width={0.75}, rotate={0}, solid}]
        table[row sep={\\}]
        {
            \\
            -17.0  -0.0707552409253196  \\
            -17.0  -0.010402871133045603  \\
        }
        ;
    \addplot[color={rgb,1:red,0.6039;green,0.1412;blue,0.0824}, name path={7b09ae57-16ee-4e77-8338-1cf18ace8480}, draw opacity={1.0}, line width={1}, solid, mark={-}, mark size={1.5 pt}, mark repeat={1}, mark options={color={rgb,1:red,0.6039;green,0.1412;blue,0.0824}, draw opacity={1.0}, fill={rgb,1:red,0.6039;green,0.1412;blue,0.0824}, fill opacity={1.0}, line width={0.75}, rotate={0}, solid}]
        table[row sep={\\}]
        {
            \\
            -16.0  -0.0664482211391317  \\
            -16.0  -0.024482323998229  \\
        }
        ;
    \addplot[color={rgb,1:red,0.6039;green,0.1412;blue,0.0824}, name path={7b09ae57-16ee-4e77-8338-1cf18ace8480}, draw opacity={1.0}, line width={1}, solid, mark={-}, mark size={1.5 pt}, mark repeat={1}, mark options={color={rgb,1:red,0.6039;green,0.1412;blue,0.0824}, draw opacity={1.0}, fill={rgb,1:red,0.6039;green,0.1412;blue,0.0824}, fill opacity={1.0}, line width={0.75}, rotate={0}, solid}]
        table[row sep={\\}]
        {
            \\
            -15.0  -0.0617102486481638  \\
            -15.0  -0.0260392522909419  \\
        }
        ;
    \addplot[color={rgb,1:red,0.6039;green,0.1412;blue,0.0824}, name path={7b09ae57-16ee-4e77-8338-1cf18ace8480}, draw opacity={1.0}, line width={1}, solid, mark={-}, mark size={1.5 pt}, mark repeat={1}, mark options={color={rgb,1:red,0.6039;green,0.1412;blue,0.0824}, draw opacity={1.0}, fill={rgb,1:red,0.6039;green,0.1412;blue,0.0824}, fill opacity={1.0}, line width={0.75}, rotate={0}, solid}]
        table[row sep={\\}]
        {
            \\
            -14.0  -0.0518356183299787  \\
            -14.0  -0.0228880750275636  \\
        }
        ;
    \addplot[color={rgb,1:red,0.6039;green,0.1412;blue,0.0824}, name path={7b09ae57-16ee-4e77-8338-1cf18ace8480}, draw opacity={1.0}, line width={1}, solid, mark={-}, mark size={1.5 pt}, mark repeat={1}, mark options={color={rgb,1:red,0.6039;green,0.1412;blue,0.0824}, draw opacity={1.0}, fill={rgb,1:red,0.6039;green,0.1412;blue,0.0824}, fill opacity={1.0}, line width={0.75}, rotate={0}, solid}]
        table[row sep={\\}]
        {
            \\
            -13.0  -0.038737595741401  \\
            -13.0  -0.01325227436325  \\
        }
        ;
    \addplot[color={rgb,1:red,0.6039;green,0.1412;blue,0.0824}, name path={7b09ae57-16ee-4e77-8338-1cf18ace8480}, draw opacity={1.0}, line width={1}, solid, mark={-}, mark size={1.5 pt}, mark repeat={1}, mark options={color={rgb,1:red,0.6039;green,0.1412;blue,0.0824}, draw opacity={1.0}, fill={rgb,1:red,0.6039;green,0.1412;blue,0.0824}, fill opacity={1.0}, line width={0.75}, rotate={0}, solid}]
        table[row sep={\\}]
        {
            \\
            -12.0  -0.0231734107842414  \\
            -12.0  0.0014055188664592894  \\
        }
        ;
    \addplot[color={rgb,1:red,0.6039;green,0.1412;blue,0.0824}, name path={7b09ae57-16ee-4e77-8338-1cf18ace8480}, draw opacity={1.0}, line width={1}, solid, mark={-}, mark size={1.5 pt}, mark repeat={1}, mark options={color={rgb,1:red,0.6039;green,0.1412;blue,0.0824}, draw opacity={1.0}, fill={rgb,1:red,0.6039;green,0.1412;blue,0.0824}, fill opacity={1.0}, line width={0.75}, rotate={0}, solid}]
        table[row sep={\\}]
        {
            \\
            -11.0  -0.0157341554717529  \\
            -11.0  0.0056399036193939  \\
        }
        ;
    \addplot[color={rgb,1:red,0.6039;green,0.1412;blue,0.0824}, name path={7b09ae57-16ee-4e77-8338-1cf18ace8480}, draw opacity={1.0}, line width={1}, solid, mark={-}, mark size={1.5 pt}, mark repeat={1}, mark options={color={rgb,1:red,0.6039;green,0.1412;blue,0.0824}, draw opacity={1.0}, fill={rgb,1:red,0.6039;green,0.1412;blue,0.0824}, fill opacity={1.0}, line width={0.75}, rotate={0}, solid}]
        table[row sep={\\}]
        {
            \\
            -10.0  -0.018009907346868  \\
            -10.0  0.003772215776803511  \\
        }
        ;
    \addplot[color={rgb,1:red,0.6039;green,0.1412;blue,0.0824}, name path={7b09ae57-16ee-4e77-8338-1cf18ace8480}, draw opacity={1.0}, line width={1}, solid, mark={-}, mark size={1.5 pt}, mark repeat={1}, mark options={color={rgb,1:red,0.6039;green,0.1412;blue,0.0824}, draw opacity={1.0}, fill={rgb,1:red,0.6039;green,0.1412;blue,0.0824}, fill opacity={1.0}, line width={0.75}, rotate={0}, solid}]
        table[row sep={\\}]
        {
            \\
            -9.0  -0.0180064957389233  \\
            -9.0  0.0016109621988452498  \\
        }
        ;
    \addplot[color={rgb,1:red,0.6039;green,0.1412;blue,0.0824}, name path={7b09ae57-16ee-4e77-8338-1cf18ace8480}, draw opacity={1.0}, line width={1}, solid, mark={-}, mark size={1.5 pt}, mark repeat={1}, mark options={color={rgb,1:red,0.6039;green,0.1412;blue,0.0824}, draw opacity={1.0}, fill={rgb,1:red,0.6039;green,0.1412;blue,0.0824}, fill opacity={1.0}, line width={0.75}, rotate={0}, solid}]
        table[row sep={\\}]
        {
            \\
            -8.0  -0.0160358300414195  \\
            -8.0  0.0018134075135123294  \\
        }
        ;
    \addplot[color={rgb,1:red,0.6039;green,0.1412;blue,0.0824}, name path={7b09ae57-16ee-4e77-8338-1cf18ace8480}, draw opacity={1.0}, line width={1}, solid, mark={-}, mark size={1.5 pt}, mark repeat={1}, mark options={color={rgb,1:red,0.6039;green,0.1412;blue,0.0824}, draw opacity={1.0}, fill={rgb,1:red,0.6039;green,0.1412;blue,0.0824}, fill opacity={1.0}, line width={0.75}, rotate={0}, solid}]
        table[row sep={\\}]
        {
            \\
            -7.0  -0.006358690999683429  \\
            -7.0  0.011875307241686398  \\
        }
        ;
    \addplot[color={rgb,1:red,0.6039;green,0.1412;blue,0.0824}, name path={7b09ae57-16ee-4e77-8338-1cf18ace8480}, draw opacity={1.0}, line width={1}, solid, mark={-}, mark size={1.5 pt}, mark repeat={1}, mark options={color={rgb,1:red,0.6039;green,0.1412;blue,0.0824}, draw opacity={1.0}, fill={rgb,1:red,0.6039;green,0.1412;blue,0.0824}, fill opacity={1.0}, line width={0.75}, rotate={0}, solid}]
        table[row sep={\\}]
        {
            \\
            -6.0  0.0017991062587725797  \\
            -6.0  0.0199753469691002  \\
        }
        ;
    \addplot[color={rgb,1:red,0.6039;green,0.1412;blue,0.0824}, name path={7b09ae57-16ee-4e77-8338-1cf18ace8480}, draw opacity={1.0}, line width={1}, solid, mark={-}, mark size={1.5 pt}, mark repeat={1}, mark options={color={rgb,1:red,0.6039;green,0.1412;blue,0.0824}, draw opacity={1.0}, fill={rgb,1:red,0.6039;green,0.1412;blue,0.0824}, fill opacity={1.0}, line width={0.75}, rotate={0}, solid}]
        table[row sep={\\}]
        {
            \\
            -5.0  0.0005646338203266973  \\
            -5.0  0.0185549519555681  \\
        }
        ;
    \addplot[color={rgb,1:red,0.6039;green,0.1412;blue,0.0824}, name path={7b09ae57-16ee-4e77-8338-1cf18ace8480}, draw opacity={1.0}, line width={1}, solid, mark={-}, mark size={1.5 pt}, mark repeat={1}, mark options={color={rgb,1:red,0.6039;green,0.1412;blue,0.0824}, draw opacity={1.0}, fill={rgb,1:red,0.6039;green,0.1412;blue,0.0824}, fill opacity={1.0}, line width={0.75}, rotate={0}, solid}]
        table[row sep={\\}]
        {
            \\
            -4.0  0.0100933294576374  \\
            -4.0  0.0294321085121031  \\
        }
        ;
    \addplot[color={rgb,1:red,0.6039;green,0.1412;blue,0.0824}, name path={7b09ae57-16ee-4e77-8338-1cf18ace8480}, draw opacity={1.0}, line width={1}, solid, mark={-}, mark size={1.5 pt}, mark repeat={1}, mark options={color={rgb,1:red,0.6039;green,0.1412;blue,0.0824}, draw opacity={1.0}, fill={rgb,1:red,0.6039;green,0.1412;blue,0.0824}, fill opacity={1.0}, line width={0.75}, rotate={0}, solid}]
        table[row sep={\\}]
        {
            \\
            -3.0  0.0136384441272192  \\
            -3.0  0.0369176634476061  \\
        }
        ;
    \addplot[color={rgb,1:red,0.6039;green,0.1412;blue,0.0824}, name path={7b09ae57-16ee-4e77-8338-1cf18ace8480}, draw opacity={1.0}, line width={1}, solid, mark={-}, mark size={1.5 pt}, mark repeat={1}, mark options={color={rgb,1:red,0.6039;green,0.1412;blue,0.0824}, draw opacity={1.0}, fill={rgb,1:red,0.6039;green,0.1412;blue,0.0824}, fill opacity={1.0}, line width={0.75}, rotate={0}, solid}]
        table[row sep={\\}]
        {
            \\
            -2.0  0.0165391836918878  \\
            -2.0  0.0426057565315229  \\
        }
        ;
    \addplot[color={rgb,1:red,0.6039;green,0.1412;blue,0.0824}, name path={7b09ae57-16ee-4e77-8338-1cf18ace8480}, draw opacity={1.0}, line width={1}, solid, mark={-}, mark size={1.5 pt}, mark repeat={1}, mark options={color={rgb,1:red,0.6039;green,0.1412;blue,0.0824}, draw opacity={1.0}, fill={rgb,1:red,0.6039;green,0.1412;blue,0.0824}, fill opacity={1.0}, line width={0.75}, rotate={0}, solid}]
        table[row sep={\\}]
        {
            \\
            -1.0  0.0225814676295955  \\
            -1.0  0.0558144954680383  \\
        }
        ;
    \addplot[color={rgb,1:red,0.2422;green,0.6433;blue,0.3044}, name path={1ae0835d-bd23-4577-b840-7eadd1a22bee}, only marks, draw opacity={1.0}, line width={0}, solid, mark={*}, mark size={1.5 pt}, mark repeat={1}, mark options={color={rgb,1:red,0.6039;green,0.1412;blue,0.0824}, draw opacity={1.0}, fill={rgb,1:red,0.6039;green,0.1412;blue,0.0824}, fill opacity={1.0}, line width={0.75}, rotate={0}, solid}]
        table[row sep={\\}]
        {
            \\
            -22.0  -0.0907650399814156  \\
            -21.0  -0.0692649108423698  \\
            -20.0  -0.04239594400627  \\
            -19.0  -0.0571339269418137  \\
            -18.0  -0.0430223359052379  \\
            -17.0  -0.0418247965054643  \\
            -16.0  -0.046410398032084  \\
            -15.0  -0.0429088822165718  \\
            -14.0  -0.0379470875238167  \\
            -13.0  -0.0263002508413485  \\
            -12.0  -0.0106380079508035  \\
            -11.0  -0.00476843512566271  \\
            -10.0  -0.00667134975084774  \\
            -9.0  -0.00797629494582106  \\
            -8.0  -0.00664883092247426  \\
            -7.0  0.00241981248964349  \\
            -6.0  0.0107015074051253  \\
            -5.0  0.00963071972176959  \\
            -4.0  0.0198581639925891  \\
            -3.0  0.0256188916857122  \\
            -2.0  0.029519973702447  \\
            -1.0  0.0392384118864939  \\
        }
        ;
    \addplot[color={rgb,1:red,0.0627;green,0.4706;blue,0.5843}, name path={60739e55-b72b-4c5c-9105-da6d2852b2a7}, draw opacity={1.0}, line width={1}, solid, mark={-}, mark size={1.5 pt}, mark repeat={1}, mark options={color={rgb,1:red,0.0627;green,0.4706;blue,0.5843}, draw opacity={1.0}, fill={rgb,1:red,0.0627;green,0.4706;blue,0.5843}, fill opacity={1.0}, line width={0.75}, rotate={0}, solid}]
        table[row sep={\\}]
        {
            \\
            0.0  0.0297833569174828  \\
            0.0  0.0704367143365895  \\
        }
        ;
    \addplot[color={rgb,1:red,0.0627;green,0.4706;blue,0.5843}, name path={60739e55-b72b-4c5c-9105-da6d2852b2a7}, draw opacity={1.0}, line width={1}, solid, mark={-}, mark size={1.5 pt}, mark repeat={1}, mark options={color={rgb,1:red,0.0627;green,0.4706;blue,0.5843}, draw opacity={1.0}, fill={rgb,1:red,0.0627;green,0.4706;blue,0.5843}, fill opacity={1.0}, line width={0.75}, rotate={0}, solid}]
        table[row sep={\\}]
        {
            \\
            1.0  0.0208475064556466  \\
            1.0  0.0645208314488823  \\
        }
        ;
    \addplot[color={rgb,1:red,0.0627;green,0.4706;blue,0.5843}, name path={60739e55-b72b-4c5c-9105-da6d2852b2a7}, draw opacity={1.0}, line width={1}, solid, mark={-}, mark size={1.5 pt}, mark repeat={1}, mark options={color={rgb,1:red,0.0627;green,0.4706;blue,0.5843}, draw opacity={1.0}, fill={rgb,1:red,0.0627;green,0.4706;blue,0.5843}, fill opacity={1.0}, line width={0.75}, rotate={0}, solid}]
        table[row sep={\\}]
        {
            \\
            2.0  0.0170907802580872  \\
            2.0  0.0672925302943784  \\
        }
        ;
    \addplot[color={rgb,1:red,0.0627;green,0.4706;blue,0.5843}, name path={60739e55-b72b-4c5c-9105-da6d2852b2a7}, draw opacity={1.0}, line width={1}, solid, mark={-}, mark size={1.5 pt}, mark repeat={1}, mark options={color={rgb,1:red,0.0627;green,0.4706;blue,0.5843}, draw opacity={1.0}, fill={rgb,1:red,0.0627;green,0.4706;blue,0.5843}, fill opacity={1.0}, line width={0.75}, rotate={0}, solid}]
        table[row sep={\\}]
        {
            \\
            3.0  0.012313775797661401  \\
            3.0  0.0708962257264235  \\
        }
        ;
    \addplot[color={rgb,1:red,0.0627;green,0.4706;blue,0.5843}, name path={60739e55-b72b-4c5c-9105-da6d2852b2a7}, draw opacity={1.0}, line width={1}, solid, mark={-}, mark size={1.5 pt}, mark repeat={1}, mark options={color={rgb,1:red,0.0627;green,0.4706;blue,0.5843}, draw opacity={1.0}, fill={rgb,1:red,0.0627;green,0.4706;blue,0.5843}, fill opacity={1.0}, line width={0.75}, rotate={0}, solid}]
        table[row sep={\\}]
        {
            \\
            4.0  0.015020032183400603  \\
            4.0  0.0782916792406457  \\
        }
        ;
    \addplot[color={rgb,1:red,0.0627;green,0.4706;blue,0.5843}, name path={60739e55-b72b-4c5c-9105-da6d2852b2a7}, draw opacity={1.0}, line width={1}, solid, mark={-}, mark size={1.5 pt}, mark repeat={1}, mark options={color={rgb,1:red,0.0627;green,0.4706;blue,0.5843}, draw opacity={1.0}, fill={rgb,1:red,0.0627;green,0.4706;blue,0.5843}, fill opacity={1.0}, line width={0.75}, rotate={0}, solid}]
        table[row sep={\\}]
        {
            \\
            5.0  0.0203057832313217  \\
            5.0  0.0923338424890995  \\
        }
        ;
    \addplot[color={rgb,1:red,0.0627;green,0.4706;blue,0.5843}, name path={60739e55-b72b-4c5c-9105-da6d2852b2a7}, draw opacity={1.0}, line width={1}, solid, mark={-}, mark size={1.5 pt}, mark repeat={1}, mark options={color={rgb,1:red,0.0627;green,0.4706;blue,0.5843}, draw opacity={1.0}, fill={rgb,1:red,0.0627;green,0.4706;blue,0.5843}, fill opacity={1.0}, line width={0.75}, rotate={0}, solid}]
        table[row sep={\\}]
        {
            \\
            6.0  0.0286932642534446  \\
            6.0  0.110176550806759  \\
        }
        ;
    \addplot[color={rgb,1:red,0.0627;green,0.4706;blue,0.5843}, name path={60739e55-b72b-4c5c-9105-da6d2852b2a7}, draw opacity={1.0}, line width={1}, solid, mark={-}, mark size={1.5 pt}, mark repeat={1}, mark options={color={rgb,1:red,0.0627;green,0.4706;blue,0.5843}, draw opacity={1.0}, fill={rgb,1:red,0.0627;green,0.4706;blue,0.5843}, fill opacity={1.0}, line width={0.75}, rotate={0}, solid}]
        table[row sep={\\}]
        {
            \\
            7.0  0.021214191422085804  \\
            7.0  0.112285159671069  \\
        }
        ;
    \addplot[color={rgb,1:red,0.0627;green,0.4706;blue,0.5843}, name path={60739e55-b72b-4c5c-9105-da6d2852b2a7}, draw opacity={1.0}, line width={1}, solid, mark={-}, mark size={1.5 pt}, mark repeat={1}, mark options={color={rgb,1:red,0.0627;green,0.4706;blue,0.5843}, draw opacity={1.0}, fill={rgb,1:red,0.0627;green,0.4706;blue,0.5843}, fill opacity={1.0}, line width={0.75}, rotate={0}, solid}]
        table[row sep={\\}]
        {
            \\
            8.0  0.015316638146041603  \\
            8.0  0.1145520666677  \\
        }
        ;
    \addplot[color={rgb,1:red,0.0627;green,0.4706;blue,0.5843}, name path={60739e55-b72b-4c5c-9105-da6d2852b2a7}, draw opacity={1.0}, line width={1}, solid, mark={-}, mark size={1.5 pt}, mark repeat={1}, mark options={color={rgb,1:red,0.0627;green,0.4706;blue,0.5843}, draw opacity={1.0}, fill={rgb,1:red,0.0627;green,0.4706;blue,0.5843}, fill opacity={1.0}, line width={0.75}, rotate={0}, solid}]
        table[row sep={\\}]
        {
            \\
            9.0  0.002515969463864648  \\
            9.0  0.111992001303607  \\
        }
        ;
    \addplot[color={rgb,1:red,0.0627;green,0.4706;blue,0.5843}, name path={60739e55-b72b-4c5c-9105-da6d2852b2a7}, draw opacity={1.0}, line width={1}, solid, mark={-}, mark size={1.5 pt}, mark repeat={1}, mark options={color={rgb,1:red,0.0627;green,0.4706;blue,0.5843}, draw opacity={1.0}, fill={rgb,1:red,0.0627;green,0.4706;blue,0.5843}, fill opacity={1.0}, line width={0.75}, rotate={0}, solid}]
        table[row sep={\\}]
        {
            \\
            10.0  0.024469474372614805  \\
            10.0  0.149547623425218  \\
        }
        ;
    \addplot[color={rgb,1:red,0.0627;green,0.4706;blue,0.5843}, name path={60739e55-b72b-4c5c-9105-da6d2852b2a7}, draw opacity={1.0}, line width={1}, solid, mark={-}, mark size={1.5 pt}, mark repeat={1}, mark options={color={rgb,1:red,0.0627;green,0.4706;blue,0.5843}, draw opacity={1.0}, fill={rgb,1:red,0.0627;green,0.4706;blue,0.5843}, fill opacity={1.0}, line width={0.75}, rotate={0}, solid}]
        table[row sep={\\}]
        {
            \\
            11.0  0.002311966736266796  \\
            11.0  0.152220602148115  \\
        }
        ;
    \addplot[color={rgb,1:red,0.0627;green,0.4706;blue,0.5843}, name path={60739e55-b72b-4c5c-9105-da6d2852b2a7}, draw opacity={1.0}, line width={1}, solid, mark={-}, mark size={1.5 pt}, mark repeat={1}, mark options={color={rgb,1:red,0.0627;green,0.4706;blue,0.5843}, draw opacity={1.0}, fill={rgb,1:red,0.0627;green,0.4706;blue,0.5843}, fill opacity={1.0}, line width={0.75}, rotate={0}, solid}]
        table[row sep={\\}]
        {
            \\
            12.0  -0.0314674225825547  \\
            12.0  0.144283477569466  \\
        }
        ;
    \addplot[color={rgb,1:red,0.0627;green,0.4706;blue,0.5843}, name path={60739e55-b72b-4c5c-9105-da6d2852b2a7}, draw opacity={1.0}, line width={1}, solid, mark={-}, mark size={1.5 pt}, mark repeat={1}, mark options={color={rgb,1:red,0.0627;green,0.4706;blue,0.5843}, draw opacity={1.0}, fill={rgb,1:red,0.0627;green,0.4706;blue,0.5843}, fill opacity={1.0}, line width={0.75}, rotate={0}, solid}]
        table[row sep={\\}]
        {
            \\
            13.0  -0.0361890383150711  \\
            13.0  0.192166155242785  \\
        }
        ;
    \addplot[color={rgb,1:red,0.7644;green,0.4441;blue,0.8243}, name path={469c7c25-c495-4c35-ab13-07a08ab43b4e}, only marks, draw opacity={1.0}, line width={0}, solid, mark={*}, mark size={1.5 pt}, mark repeat={1}, mark options={color={rgb,1:red,0.0627;green,0.4706;blue,0.5843}, draw opacity={1.0}, fill={rgb,1:red,0.0627;green,0.4706;blue,0.5843}, fill opacity={1.0}, line width={0.75}, rotate={0}, solid}]
        table[row sep={\\}]
        {
            \\
            0.0  0.0495244677468895  \\
            1.0  0.0426541592451287  \\
            2.0  0.0433951218375789  \\
            3.0  0.0429109694777819  \\
            4.0  0.0483925671363952  \\
            5.0  0.0579181514620398  \\
            6.0  0.071032447449046  \\
            7.0  0.0672076071920829  \\
            8.0  0.0652815644560324  \\
            9.0  0.0597257485740327  \\
            10.0  0.0892035010983678  \\
            11.0  0.0800929541363465  \\
            12.0  0.0561462614179206  \\
            13.0  0.0791546156012972  \\
        }
        ;
\end{axis}
\end{tikzpicture}

    \end{adjustbox}
\end{subfigure}
\hfill
\begin{subfigure}[b]{0.45\textwidth}
    \caption{Generalized Imputation Estimator}
    \begin{adjustbox}{width=\textwidth, center}
        % Recommended preamble:
% \usetikzlibrary{arrows.meta}
% \usetikzlibrary{backgrounds}
% \usepgfplotslibrary{patchplots}
% \usepgfplotslibrary{fillbetween}
% \pgfplotsset{%
%     layers/standard/.define layer set={%
%         background,axis background,axis grid,axis ticks,axis lines,axis tick labels,pre main,main,axis descriptions,axis foreground%
%     }{
%         grid style={/pgfplots/on layer=axis grid},%
%         tick style={/pgfplots/on layer=axis ticks},%
%         axis line style={/pgfplots/on layer=axis lines},%
%         label style={/pgfplots/on layer=axis descriptions},%
%         legend style={/pgfplots/on layer=axis descriptions},%
%         title style={/pgfplots/on layer=axis descriptions},%
%         colorbar style={/pgfplots/on layer=axis descriptions},%
%         ticklabel style={/pgfplots/on layer=axis tick labels},%
%         axis background@ style={/pgfplots/on layer=axis background},%
%         3d box foreground style={/pgfplots/on layer=axis foreground},%
%     },
% }

\begin{tikzpicture}[/tikz/background rectangle/.style={fill={rgb,1:red,1.0;green,1.0;blue,1.0}, draw opacity={1.0}}, show background rectangle]
\begin{axis}[point meta max={nan}, point meta min={nan}, legend cell align={left}, legend columns={1}, title={}, title style={at={{(0.5,1)}}, anchor={south}, font={{\fontsize{14 pt}{18.2 pt}\selectfont}}, color={rgb,1:red,0.0;green,0.0;blue,0.0}, draw opacity={1.0}, rotate={0.0}, align={center}}, legend style={color={rgb,1:red,0.0;green,0.0;blue,0.0}, draw opacity={1.0}, line width={1}, solid, fill={rgb,1:red,1.0;green,1.0;blue,1.0}, fill opacity={1.0}, text opacity={1.0}, font={{\fontsize{8 pt}{10.4 pt}\selectfont}}, text={rgb,1:red,0.0;green,0.0;blue,0.0}, cells={anchor={center}}, at={(1.02, 1)}, anchor={north west}}, axis background/.style={fill={rgb,1:red,1.0;green,1.0;blue,1.0}, opacity={1.0}}, anchor={north west}, xshift={1.0mm}, yshift={-1.0mm}, width={97.06mm}, height={64.04mm}, scaled x ticks={false}, xlabel={Event Time}, x tick style={color={rgb,1:red,0.0;green,0.0;blue,0.0}, opacity={1.0}}, x tick label style={color={rgb,1:red,0.0;green,0.0;blue,0.0}, opacity={1.0}, rotate={0}}, xlabel style={at={(ticklabel cs:0.5)}, anchor=near ticklabel, at={{(ticklabel cs:0.5)}}, anchor={near ticklabel}, font={{\fontsize{11 pt}{14.3 pt}\selectfont}}, color={rgb,1:red,0.0;green,0.0;blue,0.0}, draw opacity={1.0}, rotate={0.0}}, xmajorgrids={true}, xmin={-23.05}, xmax={14.05}, xticklabels={{$-20$,$-10$,$0$,$10$}}, xtick={{-20.0,-10.0,0.0,10.0}}, xtick align={inside}, xticklabel style={font={{\fontsize{8 pt}{10.4 pt}\selectfont}}, color={rgb,1:red,0.0;green,0.0;blue,0.0}, draw opacity={1.0}, rotate={0.0}}, x grid style={color={rgb,1:red,0.0;green,0.0;blue,0.0}, draw opacity={0.1}, line width={0.5}, solid}, axis x line*={left}, x axis line style={color={rgb,1:red,0.0;green,0.0;blue,0.0}, draw opacity={1.0}, line width={1}, solid}, scaled y ticks={false}, ylabel={Coefficient}, y tick style={color={rgb,1:red,0.0;green,0.0;blue,0.0}, opacity={1.0}}, y tick label style={color={rgb,1:red,0.0;green,0.0;blue,0.0}, opacity={1.0}, rotate={0}}, ylabel style={at={(ticklabel cs:0.5)}, anchor=near ticklabel, at={{(ticklabel cs:0.5)}}, anchor={near ticklabel}, font={{\fontsize{11 pt}{14.3 pt}\selectfont}}, color={rgb,1:red,0.0;green,0.0;blue,0.0}, draw opacity={1.0}, rotate={0.0}}, ymajorgrids={true}, ymin={-0.4}, ymax={0.2}, yticklabels={{$-0.4$,$-0.3$,$-0.2$,$-0.1$,$0.0$,$0.1$,$0.2$}}, ytick={{-0.4,-0.30000000000000004,-0.2,-0.1,0.0,0.1,0.2}}, ytick align={inside}, yticklabel style={font={{\fontsize{8 pt}{10.4 pt}\selectfont}}, color={rgb,1:red,0.0;green,0.0;blue,0.0}, draw opacity={1.0}, rotate={0.0}}, y grid style={color={rgb,1:red,0.0;green,0.0;blue,0.0}, draw opacity={0.1}, line width={0.5}, solid}, axis y line*={left}, y axis line style={color={rgb,1:red,0.0;green,0.0;blue,0.0}, draw opacity={1.0}, line width={1}, solid}, colorbar={false}]
    \addplot[color={rgb,1:red,0.0;green,0.0;blue,0.0}, name path={197b274a-3ee2-4995-ab22-b92199c75668}, draw opacity={1.0}, line width={1}, dashed]
        table[row sep={\\}]
        {
            \\
            -60.150000000000006  0.0  \\
            51.150000000000006  0.0  \\
        }
        ;
    \addplot[color={rgb,1:red,0.6039;green,0.1412;blue,0.0824}, name path={97dc618a-d607-4232-b701-fd7d57a63bb2}, draw opacity={1.0}, line width={1}, solid]
        table[row sep={\\}]
        {
            \\
            -60.150000000000006  -0.013290136693084124  \\
            51.150000000000006  0.01850042694099662  \\
        }
        ;
    \addplot[color={rgb,1:red,0.6039;green,0.1412;blue,0.0824}, name path={9fc0c30a-0d8b-4803-9555-a3de7ab875bf}, draw opacity={1.0}, line width={1}, solid, mark={-}, mark size={1.5 pt}, mark repeat={1}, mark options={color={rgb,1:red,0.6039;green,0.1412;blue,0.0824}, draw opacity={1.0}, fill={rgb,1:red,0.6039;green,0.1412;blue,0.0824}, fill opacity={1.0}, line width={0.75}, rotate={0}, solid}]
        table[row sep={\\}]
        {
            \\
            -22.0  -0.07039263776487353  \\
            -22.0  0.0519742327478077  \\
        }
        ;
    \addplot[color={rgb,1:red,0.6039;green,0.1412;blue,0.0824}, name path={9fc0c30a-0d8b-4803-9555-a3de7ab875bf}, draw opacity={1.0}, line width={1}, solid, mark={-}, mark size={1.5 pt}, mark repeat={1}, mark options={color={rgb,1:red,0.6039;green,0.1412;blue,0.0824}, draw opacity={1.0}, fill={rgb,1:red,0.6039;green,0.1412;blue,0.0824}, fill opacity={1.0}, line width={0.75}, rotate={0}, solid}]
        table[row sep={\\}]
        {
            \\
            -21.0  -0.07914564127380512  \\
            -21.0  0.04296906011127828  \\
        }
        ;
    \addplot[color={rgb,1:red,0.6039;green,0.1412;blue,0.0824}, name path={9fc0c30a-0d8b-4803-9555-a3de7ab875bf}, draw opacity={1.0}, line width={1}, solid, mark={-}, mark size={1.5 pt}, mark repeat={1}, mark options={color={rgb,1:red,0.6039;green,0.1412;blue,0.0824}, draw opacity={1.0}, fill={rgb,1:red,0.6039;green,0.1412;blue,0.0824}, fill opacity={1.0}, line width={0.75}, rotate={0}, solid}]
        table[row sep={\\}]
        {
            \\
            -20.0  -0.010737901939283666  \\
            -20.0  0.08347584880018147  \\
        }
        ;
    \addplot[color={rgb,1:red,0.6039;green,0.1412;blue,0.0824}, name path={9fc0c30a-0d8b-4803-9555-a3de7ab875bf}, draw opacity={1.0}, line width={1}, solid, mark={-}, mark size={1.5 pt}, mark repeat={1}, mark options={color={rgb,1:red,0.6039;green,0.1412;blue,0.0824}, draw opacity={1.0}, fill={rgb,1:red,0.6039;green,0.1412;blue,0.0824}, fill opacity={1.0}, line width={0.75}, rotate={0}, solid}]
        table[row sep={\\}]
        {
            \\
            -19.0  -0.03126663322784508  \\
            -19.0  0.05091642428737554  \\
        }
        ;
    \addplot[color={rgb,1:red,0.6039;green,0.1412;blue,0.0824}, name path={9fc0c30a-0d8b-4803-9555-a3de7ab875bf}, draw opacity={1.0}, line width={1}, solid, mark={-}, mark size={1.5 pt}, mark repeat={1}, mark options={color={rgb,1:red,0.6039;green,0.1412;blue,0.0824}, draw opacity={1.0}, fill={rgb,1:red,0.6039;green,0.1412;blue,0.0824}, fill opacity={1.0}, line width={0.75}, rotate={0}, solid}]
        table[row sep={\\}]
        {
            \\
            -18.0  -0.011934372521427736  \\
            -18.0  0.05955524091632172  \\
        }
        ;
    \addplot[color={rgb,1:red,0.6039;green,0.1412;blue,0.0824}, name path={9fc0c30a-0d8b-4803-9555-a3de7ab875bf}, draw opacity={1.0}, line width={1}, solid, mark={-}, mark size={1.5 pt}, mark repeat={1}, mark options={color={rgb,1:red,0.6039;green,0.1412;blue,0.0824}, draw opacity={1.0}, fill={rgb,1:red,0.6039;green,0.1412;blue,0.0824}, fill opacity={1.0}, line width={0.75}, rotate={0}, solid}]
        table[row sep={\\}]
        {
            \\
            -17.0  -0.013165838174861846  \\
            -17.0  0.045275973499523454  \\
        }
        ;
    \addplot[color={rgb,1:red,0.6039;green,0.1412;blue,0.0824}, name path={9fc0c30a-0d8b-4803-9555-a3de7ab875bf}, draw opacity={1.0}, line width={1}, solid, mark={-}, mark size={1.5 pt}, mark repeat={1}, mark options={color={rgb,1:red,0.6039;green,0.1412;blue,0.0824}, draw opacity={1.0}, fill={rgb,1:red,0.6039;green,0.1412;blue,0.0824}, fill opacity={1.0}, line width={0.75}, rotate={0}, solid}]
        table[row sep={\\}]
        {
            \\
            -16.0  -0.0093638966246982  \\
            -16.0  0.04365739894616706  \\
        }
        ;
    \addplot[color={rgb,1:red,0.6039;green,0.1412;blue,0.0824}, name path={9fc0c30a-0d8b-4803-9555-a3de7ab875bf}, draw opacity={1.0}, line width={1}, solid, mark={-}, mark size={1.5 pt}, mark repeat={1}, mark options={color={rgb,1:red,0.6039;green,0.1412;blue,0.0824}, draw opacity={1.0}, fill={rgb,1:red,0.6039;green,0.1412;blue,0.0824}, fill opacity={1.0}, line width={0.75}, rotate={0}, solid}]
        table[row sep={\\}]
        {
            \\
            -15.0  -0.01963009312764935  \\
            -15.0  0.01954558840433373  \\
        }
        ;
    \addplot[color={rgb,1:red,0.6039;green,0.1412;blue,0.0824}, name path={9fc0c30a-0d8b-4803-9555-a3de7ab875bf}, draw opacity={1.0}, line width={1}, solid, mark={-}, mark size={1.5 pt}, mark repeat={1}, mark options={color={rgb,1:red,0.6039;green,0.1412;blue,0.0824}, draw opacity={1.0}, fill={rgb,1:red,0.6039;green,0.1412;blue,0.0824}, fill opacity={1.0}, line width={0.75}, rotate={0}, solid}]
        table[row sep={\\}]
        {
            \\
            -14.0  -0.015451124627806914  \\
            -14.0  0.018511773727996346  \\
        }
        ;
    \addplot[color={rgb,1:red,0.6039;green,0.1412;blue,0.0824}, name path={9fc0c30a-0d8b-4803-9555-a3de7ab875bf}, draw opacity={1.0}, line width={1}, solid, mark={-}, mark size={1.5 pt}, mark repeat={1}, mark options={color={rgb,1:red,0.6039;green,0.1412;blue,0.0824}, draw opacity={1.0}, fill={rgb,1:red,0.6039;green,0.1412;blue,0.0824}, fill opacity={1.0}, line width={0.75}, rotate={0}, solid}]
        table[row sep={\\}]
        {
            \\
            -13.0  -0.015739256817929603  \\
            -13.0  0.01039414876623459  \\
        }
        ;
    \addplot[color={rgb,1:red,0.6039;green,0.1412;blue,0.0824}, name path={9fc0c30a-0d8b-4803-9555-a3de7ab875bf}, draw opacity={1.0}, line width={1}, solid, mark={-}, mark size={1.5 pt}, mark repeat={1}, mark options={color={rgb,1:red,0.6039;green,0.1412;blue,0.0824}, draw opacity={1.0}, fill={rgb,1:red,0.6039;green,0.1412;blue,0.0824}, fill opacity={1.0}, line width={0.75}, rotate={0}, solid}]
        table[row sep={\\}]
        {
            \\
            -12.0  -0.0074453598577546645  \\
            -12.0  0.017686071757078294  \\
        }
        ;
    \addplot[color={rgb,1:red,0.6039;green,0.1412;blue,0.0824}, name path={9fc0c30a-0d8b-4803-9555-a3de7ab875bf}, draw opacity={1.0}, line width={1}, solid, mark={-}, mark size={1.5 pt}, mark repeat={1}, mark options={color={rgb,1:red,0.6039;green,0.1412;blue,0.0824}, draw opacity={1.0}, fill={rgb,1:red,0.6039;green,0.1412;blue,0.0824}, fill opacity={1.0}, line width={0.75}, rotate={0}, solid}]
        table[row sep={\\}]
        {
            \\
            -11.0  -0.003257587461707666  \\
            -11.0  0.019376297551713974  \\
        }
        ;
    \addplot[color={rgb,1:red,0.6039;green,0.1412;blue,0.0824}, name path={9fc0c30a-0d8b-4803-9555-a3de7ab875bf}, draw opacity={1.0}, line width={1}, solid, mark={-}, mark size={1.5 pt}, mark repeat={1}, mark options={color={rgb,1:red,0.6039;green,0.1412;blue,0.0824}, draw opacity={1.0}, fill={rgb,1:red,0.6039;green,0.1412;blue,0.0824}, fill opacity={1.0}, line width={0.75}, rotate={0}, solid}]
        table[row sep={\\}]
        {
            \\
            -10.0  -0.011354135827520285  \\
            -10.0  0.011684531596160634  \\
        }
        ;
    \addplot[color={rgb,1:red,0.6039;green,0.1412;blue,0.0824}, name path={9fc0c30a-0d8b-4803-9555-a3de7ab875bf}, draw opacity={1.0}, line width={1}, solid, mark={-}, mark size={1.5 pt}, mark repeat={1}, mark options={color={rgb,1:red,0.6039;green,0.1412;blue,0.0824}, draw opacity={1.0}, fill={rgb,1:red,0.6039;green,0.1412;blue,0.0824}, fill opacity={1.0}, line width={0.75}, rotate={0}, solid}]
        table[row sep={\\}]
        {
            \\
            -9.0  -0.015519660425696153  \\
            -9.0  0.008726578909375688  \\
        }
        ;
    \addplot[color={rgb,1:red,0.6039;green,0.1412;blue,0.0824}, name path={9fc0c30a-0d8b-4803-9555-a3de7ab875bf}, draw opacity={1.0}, line width={1}, solid, mark={-}, mark size={1.5 pt}, mark repeat={1}, mark options={color={rgb,1:red,0.6039;green,0.1412;blue,0.0824}, draw opacity={1.0}, fill={rgb,1:red,0.6039;green,0.1412;blue,0.0824}, fill opacity={1.0}, line width={0.75}, rotate={0}, solid}]
        table[row sep={\\}]
        {
            \\
            -8.0  -0.018991849262065423  \\
            -8.0  0.006540926076645646  \\
        }
        ;
    \addplot[color={rgb,1:red,0.6039;green,0.1412;blue,0.0824}, name path={9fc0c30a-0d8b-4803-9555-a3de7ab875bf}, draw opacity={1.0}, line width={1}, solid, mark={-}, mark size={1.5 pt}, mark repeat={1}, mark options={color={rgb,1:red,0.6039;green,0.1412;blue,0.0824}, draw opacity={1.0}, fill={rgb,1:red,0.6039;green,0.1412;blue,0.0824}, fill opacity={1.0}, line width={0.75}, rotate={0}, solid}]
        table[row sep={\\}]
        {
            \\
            -7.0  -0.01568892352240435  \\
            -7.0  0.012338244495385852  \\
        }
        ;
    \addplot[color={rgb,1:red,0.6039;green,0.1412;blue,0.0824}, name path={9fc0c30a-0d8b-4803-9555-a3de7ab875bf}, draw opacity={1.0}, line width={1}, solid, mark={-}, mark size={1.5 pt}, mark repeat={1}, mark options={color={rgb,1:red,0.6039;green,0.1412;blue,0.0824}, draw opacity={1.0}, fill={rgb,1:red,0.6039;green,0.1412;blue,0.0824}, fill opacity={1.0}, line width={0.75}, rotate={0}, solid}]
        table[row sep={\\}]
        {
            \\
            -6.0  -0.010024142987784734  \\
            -6.0  0.016266865567346984  \\
        }
        ;
    \addplot[color={rgb,1:red,0.6039;green,0.1412;blue,0.0824}, name path={9fc0c30a-0d8b-4803-9555-a3de7ab875bf}, draw opacity={1.0}, line width={1}, solid, mark={-}, mark size={1.5 pt}, mark repeat={1}, mark options={color={rgb,1:red,0.6039;green,0.1412;blue,0.0824}, draw opacity={1.0}, fill={rgb,1:red,0.6039;green,0.1412;blue,0.0824}, fill opacity={1.0}, line width={0.75}, rotate={0}, solid}]
        table[row sep={\\}]
        {
            \\
            -5.0  -0.015866053544279717  \\
            -5.0  0.009514508929115519  \\
        }
        ;
    \addplot[color={rgb,1:red,0.6039;green,0.1412;blue,0.0824}, name path={9fc0c30a-0d8b-4803-9555-a3de7ab875bf}, draw opacity={1.0}, line width={1}, solid, mark={-}, mark size={1.5 pt}, mark repeat={1}, mark options={color={rgb,1:red,0.6039;green,0.1412;blue,0.0824}, draw opacity={1.0}, fill={rgb,1:red,0.6039;green,0.1412;blue,0.0824}, fill opacity={1.0}, line width={0.75}, rotate={0}, solid}]
        table[row sep={\\}]
        {
            \\
            -4.0  -0.008062579349061839  \\
            -4.0  0.021682803194609808  \\
        }
        ;
    \addplot[color={rgb,1:red,0.6039;green,0.1412;blue,0.0824}, name path={9fc0c30a-0d8b-4803-9555-a3de7ab875bf}, draw opacity={1.0}, line width={1}, solid, mark={-}, mark size={1.5 pt}, mark repeat={1}, mark options={color={rgb,1:red,0.6039;green,0.1412;blue,0.0824}, draw opacity={1.0}, fill={rgb,1:red,0.6039;green,0.1412;blue,0.0824}, fill opacity={1.0}, line width={0.75}, rotate={0}, solid}]
        table[row sep={\\}]
        {
            \\
            -3.0  -0.007918929572663054  \\
            -3.0  0.022957715798505876  \\
        }
        ;
    \addplot[color={rgb,1:red,0.6039;green,0.1412;blue,0.0824}, name path={9fc0c30a-0d8b-4803-9555-a3de7ab875bf}, draw opacity={1.0}, line width={1}, solid, mark={-}, mark size={1.5 pt}, mark repeat={1}, mark options={color={rgb,1:red,0.6039;green,0.1412;blue,0.0824}, draw opacity={1.0}, fill={rgb,1:red,0.6039;green,0.1412;blue,0.0824}, fill opacity={1.0}, line width={0.75}, rotate={0}, solid}]
        table[row sep={\\}]
        {
            \\
            -2.0  -0.01228874873003893  \\
            -2.0  0.023065959854547274  \\
        }
        ;
    \addplot[color={rgb,1:red,0.6039;green,0.1412;blue,0.0824}, name path={9fc0c30a-0d8b-4803-9555-a3de7ab875bf}, draw opacity={1.0}, line width={1}, solid, mark={-}, mark size={1.5 pt}, mark repeat={1}, mark options={color={rgb,1:red,0.6039;green,0.1412;blue,0.0824}, draw opacity={1.0}, fill={rgb,1:red,0.6039;green,0.1412;blue,0.0824}, fill opacity={1.0}, line width={0.75}, rotate={0}, solid}]
        table[row sep={\\}]
        {
            \\
            -1.0  -0.018925345650982313  \\
            -1.0  0.02603503208698143  \\
        }
        ;
    \addplot[color={rgb,1:red,0.2422;green,0.6433;blue,0.3044}, name path={c8e6341d-b7e5-440d-bb77-02e7b5e31bd6}, only marks, draw opacity={1.0}, line width={0}, solid, mark={*}, mark size={1.5 pt}, mark repeat={1}, mark options={color={rgb,1:red,0.6039;green,0.1412;blue,0.0824}, draw opacity={1.0}, fill={rgb,1:red,0.6039;green,0.1412;blue,0.0824}, fill opacity={1.0}, line width={0.75}, rotate={0}, solid}]
        table[row sep={\\}]
        {
            \\
            -22.0  -0.009209202508532917  \\
            -21.0  -0.01808829058126342  \\
            -20.0  0.0363689734304489  \\
            -19.0  0.00982489552976523  \\
            -18.0  0.02381043419744699  \\
            -17.0  0.016055067662330804  \\
            -16.0  0.01714675116073443  \\
            -15.0  -4.225236165781047e-5  \\
            -14.0  0.0015303245500947152  \\
            -13.0  -0.0026725540258475057  \\
            -12.0  0.0051203559496618145  \\
            -11.0  0.008059355045003153  \\
            -10.0  0.00016519788432017523  \\
            -9.0  -0.003396540758160233  \\
            -8.0  -0.006225461592709889  \\
            -7.0  -0.001675339513509248  \\
            -6.0  0.0031213612897811253  \\
            -5.0  -0.0031757723075820986  \\
            -4.0  0.006810111922773984  \\
            -3.0  0.00751939311292141  \\
            -2.0  0.0053886055622541715  \\
            -1.0  0.0035548432179995566  \\
        }
        ;
    \addplot[color={rgb,1:red,0.0627;green,0.4706;blue,0.5843}, name path={08d1ff0e-e807-4e1f-826a-9adf872c2f5c}, draw opacity={1.0}, line width={1}, solid, mark={-}, mark size={1.5 pt}, mark repeat={1}, mark options={color={rgb,1:red,0.0627;green,0.4706;blue,0.5843}, draw opacity={1.0}, fill={rgb,1:red,0.0627;green,0.4706;blue,0.5843}, fill opacity={1.0}, line width={0.75}, rotate={0}, solid}]
        table[row sep={\\}]
        {
            \\
            0.0  -0.02558127123913143  \\
            0.0  0.032515294245233486  \\
        }
        ;
    \addplot[color={rgb,1:red,0.0627;green,0.4706;blue,0.5843}, name path={08d1ff0e-e807-4e1f-826a-9adf872c2f5c}, draw opacity={1.0}, line width={1}, solid, mark={-}, mark size={1.5 pt}, mark repeat={1}, mark options={color={rgb,1:red,0.0627;green,0.4706;blue,0.5843}, draw opacity={1.0}, fill={rgb,1:red,0.0627;green,0.4706;blue,0.5843}, fill opacity={1.0}, line width={0.75}, rotate={0}, solid}]
        table[row sep={\\}]
        {
            \\
            1.0  -0.053629293361632094  \\
            1.0  0.02350950850599552  \\
        }
        ;
    \addplot[color={rgb,1:red,0.0627;green,0.4706;blue,0.5843}, name path={08d1ff0e-e807-4e1f-826a-9adf872c2f5c}, draw opacity={1.0}, line width={1}, solid, mark={-}, mark size={1.5 pt}, mark repeat={1}, mark options={color={rgb,1:red,0.0627;green,0.4706;blue,0.5843}, draw opacity={1.0}, fill={rgb,1:red,0.0627;green,0.4706;blue,0.5843}, fill opacity={1.0}, line width={0.75}, rotate={0}, solid}]
        table[row sep={\\}]
        {
            \\
            2.0  -0.0747144690229932  \\
            2.0  0.021110769075139167  \\
        }
        ;
    \addplot[color={rgb,1:red,0.0627;green,0.4706;blue,0.5843}, name path={08d1ff0e-e807-4e1f-826a-9adf872c2f5c}, draw opacity={1.0}, line width={1}, solid, mark={-}, mark size={1.5 pt}, mark repeat={1}, mark options={color={rgb,1:red,0.0627;green,0.4706;blue,0.5843}, draw opacity={1.0}, fill={rgb,1:red,0.0627;green,0.4706;blue,0.5843}, fill opacity={1.0}, line width={0.75}, rotate={0}, solid}]
        table[row sep={\\}]
        {
            \\
            3.0  -0.09629783945360876  \\
            3.0  0.02017388144229007  \\
        }
        ;
    \addplot[color={rgb,1:red,0.0627;green,0.4706;blue,0.5843}, name path={08d1ff0e-e807-4e1f-826a-9adf872c2f5c}, draw opacity={1.0}, line width={1}, solid, mark={-}, mark size={1.5 pt}, mark repeat={1}, mark options={color={rgb,1:red,0.0627;green,0.4706;blue,0.5843}, draw opacity={1.0}, fill={rgb,1:red,0.0627;green,0.4706;blue,0.5843}, fill opacity={1.0}, line width={0.75}, rotate={0}, solid}]
        table[row sep={\\}]
        {
            \\
            4.0  -0.11311252135769173  \\
            4.0  0.023633976136678315  \\
        }
        ;
    \addplot[color={rgb,1:red,0.0627;green,0.4706;blue,0.5843}, name path={08d1ff0e-e807-4e1f-826a-9adf872c2f5c}, draw opacity={1.0}, line width={1}, solid, mark={-}, mark size={1.5 pt}, mark repeat={1}, mark options={color={rgb,1:red,0.0627;green,0.4706;blue,0.5843}, draw opacity={1.0}, fill={rgb,1:red,0.0627;green,0.4706;blue,0.5843}, fill opacity={1.0}, line width={0.75}, rotate={0}, solid}]
        table[row sep={\\}]
        {
            \\
            5.0  -0.13382795518957194  \\
            5.0  0.029232150943564454  \\
        }
        ;
    \addplot[color={rgb,1:red,0.0627;green,0.4706;blue,0.5843}, name path={08d1ff0e-e807-4e1f-826a-9adf872c2f5c}, draw opacity={1.0}, line width={1}, solid, mark={-}, mark size={1.5 pt}, mark repeat={1}, mark options={color={rgb,1:red,0.0627;green,0.4706;blue,0.5843}, draw opacity={1.0}, fill={rgb,1:red,0.0627;green,0.4706;blue,0.5843}, fill opacity={1.0}, line width={0.75}, rotate={0}, solid}]
        table[row sep={\\}]
        {
            \\
            6.0  -0.14473272248033603  \\
            6.0  0.04419978527155224  \\
        }
        ;
    \addplot[color={rgb,1:red,0.0627;green,0.4706;blue,0.5843}, name path={08d1ff0e-e807-4e1f-826a-9adf872c2f5c}, draw opacity={1.0}, line width={1}, solid, mark={-}, mark size={1.5 pt}, mark repeat={1}, mark options={color={rgb,1:red,0.0627;green,0.4706;blue,0.5843}, draw opacity={1.0}, fill={rgb,1:red,0.0627;green,0.4706;blue,0.5843}, fill opacity={1.0}, line width={0.75}, rotate={0}, solid}]
        table[row sep={\\}]
        {
            \\
            7.0  -0.1398749615005718  \\
            7.0  0.05839623080742015  \\
        }
        ;
    \addplot[color={rgb,1:red,0.0627;green,0.4706;blue,0.5843}, name path={08d1ff0e-e807-4e1f-826a-9adf872c2f5c}, draw opacity={1.0}, line width={1}, solid, mark={-}, mark size={1.5 pt}, mark repeat={1}, mark options={color={rgb,1:red,0.0627;green,0.4706;blue,0.5843}, draw opacity={1.0}, fill={rgb,1:red,0.0627;green,0.4706;blue,0.5843}, fill opacity={1.0}, line width={0.75}, rotate={0}, solid}]
        table[row sep={\\}]
        {
            \\
            8.0  -0.16411767396461938  \\
            8.0  0.0513997028336253  \\
        }
        ;
    \addplot[color={rgb,1:red,0.0627;green,0.4706;blue,0.5843}, name path={08d1ff0e-e807-4e1f-826a-9adf872c2f5c}, draw opacity={1.0}, line width={1}, solid, mark={-}, mark size={1.5 pt}, mark repeat={1}, mark options={color={rgb,1:red,0.0627;green,0.4706;blue,0.5843}, draw opacity={1.0}, fill={rgb,1:red,0.0627;green,0.4706;blue,0.5843}, fill opacity={1.0}, line width={0.75}, rotate={0}, solid}]
        table[row sep={\\}]
        {
            \\
            9.0  -0.1686821117709031  \\
            9.0  0.050793210679865274  \\
        }
        ;
    \addplot[color={rgb,1:red,0.0627;green,0.4706;blue,0.5843}, name path={08d1ff0e-e807-4e1f-826a-9adf872c2f5c}, draw opacity={1.0}, line width={1}, solid, mark={-}, mark size={1.5 pt}, mark repeat={1}, mark options={color={rgb,1:red,0.0627;green,0.4706;blue,0.5843}, draw opacity={1.0}, fill={rgb,1:red,0.0627;green,0.4706;blue,0.5843}, fill opacity={1.0}, line width={0.75}, rotate={0}, solid}]
        table[row sep={\\}]
        {
            \\
            10.0  -0.197307904021417  \\
            10.0  0.06606893355270488  \\
        }
        ;
    \addplot[color={rgb,1:red,0.0627;green,0.4706;blue,0.5843}, name path={08d1ff0e-e807-4e1f-826a-9adf872c2f5c}, draw opacity={1.0}, line width={1}, solid, mark={-}, mark size={1.5 pt}, mark repeat={1}, mark options={color={rgb,1:red,0.0627;green,0.4706;blue,0.5843}, draw opacity={1.0}, fill={rgb,1:red,0.0627;green,0.4706;blue,0.5843}, fill opacity={1.0}, line width={0.75}, rotate={0}, solid}]
        table[row sep={\\}]
        {
            \\
            11.0  -0.2876587850948969  \\
            11.0  0.03580532229186267  \\
        }
        ;
    \addplot[color={rgb,1:red,0.0627;green,0.4706;blue,0.5843}, name path={08d1ff0e-e807-4e1f-826a-9adf872c2f5c}, draw opacity={1.0}, line width={1}, solid, mark={-}, mark size={1.5 pt}, mark repeat={1}, mark options={color={rgb,1:red,0.0627;green,0.4706;blue,0.5843}, draw opacity={1.0}, fill={rgb,1:red,0.0627;green,0.4706;blue,0.5843}, fill opacity={1.0}, line width={0.75}, rotate={0}, solid}]
        table[row sep={\\}]
        {
            \\
            12.0  -0.3557384771402911  \\
            12.0  0.014402140424856602  \\
        }
        ;
    \addplot[color={rgb,1:red,0.0627;green,0.4706;blue,0.5843}, name path={08d1ff0e-e807-4e1f-826a-9adf872c2f5c}, draw opacity={1.0}, line width={1}, solid, mark={-}, mark size={1.5 pt}, mark repeat={1}, mark options={color={rgb,1:red,0.0627;green,0.4706;blue,0.5843}, draw opacity={1.0}, fill={rgb,1:red,0.0627;green,0.4706;blue,0.5843}, fill opacity={1.0}, line width={0.75}, rotate={0}, solid}]
        table[row sep={\\}]
        {
            \\
            13.0  -0.3509524074261113  \\
            13.0  0.08684460488662388  \\
        }
        ;
    \addplot[color={rgb,1:red,0.7644;green,0.4441;blue,0.8243}, name path={47a1fd1c-a16e-4df7-b075-1864f0c90230}, only marks, draw opacity={1.0}, line width={0}, solid, mark={*}, mark size={1.5 pt}, mark repeat={1}, mark options={color={rgb,1:red,0.0627;green,0.4706;blue,0.5843}, draw opacity={1.0}, fill={rgb,1:red,0.0627;green,0.4706;blue,0.5843}, fill opacity={1.0}, line width={0.75}, rotate={0}, solid}]
        table[row sep={\\}]
        {
            \\
            0.0  0.0034670115030510304  \\
            1.0  -0.015059892427818288  \\
            2.0  -0.02680184997392701  \\
            3.0  -0.038061979005659344  \\
            4.0  -0.044739272610506704  \\
            5.0  -0.052297902123003746  \\
            6.0  -0.0502664686043919  \\
            7.0  -0.04073936534657582  \\
            8.0  -0.05635898556549704  \\
            9.0  -0.05894445054551891  \\
            10.0  -0.06561948523435607  \\
            11.0  -0.1259267314015171  \\
            12.0  -0.17066816835771725  \\
            13.0  -0.1320539012697437  \\
        }
        ;
\end{axis}
\end{tikzpicture}

    \end{adjustbox}
\end{subfigure}
\end{figure}
\begin{itemize}
    \item Average coefficient estimate of $-6\%$. 
    \begin{itemize}
        \item Very noisy.
        \item Also in line with Basker (2005). 
    \end{itemize}
\end{itemize}
\end{frame}

%%%%%%%%%%%%%%%%%%%%%%%%%%%%%%%%%%%%%%%%%%%%%%%%%%%%%%%%%%%%%%%%%%%%%%%

\begin{frame}{Results}
    \begin{figure}
\caption*{Synthetic Control Style Plot of the Effect of Walmart on County Employment}
\label{fig:synthetic_control_plot}

\begin{subfigure}[b]{0.45\textwidth}
    \caption{$\log$ Retail Employment}
    \begin{adjustbox}{width=\textwidth, center}
        % Recommended preamble:
% \usetikzlibrary{arrows.meta}
% \usetikzlibrary{backgrounds}
% \usepgfplotslibrary{patchplots}
% \usepgfplotslibrary{fillbetween}
% \pgfplotsset{%
%     layers/standard/.define layer set={%
%         background,axis background,axis grid,axis ticks,axis lines,axis tick labels,pre main,main,axis descriptions,axis foreground%
%     }{
%         grid style={/pgfplots/on layer=axis grid},%
%         tick style={/pgfplots/on layer=axis ticks},%
%         axis line style={/pgfplots/on layer=axis lines},%
%         label style={/pgfplots/on layer=axis descriptions},%
%         legend style={/pgfplots/on layer=axis descriptions},%
%         title style={/pgfplots/on layer=axis descriptions},%
%         colorbar style={/pgfplots/on layer=axis descriptions},%
%         ticklabel style={/pgfplots/on layer=axis tick labels},%
%         axis background@ style={/pgfplots/on layer=axis background},%
%         3d box foreground style={/pgfplots/on layer=axis foreground},%
%     },
% }

\begin{tikzpicture}[/tikz/background rectangle/.style={fill={rgb,1:red,1.0;green,1.0;blue,1.0}, draw opacity={1.0}}, show background rectangle]
\begin{axis}[point meta max={nan}, point meta min={nan}, legend cell align={left}, legend columns={1}, title={}, title style={at={{(0.5,1)}}, anchor={south}, font={{\fontsize{14 pt}{18.2 pt}\selectfont}}, color={rgb,1:red,0.0;green,0.0;blue,0.0}, draw opacity={1.0}, rotate={0.0}, align={center}}, legend style={color={rgb,1:red,0.0;green,0.0;blue,0.0}, draw opacity={1.0}, line width={1}, solid, fill={rgb,1:red,1.0;green,1.0;blue,1.0}, fill opacity={1.0}, text opacity={1.0}, font={{\fontsize{8 pt}{10.4 pt}\selectfont}}, text={rgb,1:red,0.0;green,0.0;blue,0.0}, cells={anchor={center}}, at={(0.02, 0.98)}, anchor={north west}}, axis background/.style={fill={rgb,1:red,1.0;green,1.0;blue,1.0}, opacity={1.0}}, anchor={north west}, xshift={1.0mm}, yshift={-1.0mm}, width={112.3mm}, height={55.15mm}, scaled x ticks={false}, xlabel={Event Time}, x tick style={color={rgb,1:red,0.0;green,0.0;blue,0.0}, opacity={1.0}}, x tick label style={color={rgb,1:red,0.0;green,0.0;blue,0.0}, opacity={1.0}, rotate={0}}, xlabel style={at={(ticklabel cs:0.5)}, anchor=near ticklabel, at={{(ticklabel cs:0.5)}}, anchor={near ticklabel}, font={{\fontsize{11 pt}{14.3 pt}\selectfont}}, color={rgb,1:red,0.0;green,0.0;blue,0.0}, draw opacity={1.0}, rotate={0.0}}, xmajorgrids={true}, xmin={-23.05}, xmax={14.05}, xticklabels={{$-20$,$-10$,$0$,$10$}}, xtick={{-20.0,-10.0,0.0,10.0}}, xtick align={inside}, xticklabel style={font={{\fontsize{8 pt}{10.4 pt}\selectfont}}, color={rgb,1:red,0.0;green,0.0;blue,0.0}, draw opacity={1.0}, rotate={0.0}}, x grid style={color={rgb,1:red,0.0;green,0.0;blue,0.0}, draw opacity={0.1}, line width={0.5}, solid}, axis x line*={left}, x axis line style={color={rgb,1:red,0.0;green,0.0;blue,0.0}, draw opacity={1.0}, line width={1}, solid}, scaled y ticks={false}, ylabel={}, y tick style={color={rgb,1:red,0.0;green,0.0;blue,0.0}, opacity={1.0}}, y tick label style={color={rgb,1:red,0.0;green,0.0;blue,0.0}, opacity={1.0}, rotate={0}}, ylabel style={at={(ticklabel cs:0.5)}, anchor=near ticklabel, at={{(ticklabel cs:0.5)}}, anchor={near ticklabel}, font={{\fontsize{11 pt}{14.3 pt}\selectfont}}, color={rgb,1:red,0.0;green,0.0;blue,0.0}, draw opacity={1.0}, rotate={0.0}}, ymajorgrids={true}, ymin={-0.06715525754858861}, ymax={0.22359953092582316}, yticklabels={{$-0.05$,$0.00$,$0.05$,$0.10$,$0.15$,$0.20$}}, ytick={{-0.05000000000000001,0.0,0.05000000000000001,0.10000000000000002,0.15000000000000002,0.20000000000000004}}, ytick align={inside}, yticklabel style={font={{\fontsize{8 pt}{10.4 pt}\selectfont}}, color={rgb,1:red,0.0;green,0.0;blue,0.0}, draw opacity={1.0}, rotate={0.0}}, y grid style={color={rgb,1:red,0.0;green,0.0;blue,0.0}, draw opacity={0.1}, line width={0.5}, solid}, axis y line*={left}, y axis line style={color={rgb,1:red,0.0;green,0.0;blue,0.0}, draw opacity={1.0}, line width={1}, solid}, colorbar={false}]
    \addplot[color={rgb,1:red,0.6039;green,0.1412;blue,0.0824}, name path={0a2cbe96-4027-4469-896e-1c6c86c633af}, draw opacity={1.0}, line width={1.4}, solid]
        table[row sep={\\}]
        {
            \\
            -22.0  -0.05892634844082224  \\
            -21.0  -0.04562546449707075  \\
            -20.0  -0.03858102515522281  \\
            -19.0  -0.027269685097433924  \\
            -18.0  -0.020237974418263454  \\
            -17.0  -0.017340424127545412  \\
            -16.0  -0.00994274890564034  \\
            -15.0  -0.004463372382019268  \\
            -14.0  -0.0022634587353578413  \\
            -13.0  -0.004616910679576958  \\
            -12.0  -0.000797858887193037  \\
            -11.0  0.004216900611929656  \\
            -10.0  0.011320103591108362  \\
            -9.0  0.015281848364703978  \\
            -8.0  0.02693148721710699  \\
            -7.0  0.03886627295410842  \\
            -6.0  0.047462260104242496  \\
            -5.0  0.05883368093022003  \\
            -4.0  0.06987050292254388  \\
            -3.0  0.07492618214284102  \\
            -2.0  0.07964496727770305  \\
            -1.0  0.08286550500746484  \\
            0.0  0.10769318563121515  \\
            1.0  0.13807828444928244  \\
            2.0  0.13661164444593782  \\
            3.0  0.13460411137060474  \\
            4.0  0.13615868465658046  \\
            5.0  0.14455015580690844  \\
            6.0  0.1527053398356005  \\
            7.0  0.16522451587366585  \\
            8.0  0.17317469966443155  \\
            9.0  0.17189123977535048  \\
            10.0  0.18862050529042304  \\
            11.0  0.18979666324306652  \\
            12.0  0.18460662103596182  \\
            13.0  0.21537062181805677  \\
        }
        ;
    \addlegendentry {Average of $\tilde{y}_{it}$}
    \addplot[color={rgb,1:red,0.0627;green,0.4706;blue,0.5843}, name path={9618621a-408b-4db5-a7f5-7dd8c3220982}, draw opacity={1.0}, line width={1.4}, solid]
        table[row sep={\\}]
        {
            \\
            -22.0  -0.032684565911096655  \\
            -21.0  -0.019400194921610266  \\
            -20.0  -0.03168137596398499  \\
            -19.0  -0.02262475937661108  \\
            -18.0  -0.02053871051609726  \\
            -17.0  -0.014609709744938729  \\
            -16.0  -0.017869073084176797  \\
            -15.0  -0.010146355259172466  \\
            -14.0  -0.00933759457543311  \\
            -13.0  -0.001563612352735152  \\
            -12.0  0.004207956988893188  \\
            -11.0  0.005893244885794144  \\
            -10.0  0.013005800602391935  \\
            -9.0  0.017478418015503868  \\
            -8.0  0.02796024148386812  \\
            -7.0  0.03504682300537381  \\
            -6.0  0.04655596249685082  \\
            -5.0  0.05596630492498878  \\
            -4.0  0.06506189904446096  \\
            -3.0  0.0730013857968543  \\
            -2.0  0.08028468532611678  \\
            -1.0  0.08638258273443368  \\
            0.0  0.09346365015330285  \\
            1.0  0.09512574221633611  \\
            2.0  0.09548385395683853  \\
            3.0  0.09661042264165963  \\
            4.0  0.09412892127441104  \\
            5.0  0.09545814974984958  \\
            6.0  0.0927101291325353  \\
            7.0  0.09070445424237983  \\
            8.0  0.08727895531221316  \\
            9.0  0.0827616508778645  \\
            10.0  0.11151491496815152  \\
            11.0  0.12481672556205874  \\
            12.0  0.12519075035059704  \\
            13.0  0.1072041645166839  \\
        }
        ;
    \addlegendentry {Average of $\hat{\tilde{y}}_{it}(0)$}
    \addplot[color={rgb,1:red,0.0;green,0.0;blue,0.0}, name path={c1e5b114-b82d-4306-bd44-dd151d34333c}, draw opacity={1.0}, line width={1}, dashed, forget plot]
        table[row sep={\\}]
        {
            \\
            -0.5  -0.35791004602300036  \\
            -0.5  0.5143543194002349  \\
        }
        ;
\end{axis}
\end{tikzpicture}

    \end{adjustbox}
\end{subfigure}
\hfill
\begin{subfigure}[b]{0.45\textwidth}
    \caption{$\log$ Wholesale Retail Employment}
    \begin{adjustbox}{width=\textwidth, center}
        % Recommended preamble:
% \usetikzlibrary{arrows.meta}
% \usetikzlibrary{backgrounds}
% \usepgfplotslibrary{patchplots}
% \usepgfplotslibrary{fillbetween}
% \pgfplotsset{%
%     layers/standard/.define layer set={%
%         background,axis background,axis grid,axis ticks,axis lines,axis tick labels,pre main,main,axis descriptions,axis foreground%
%     }{
%         grid style={/pgfplots/on layer=axis grid},%
%         tick style={/pgfplots/on layer=axis ticks},%
%         axis line style={/pgfplots/on layer=axis lines},%
%         label style={/pgfplots/on layer=axis descriptions},%
%         legend style={/pgfplots/on layer=axis descriptions},%
%         title style={/pgfplots/on layer=axis descriptions},%
%         colorbar style={/pgfplots/on layer=axis descriptions},%
%         ticklabel style={/pgfplots/on layer=axis tick labels},%
%         axis background@ style={/pgfplots/on layer=axis background},%
%         3d box foreground style={/pgfplots/on layer=axis foreground},%
%     },
% }

\begin{tikzpicture}[/tikz/background rectangle/.style={fill={rgb,1:red,1.0;green,1.0;blue,1.0}, draw opacity={1.0}}, show background rectangle]
\begin{axis}[point meta max={nan}, point meta min={nan}, legend cell align={left}, legend columns={1}, title={}, title style={at={{(0.5,1)}}, anchor={south}, font={{\fontsize{14 pt}{18.2 pt}\selectfont}}, color={rgb,1:red,0.0;green,0.0;blue,0.0}, draw opacity={1.0}, rotate={0.0}, align={center}}, legend style={color={rgb,1:red,0.0;green,0.0;blue,0.0}, draw opacity={1.0}, line width={1}, solid, fill={rgb,1:red,1.0;green,1.0;blue,1.0}, fill opacity={1.0}, text opacity={1.0}, font={{\fontsize{8 pt}{10.4 pt}\selectfont}}, text={rgb,1:red,0.0;green,0.0;blue,0.0}, cells={anchor={center}}, at={(0.02, 0.98)}, anchor={north west}}, axis background/.style={fill={rgb,1:red,1.0;green,1.0;blue,1.0}, opacity={1.0}}, anchor={north west}, xshift={1.0mm}, yshift={-1.0mm}, width={112.3mm}, height={55.15mm}, scaled x ticks={false}, xlabel={Event Time}, x tick style={color={rgb,1:red,0.0;green,0.0;blue,0.0}, opacity={1.0}}, x tick label style={color={rgb,1:red,0.0;green,0.0;blue,0.0}, opacity={1.0}, rotate={0}}, xlabel style={at={(ticklabel cs:0.5)}, anchor=near ticklabel, at={{(ticklabel cs:0.5)}}, anchor={near ticklabel}, font={{\fontsize{11 pt}{14.3 pt}\selectfont}}, color={rgb,1:red,0.0;green,0.0;blue,0.0}, draw opacity={1.0}, rotate={0.0}}, xmajorgrids={true}, xmin={-23.05}, xmax={14.05}, xticklabels={{$-20$,$-10$,$0$,$10$}}, xtick={{-20.0,-10.0,0.0,10.0}}, xtick align={inside}, xticklabel style={font={{\fontsize{8 pt}{10.4 pt}\selectfont}}, color={rgb,1:red,0.0;green,0.0;blue,0.0}, draw opacity={1.0}, rotate={0.0}}, x grid style={color={rgb,1:red,0.0;green,0.0;blue,0.0}, draw opacity={0.1}, line width={0.5}, solid}, axis x line*={left}, x axis line style={color={rgb,1:red,0.0;green,0.0;blue,0.0}, draw opacity={1.0}, line width={1}, solid}, scaled y ticks={false}, ylabel={}, y tick style={color={rgb,1:red,0.0;green,0.0;blue,0.0}, opacity={1.0}}, y tick label style={color={rgb,1:red,0.0;green,0.0;blue,0.0}, opacity={1.0}, rotate={0}}, ylabel style={at={(ticklabel cs:0.5)}, anchor=near ticklabel, at={{(ticklabel cs:0.5)}}, anchor={near ticklabel}, font={{\fontsize{11 pt}{14.3 pt}\selectfont}}, color={rgb,1:red,0.0;green,0.0;blue,0.0}, draw opacity={1.0}, rotate={0.0}}, ymajorgrids={true}, ymin={-0.06603488518899277}, ymax={0.27725574478770215}, yticklabels={{$-0.05$,$0.00$,$0.05$,$0.10$,$0.15$,$0.20$,$0.25$}}, ytick={{-0.05000000000000001,0.0,0.05000000000000001,0.10000000000000002,0.15000000000000002,0.20000000000000004,0.25000000000000006}}, ytick align={inside}, yticklabel style={font={{\fontsize{8 pt}{10.4 pt}\selectfont}}, color={rgb,1:red,0.0;green,0.0;blue,0.0}, draw opacity={1.0}, rotate={0.0}}, y grid style={color={rgb,1:red,0.0;green,0.0;blue,0.0}, draw opacity={0.1}, line width={0.5}, solid}, axis y line*={left}, y axis line style={color={rgb,1:red,0.0;green,0.0;blue,0.0}, draw opacity={1.0}, line width={1}, solid}, colorbar={false}]
    \addplot[color={rgb,1:red,0.6039;green,0.1412;blue,0.0824}, name path={ba7b1f06-2605-4540-a515-b05e09f187ed}, draw opacity={1.0}, line width={1.4}, solid]
        table[row sep={\\}]
        {
            \\
            -22.0  -0.05631911264248254  \\
            -21.0  -0.04794516074012633  \\
            -20.0  -0.006766870891158547  \\
            -19.0  -0.02510278370911196  \\
            -18.0  -0.013701736213038213  \\
            -17.0  -0.013760876694094966  \\
            -16.0  -0.018602144598848423  \\
            -15.0  -0.02175335187660577  \\
            -14.0  -0.01938872730555203  \\
            -13.0  -0.011650694487151076  \\
            -12.0  0.005355532690269773  \\
            -11.0  0.014466886089675876  \\
            -10.0  0.015015425160958074  \\
            -9.0  0.0174504504774845  \\
            -8.0  0.026030525515763996  \\
            -7.0  0.04292484171642989  \\
            -6.0  0.05928940561271417  \\
            -5.0  0.06564125242499204  \\
            -4.0  0.08287020897415288  \\
            -3.0  0.09433536485321842  \\
            -2.0  0.10225019864402499  \\
            -1.0  0.11448870382394986  \\
            0.0  0.1259197470140344  \\
            1.0  0.11818863821624183  \\
            2.0  0.1171837623786552  \\
            3.0  0.11381510214548006  \\
            4.0  0.1155072105946236  \\
            5.0  0.12071832042196164  \\
            6.0  0.12973320333931385  \\
            7.0  0.11902019539114696  \\
            8.0  0.1147774308263064  \\
            9.0  0.10410167926229548  \\
            10.0  0.13533621127172793  \\
            11.0  0.12519838935938662  \\
            12.0  0.0968718038834746  \\
            13.0  0.11381113985116162  \\
        }
        ;
    \addlegendentry {Average of $\tilde{y}_{it}$}
    \addplot[color={rgb,1:red,0.0627;green,0.4706;blue,0.5843}, name path={191c59ad-38a0-4b23-a725-c59f4c02d0d5}, draw opacity={1.0}, line width={1.4}, solid]
        table[row sep={\\}]
        {
            \\
            -22.0  -0.04710991013394964  \\
            -21.0  -0.0298568701588629  \\
            -20.0  -0.04313584432160746  \\
            -19.0  -0.03492767923887719  \\
            -18.0  -0.0375121704104852  \\
            -17.0  -0.029815944356425778  \\
            -16.0  -0.03574889575958288  \\
            -15.0  -0.021711099514947964  \\
            -14.0  -0.02091905185564675  \\
            -13.0  -0.00897814046130354  \\
            -12.0  0.00023517674060796162  \\
            -11.0  0.0064075310446727336  \\
            -10.0  0.014850227276637887  \\
            -9.0  0.02084699123564475  \\
            -8.0  0.03225598710847384  \\
            -7.0  0.04460018122993908  \\
            -6.0  0.05616804432293301  \\
            -5.0  0.06881702473257423  \\
            -4.0  0.07606009705137874  \\
            -3.0  0.08681597174029702  \\
            -2.0  0.09686159308177057  \\
            -1.0  0.1109338606059503  \\
            0.0  0.12245273551098369  \\
            1.0  0.13324853064406025  \\
            2.0  0.14398561235258217  \\
            3.0  0.15187708115113932  \\
            4.0  0.16024648320513019  \\
            5.0  0.17301622254496568  \\
            6.0  0.1799996719437058  \\
            7.0  0.1597595607377226  \\
            8.0  0.17113641639180333  \\
            9.0  0.1630461298078145  \\
            10.0  0.20095569650608391  \\
            11.0  0.2511251207609037  \\
            12.0  0.2675399722411919  \\
            13.0  0.24586504112090532  \\
        }
        ;
    \addlegendentry {Average of $\hat{\tilde{y}}_{it}(0)$}
    \addplot[color={rgb,1:red,0.0;green,0.0;blue,0.0}, name path={1d3c6a1d-cb12-43b0-807b-a4d912d868c7}, draw opacity={1.0}, line width={1}, dashed, forget plot]
        table[row sep={\\}]
        {
            \\
            -0.5  -0.40932551516568766  \\
            -0.5  0.6205463747643971  \\
        }
        ;
\end{axis}
\end{tikzpicture}

    \end{adjustbox}
\end{subfigure}
\end{figure}
\end{frame}

%%%%%%%%%%%%%%%%%%%%%%%%%%%%%%%%%%%%%%%%%%%%%%%%%%%%%%%%%%%%%%%%%%%%%%%

\begin{frame}{Results}
    \begin{figure}
\caption*{Generalized Imputation Estimator for Effect of Walmart on County Employment with Naive Standard Errors}
\label{fig:walmart_naive_se}

\begin{subfigure}[b]{0.45\textwidth}
    \caption{$\log$ Retail Employment}
    \begin{adjustbox}{width=\textwidth, center}
        % Recommended preamble:
% \usetikzlibrary{arrows.meta}
% \usetikzlibrary{backgrounds}
% \usepgfplotslibrary{patchplots}
% \usepgfplotslibrary{fillbetween}
% \pgfplotsset{%
%     layers/standard/.define layer set={%
%         background,axis background,axis grid,axis ticks,axis lines,axis tick labels,pre main,main,axis descriptions,axis foreground%
%     }{
%         grid style={/pgfplots/on layer=axis grid},%
%         tick style={/pgfplots/on layer=axis ticks},%
%         axis line style={/pgfplots/on layer=axis lines},%
%         label style={/pgfplots/on layer=axis descriptions},%
%         legend style={/pgfplots/on layer=axis descriptions},%
%         title style={/pgfplots/on layer=axis descriptions},%
%         colorbar style={/pgfplots/on layer=axis descriptions},%
%         ticklabel style={/pgfplots/on layer=axis tick labels},%
%         axis background@ style={/pgfplots/on layer=axis background},%
%         3d box foreground style={/pgfplots/on layer=axis foreground},%
%     },
% }

\begin{tikzpicture}[/tikz/background rectangle/.style={fill={rgb,1:red,1.0;green,1.0;blue,1.0}, draw opacity={1.0}}, show background rectangle]
\begin{axis}[point meta max={nan}, point meta min={nan}, legend cell align={left}, legend columns={1}, title={}, title style={at={{(0.5,1)}}, anchor={south}, font={{\fontsize{14 pt}{18.2 pt}\selectfont}}, color={rgb,1:red,0.0;green,0.0;blue,0.0}, draw opacity={1.0}, rotate={0.0}, align={center}}, legend style={color={rgb,1:red,0.0;green,0.0;blue,0.0}, draw opacity={1.0}, line width={1}, solid, fill={rgb,1:red,1.0;green,1.0;blue,1.0}, fill opacity={1.0}, text opacity={1.0}, font={{\fontsize{8 pt}{10.4 pt}\selectfont}}, text={rgb,1:red,0.0;green,0.0;blue,0.0}, cells={anchor={center}}, at={(1.02, 1)}, anchor={north west}}, axis background/.style={fill={rgb,1:red,1.0;green,1.0;blue,1.0}, opacity={1.0}}, anchor={north west}, xshift={1.0mm}, yshift={-1.0mm}, width={112.3mm}, height={74.2mm}, scaled x ticks={false}, xlabel={Event Time}, x tick style={color={rgb,1:red,0.0;green,0.0;blue,0.0}, opacity={1.0}}, x tick label style={color={rgb,1:red,0.0;green,0.0;blue,0.0}, opacity={1.0}, rotate={0}}, xlabel style={at={(ticklabel cs:0.5)}, anchor=near ticklabel, at={{(ticklabel cs:0.5)}}, anchor={near ticklabel}, font={{\fontsize{11 pt}{14.3 pt}\selectfont}}, color={rgb,1:red,0.0;green,0.0;blue,0.0}, draw opacity={1.0}, rotate={0.0}}, xmajorgrids={true}, xmin={-23.05}, xmax={14.05}, xticklabels={{$-20$,$-10$,$0$,$10$}}, xtick={{-20.0,-10.0,0.0,10.0}}, xtick align={inside}, xticklabel style={font={{\fontsize{8 pt}{10.4 pt}\selectfont}}, color={rgb,1:red,0.0;green,0.0;blue,0.0}, draw opacity={1.0}, rotate={0.0}}, x grid style={color={rgb,1:red,0.0;green,0.0;blue,0.0}, draw opacity={0.1}, line width={0.5}, solid}, axis x line*={left}, x axis line style={color={rgb,1:red,0.0;green,0.0;blue,0.0}, draw opacity={1.0}, line width={1}, solid}, scaled y ticks={false}, ylabel={Coefficient}, y tick style={color={rgb,1:red,0.0;green,0.0;blue,0.0}, opacity={1.0}}, y tick label style={color={rgb,1:red,0.0;green,0.0;blue,0.0}, opacity={1.0}, rotate={0}}, ylabel style={at={(ticklabel cs:0.5)}, anchor=near ticklabel, at={{(ticklabel cs:0.5)}}, anchor={near ticklabel}, font={{\fontsize{11 pt}{14.3 pt}\selectfont}}, color={rgb,1:red,0.0;green,0.0;blue,0.0}, draw opacity={1.0}, rotate={0.0}}, ymajorgrids={true}, ymin={-0.175}, ymax={0.3}, yticklabels={{$-0.1$,$0.0$,$0.1$,$0.2$}}, ytick={{-0.1,0.0,0.1,0.2}}, ytick align={inside}, yticklabel style={font={{\fontsize{8 pt}{10.4 pt}\selectfont}}, color={rgb,1:red,0.0;green,0.0;blue,0.0}, draw opacity={1.0}, rotate={0.0}}, y grid style={color={rgb,1:red,0.0;green,0.0;blue,0.0}, draw opacity={0.1}, line width={0.5}, solid}, axis y line*={left}, y axis line style={color={rgb,1:red,0.0;green,0.0;blue,0.0}, draw opacity={1.0}, line width={1}, solid}, colorbar={false}]
    \addplot[color={rgb,1:red,0.0;green,0.0;blue,0.0}, name path={37d563ef-1dc6-43b2-b6be-cb6dc05a3e6d}, draw opacity={1.0}, line width={1}, dashed]
        table[row sep={\\}]
        {
            \\
            -60.150000000000006  0.0  \\
            51.150000000000006  0.0  \\
        }
        ;
    \addplot[color={rgb,1:red,0.6039;green,0.1412;blue,0.0824}, name path={d1800307-d142-4970-8ffe-9cb9cfbdb75c}, draw opacity={1.0}, line width={1}, solid, mark={-}, mark size={1.5 pt}, mark repeat={1}, mark options={color={rgb,1:red,0.6039;green,0.1412;blue,0.0824}, draw opacity={1.0}, fill={rgb,1:red,0.6039;green,0.1412;blue,0.0824}, fill opacity={1.0}, line width={0.75}, rotate={0}, solid}]
        table[row sep={\\}]
        {
            \\
            -22.0  -0.07346809142974857  \\
            -22.0  0.02098452637029741  \\
        }
        ;
    \addplot[color={rgb,1:red,0.6039;green,0.1412;blue,0.0824}, name path={d1800307-d142-4970-8ffe-9cb9cfbdb75c}, draw opacity={1.0}, line width={1}, solid, mark={-}, mark size={1.5 pt}, mark repeat={1}, mark options={color={rgb,1:red,0.6039;green,0.1412;blue,0.0824}, draw opacity={1.0}, fill={rgb,1:red,0.6039;green,0.1412;blue,0.0824}, fill opacity={1.0}, line width={0.75}, rotate={0}, solid}]
        table[row sep={\\}]
        {
            \\
            -21.0  -0.058851531062297416  \\
            -21.0  0.006400991911376441  \\
        }
        ;
    \addplot[color={rgb,1:red,0.6039;green,0.1412;blue,0.0824}, name path={d1800307-d142-4970-8ffe-9cb9cfbdb75c}, draw opacity={1.0}, line width={1}, solid, mark={-}, mark size={1.5 pt}, mark repeat={1}, mark options={color={rgb,1:red,0.6039;green,0.1412;blue,0.0824}, draw opacity={1.0}, fill={rgb,1:red,0.6039;green,0.1412;blue,0.0824}, fill opacity={1.0}, line width={0.75}, rotate={0}, solid}]
        table[row sep={\\}]
        {
            \\
            -20.0  -0.0327665638834874  \\
            -20.0  0.01896726550101175  \\
        }
        ;
    \addplot[color={rgb,1:red,0.6039;green,0.1412;blue,0.0824}, name path={d1800307-d142-4970-8ffe-9cb9cfbdb75c}, draw opacity={1.0}, line width={1}, solid, mark={-}, mark size={1.5 pt}, mark repeat={1}, mark options={color={rgb,1:red,0.6039;green,0.1412;blue,0.0824}, draw opacity={1.0}, fill={rgb,1:red,0.6039;green,0.1412;blue,0.0824}, fill opacity={1.0}, line width={0.75}, rotate={0}, solid}]
        table[row sep={\\}]
        {
            \\
            -19.0  -0.02684895790531064  \\
            -19.0  0.01755910646366498  \\
        }
        ;
    \addplot[color={rgb,1:red,0.6039;green,0.1412;blue,0.0824}, name path={d1800307-d142-4970-8ffe-9cb9cfbdb75c}, draw opacity={1.0}, line width={1}, solid, mark={-}, mark size={1.5 pt}, mark repeat={1}, mark options={color={rgb,1:red,0.6039;green,0.1412;blue,0.0824}, draw opacity={1.0}, fill={rgb,1:red,0.6039;green,0.1412;blue,0.0824}, fill opacity={1.0}, line width={0.75}, rotate={0}, solid}]
        table[row sep={\\}]
        {
            \\
            -18.0  -0.017488725043361828  \\
            -18.0  0.018090197239029434  \\
        }
        ;
    \addplot[color={rgb,1:red,0.6039;green,0.1412;blue,0.0824}, name path={d1800307-d142-4970-8ffe-9cb9cfbdb75c}, draw opacity={1.0}, line width={1}, solid, mark={-}, mark size={1.5 pt}, mark repeat={1}, mark options={color={rgb,1:red,0.6039;green,0.1412;blue,0.0824}, draw opacity={1.0}, fill={rgb,1:red,0.6039;green,0.1412;blue,0.0824}, fill opacity={1.0}, line width={0.75}, rotate={0}, solid}]
        table[row sep={\\}]
        {
            \\
            -17.0  -0.01818915262030111  \\
            -17.0  0.012727723855087743  \\
        }
        ;
    \addplot[color={rgb,1:red,0.6039;green,0.1412;blue,0.0824}, name path={d1800307-d142-4970-8ffe-9cb9cfbdb75c}, draw opacity={1.0}, line width={1}, solid, mark={-}, mark size={1.5 pt}, mark repeat={1}, mark options={color={rgb,1:red,0.6039;green,0.1412;blue,0.0824}, draw opacity={1.0}, fill={rgb,1:red,0.6039;green,0.1412;blue,0.0824}, fill opacity={1.0}, line width={0.75}, rotate={0}, solid}]
        table[row sep={\\}]
        {
            \\
            -16.0  -0.004695437789023468  \\
            -16.0  0.020548086146096378  \\
        }
        ;
    \addplot[color={rgb,1:red,0.6039;green,0.1412;blue,0.0824}, name path={d1800307-d142-4970-8ffe-9cb9cfbdb75c}, draw opacity={1.0}, line width={1}, solid, mark={-}, mark size={1.5 pt}, mark repeat={1}, mark options={color={rgb,1:red,0.6039;green,0.1412;blue,0.0824}, draw opacity={1.0}, fill={rgb,1:red,0.6039;green,0.1412;blue,0.0824}, fill opacity={1.0}, line width={0.75}, rotate={0}, solid}]
        table[row sep={\\}]
        {
            \\
            -15.0  -0.005389932205425098  \\
            -15.0  0.0167558979597315  \\
        }
        ;
    \addplot[color={rgb,1:red,0.6039;green,0.1412;blue,0.0824}, name path={d1800307-d142-4970-8ffe-9cb9cfbdb75c}, draw opacity={1.0}, line width={1}, solid, mark={-}, mark size={1.5 pt}, mark repeat={1}, mark options={color={rgb,1:red,0.6039;green,0.1412;blue,0.0824}, draw opacity={1.0}, fill={rgb,1:red,0.6039;green,0.1412;blue,0.0824}, fill opacity={1.0}, line width={0.75}, rotate={0}, solid}]
        table[row sep={\\}]
        {
            \\
            -14.0  -0.002653135964758158  \\
            -14.0  0.016801407644908707  \\
        }
        ;
    \addplot[color={rgb,1:red,0.6039;green,0.1412;blue,0.0824}, name path={d1800307-d142-4970-8ffe-9cb9cfbdb75c}, draw opacity={1.0}, line width={1}, solid, mark={-}, mark size={1.5 pt}, mark repeat={1}, mark options={color={rgb,1:red,0.6039;green,0.1412;blue,0.0824}, draw opacity={1.0}, fill={rgb,1:red,0.6039;green,0.1412;blue,0.0824}, fill opacity={1.0}, line width={0.75}, rotate={0}, solid}]
        table[row sep={\\}]
        {
            \\
            -13.0  -0.011794344484261813  \\
            -13.0  0.005687747830578206  \\
        }
        ;
    \addplot[color={rgb,1:red,0.6039;green,0.1412;blue,0.0824}, name path={d1800307-d142-4970-8ffe-9cb9cfbdb75c}, draw opacity={1.0}, line width={1}, solid, mark={-}, mark size={1.5 pt}, mark repeat={1}, mark options={color={rgb,1:red,0.6039;green,0.1412;blue,0.0824}, draw opacity={1.0}, fill={rgb,1:red,0.6039;green,0.1412;blue,0.0824}, fill opacity={1.0}, line width={0.75}, rotate={0}, solid}]
        table[row sep={\\}]
        {
            \\
            -12.0  -0.013244713482336263  \\
            -12.0  0.0032330817301638266  \\
        }
        ;
    \addplot[color={rgb,1:red,0.6039;green,0.1412;blue,0.0824}, name path={d1800307-d142-4970-8ffe-9cb9cfbdb75c}, draw opacity={1.0}, line width={1}, solid, mark={-}, mark size={1.5 pt}, mark repeat={1}, mark options={color={rgb,1:red,0.6039;green,0.1412;blue,0.0824}, draw opacity={1.0}, fill={rgb,1:red,0.6039;green,0.1412;blue,0.0824}, fill opacity={1.0}, line width={0.75}, rotate={0}, solid}]
        table[row sep={\\}]
        {
            \\
            -11.0  -0.009578817245344417  \\
            -11.0  0.006226128697615454  \\
        }
        ;
    \addplot[color={rgb,1:red,0.6039;green,0.1412;blue,0.0824}, name path={d1800307-d142-4970-8ffe-9cb9cfbdb75c}, draw opacity={1.0}, line width={1}, solid, mark={-}, mark size={1.5 pt}, mark repeat={1}, mark options={color={rgb,1:red,0.6039;green,0.1412;blue,0.0824}, draw opacity={1.0}, fill={rgb,1:red,0.6039;green,0.1412;blue,0.0824}, fill opacity={1.0}, line width={0.75}, rotate={0}, solid}]
        table[row sep={\\}]
        {
            \\
            -10.0  -0.009224979999996318  \\
            -10.0  0.005853585977429181  \\
        }
        ;
    \addplot[color={rgb,1:red,0.6039;green,0.1412;blue,0.0824}, name path={d1800307-d142-4970-8ffe-9cb9cfbdb75c}, draw opacity={1.0}, line width={1}, solid, mark={-}, mark size={1.5 pt}, mark repeat={1}, mark options={color={rgb,1:red,0.6039;green,0.1412;blue,0.0824}, draw opacity={1.0}, fill={rgb,1:red,0.6039;green,0.1412;blue,0.0824}, fill opacity={1.0}, line width={0.75}, rotate={0}, solid}]
        table[row sep={\\}]
        {
            \\
            -9.0  -0.00943872641526165  \\
            -9.0  0.005045587113661897  \\
        }
        ;
    \addplot[color={rgb,1:red,0.6039;green,0.1412;blue,0.0824}, name path={d1800307-d142-4970-8ffe-9cb9cfbdb75c}, draw opacity={1.0}, line width={1}, solid, mark={-}, mark size={1.5 pt}, mark repeat={1}, mark options={color={rgb,1:red,0.6039;green,0.1412;blue,0.0824}, draw opacity={1.0}, fill={rgb,1:red,0.6039;green,0.1412;blue,0.0824}, fill opacity={1.0}, line width={0.75}, rotate={0}, solid}]
        table[row sep={\\}]
        {
            \\
            -8.0  -0.008270911031222905  \\
            -8.0  0.00621340249770064  \\
        }
        ;
    \addplot[color={rgb,1:red,0.6039;green,0.1412;blue,0.0824}, name path={d1800307-d142-4970-8ffe-9cb9cfbdb75c}, draw opacity={1.0}, line width={1}, solid, mark={-}, mark size={1.5 pt}, mark repeat={1}, mark options={color={rgb,1:red,0.6039;green,0.1412;blue,0.0824}, draw opacity={1.0}, fill={rgb,1:red,0.6039;green,0.1412;blue,0.0824}, fill opacity={1.0}, line width={0.75}, rotate={0}, solid}]
        table[row sep={\\}]
        {
            \\
            -7.0  -0.00342270681572716  \\
            -7.0  0.011061606713196385  \\
        }
        ;
    \addplot[color={rgb,1:red,0.6039;green,0.1412;blue,0.0824}, name path={d1800307-d142-4970-8ffe-9cb9cfbdb75c}, draw opacity={1.0}, line width={1}, solid, mark={-}, mark size={1.5 pt}, mark repeat={1}, mark options={color={rgb,1:red,0.6039;green,0.1412;blue,0.0824}, draw opacity={1.0}, fill={rgb,1:red,0.6039;green,0.1412;blue,0.0824}, fill opacity={1.0}, line width={0.75}, rotate={0}, solid}]
        table[row sep={\\}]
        {
            \\
            -6.0  -0.006335859157070088  \\
            -6.0  0.008148454371853457  \\
        }
        ;
    \addplot[color={rgb,1:red,0.6039;green,0.1412;blue,0.0824}, name path={d1800307-d142-4970-8ffe-9cb9cfbdb75c}, draw opacity={1.0}, line width={1}, solid, mark={-}, mark size={1.5 pt}, mark repeat={1}, mark options={color={rgb,1:red,0.6039;green,0.1412;blue,0.0824}, draw opacity={1.0}, fill={rgb,1:red,0.6039;green,0.1412;blue,0.0824}, fill opacity={1.0}, line width={0.75}, rotate={0}, solid}]
        table[row sep={\\}]
        {
            \\
            -5.0  -0.004374780759230534  \\
            -5.0  0.010109532769693012  \\
        }
        ;
    \addplot[color={rgb,1:red,0.6039;green,0.1412;blue,0.0824}, name path={d1800307-d142-4970-8ffe-9cb9cfbdb75c}, draw opacity={1.0}, line width={1}, solid, mark={-}, mark size={1.5 pt}, mark repeat={1}, mark options={color={rgb,1:red,0.6039;green,0.1412;blue,0.0824}, draw opacity={1.0}, fill={rgb,1:red,0.6039;green,0.1412;blue,0.0824}, fill opacity={1.0}, line width={0.75}, rotate={0}, solid}]
        table[row sep={\\}]
        {
            \\
            -4.0  -0.002433552886378856  \\
            -4.0  0.01205076064254469  \\
        }
        ;
    \addplot[color={rgb,1:red,0.6039;green,0.1412;blue,0.0824}, name path={d1800307-d142-4970-8ffe-9cb9cfbdb75c}, draw opacity={1.0}, line width={1}, solid, mark={-}, mark size={1.5 pt}, mark repeat={1}, mark options={color={rgb,1:red,0.6039;green,0.1412;blue,0.0824}, draw opacity={1.0}, fill={rgb,1:red,0.6039;green,0.1412;blue,0.0824}, fill opacity={1.0}, line width={0.75}, rotate={0}, solid}]
        table[row sep={\\}]
        {
            \\
            -3.0  -0.005317360418474976  \\
            -3.0  0.009166953110448569  \\
        }
        ;
    \addplot[color={rgb,1:red,0.6039;green,0.1412;blue,0.0824}, name path={d1800307-d142-4970-8ffe-9cb9cfbdb75c}, draw opacity={1.0}, line width={1}, solid, mark={-}, mark size={1.5 pt}, mark repeat={1}, mark options={color={rgb,1:red,0.6039;green,0.1412;blue,0.0824}, draw opacity={1.0}, fill={rgb,1:red,0.6039;green,0.1412;blue,0.0824}, fill opacity={1.0}, line width={0.75}, rotate={0}, solid}]
        table[row sep={\\}]
        {
            \\
            -2.0  -0.007881874812875571  \\
            -2.0  0.006602438716047975  \\
        }
        ;
    \addplot[color={rgb,1:red,0.6039;green,0.1412;blue,0.0824}, name path={d1800307-d142-4970-8ffe-9cb9cfbdb75c}, draw opacity={1.0}, line width={1}, solid, mark={-}, mark size={1.5 pt}, mark repeat={1}, mark options={color={rgb,1:red,0.6039;green,0.1412;blue,0.0824}, draw opacity={1.0}, fill={rgb,1:red,0.6039;green,0.1412;blue,0.0824}, fill opacity={1.0}, line width={0.75}, rotate={0}, solid}]
        table[row sep={\\}]
        {
            \\
            -1.0  -0.010759234491430602  \\
            -1.0  0.0037250790374929433  \\
        }
        ;
    \addplot[color={rgb,1:red,0.8889;green,0.4356;blue,0.2781}, name path={a7d4b7cc-bc61-4ed5-a9b2-c54d64144c11}, only marks, draw opacity={1.0}, line width={0}, solid, mark={*}, mark size={1.5 pt}, mark repeat={1}, mark options={color={rgb,1:red,0.6039;green,0.1412;blue,0.0824}, draw opacity={1.0}, fill={rgb,1:red,0.6039;green,0.1412;blue,0.0824}, fill opacity={1.0}, line width={0.75}, rotate={0}, solid}]
        table[row sep={\\}]
        {
            \\
            -22.0  -0.026241782529725578  \\
            -21.0  -0.026225269575460487  \\
            -20.0  -0.006899649191237824  \\
            -19.0  -0.004644925720822831  \\
            -18.0  0.00030073609783380433  \\
            -17.0  -0.002730714382606683  \\
            -16.0  0.007926324178536455  \\
            -15.0  0.005682982877153201  \\
            -14.0  0.0070741358400752755  \\
            -13.0  -0.003053298326841804  \\
            -12.0  -0.005005815876086219  \\
            -11.0  -0.0016763442738644816  \\
            -10.0  -0.0016856970112835684  \\
            -9.0  -0.0021965696507998756  \\
            -8.0  -0.0010287542667611327  \\
            -7.0  0.003819449948734613  \\
            -6.0  0.000906297607391685  \\
            -5.0  0.002867376005231239  \\
            -4.0  0.004808603878082917  \\
            -3.0  0.0019247963459867968  \\
            -2.0  -0.000639718048413798  \\
            -1.0  -0.0035170777269688297  \\
        }
        ;
    \addplot[color={rgb,1:red,0.0627;green,0.4706;blue,0.5843}, name path={c93d90b3-c20f-4ae8-974b-2a70e06e72a7}, draw opacity={1.0}, line width={1}, solid, mark={-}, mark size={1.5 pt}, mark repeat={1}, mark options={color={rgb,1:red,0.0627;green,0.4706;blue,0.5843}, draw opacity={1.0}, fill={rgb,1:red,0.0627;green,0.4706;blue,0.5843}, fill opacity={1.0}, line width={0.75}, rotate={0}, solid}]
        table[row sep={\\}]
        {
            \\
            0.0  0.006987378713450478  \\
            0.0  0.021471692242374024  \\
        }
        ;
    \addplot[color={rgb,1:red,0.0627;green,0.4706;blue,0.5843}, name path={c93d90b3-c20f-4ae8-974b-2a70e06e72a7}, draw opacity={1.0}, line width={1}, solid, mark={-}, mark size={1.5 pt}, mark repeat={1}, mark options={color={rgb,1:red,0.0627;green,0.4706;blue,0.5843}, draw opacity={1.0}, fill={rgb,1:red,0.0627;green,0.4706;blue,0.5843}, fill opacity={1.0}, line width={0.75}, rotate={0}, solid}]
        table[row sep={\\}]
        {
            \\
            1.0  0.035623699400057565  \\
            1.0  0.05028138506583492  \\
        }
        ;
    \addplot[color={rgb,1:red,0.0627;green,0.4706;blue,0.5843}, name path={c93d90b3-c20f-4ae8-974b-2a70e06e72a7}, draw opacity={1.0}, line width={1}, solid, mark={-}, mark size={1.5 pt}, mark repeat={1}, mark options={color={rgb,1:red,0.0627;green,0.4706;blue,0.5843}, draw opacity={1.0}, fill={rgb,1:red,0.0627;green,0.4706;blue,0.5843}, fill opacity={1.0}, line width={0.75}, rotate={0}, solid}]
        table[row sep={\\}]
        {
            \\
            2.0  0.033700339312812404  \\
            2.0  0.04855524166538581  \\
        }
        ;
    \addplot[color={rgb,1:red,0.0627;green,0.4706;blue,0.5843}, name path={c93d90b3-c20f-4ae8-974b-2a70e06e72a7}, draw opacity={1.0}, line width={1}, solid, mark={-}, mark size={1.5 pt}, mark repeat={1}, mark options={color={rgb,1:red,0.0627;green,0.4706;blue,0.5843}, draw opacity={1.0}, fill={rgb,1:red,0.0627;green,0.4706;blue,0.5843}, fill opacity={1.0}, line width={0.75}, rotate={0}, solid}]
        table[row sep={\\}]
        {
            \\
            3.0  0.030449826764414885  \\
            3.0  0.04553755069347568  \\
        }
        ;
    \addplot[color={rgb,1:red,0.0627;green,0.4706;blue,0.5843}, name path={c93d90b3-c20f-4ae8-974b-2a70e06e72a7}, draw opacity={1.0}, line width={1}, solid, mark={-}, mark size={1.5 pt}, mark repeat={1}, mark options={color={rgb,1:red,0.0627;green,0.4706;blue,0.5843}, draw opacity={1.0}, fill={rgb,1:red,0.0627;green,0.4706;blue,0.5843}, fill opacity={1.0}, line width={0.75}, rotate={0}, solid}]
        table[row sep={\\}]
        {
            \\
            4.0  0.03436864443764722  \\
            4.0  0.049690882326691437  \\
        }
        ;
    \addplot[color={rgb,1:red,0.0627;green,0.4706;blue,0.5843}, name path={c93d90b3-c20f-4ae8-974b-2a70e06e72a7}, draw opacity={1.0}, line width={1}, solid, mark={-}, mark size={1.5 pt}, mark repeat={1}, mark options={color={rgb,1:red,0.0627;green,0.4706;blue,0.5843}, draw opacity={1.0}, fill={rgb,1:red,0.0627;green,0.4706;blue,0.5843}, fill opacity={1.0}, line width={0.75}, rotate={0}, solid}]
        table[row sep={\\}]
        {
            \\
            5.0  0.04116305906511999  \\
            5.0  0.057020953048997854  \\
        }
        ;
    \addplot[color={rgb,1:red,0.0627;green,0.4706;blue,0.5843}, name path={c93d90b3-c20f-4ae8-974b-2a70e06e72a7}, draw opacity={1.0}, line width={1}, solid, mark={-}, mark size={1.5 pt}, mark repeat={1}, mark options={color={rgb,1:red,0.0627;green,0.4706;blue,0.5843}, draw opacity={1.0}, fill={rgb,1:red,0.0627;green,0.4706;blue,0.5843}, fill opacity={1.0}, line width={0.75}, rotate={0}, solid}]
        table[row sep={\\}]
        {
            \\
            6.0  0.05179778929211484  \\
            6.0  0.06819263211401541  \\
        }
        ;
    \addplot[color={rgb,1:red,0.0627;green,0.4706;blue,0.5843}, name path={c93d90b3-c20f-4ae8-974b-2a70e06e72a7}, draw opacity={1.0}, line width={1}, solid, mark={-}, mark size={1.5 pt}, mark repeat={1}, mark options={color={rgb,1:red,0.0627;green,0.4706;blue,0.5843}, draw opacity={1.0}, fill={rgb,1:red,0.0627;green,0.4706;blue,0.5843}, fill opacity={1.0}, line width={0.75}, rotate={0}, solid}]
        table[row sep={\\}]
        {
            \\
            7.0  0.06567745634334371  \\
            7.0  0.0833626669192282  \\
        }
        ;
    \addplot[color={rgb,1:red,0.0627;green,0.4706;blue,0.5843}, name path={c93d90b3-c20f-4ae8-974b-2a70e06e72a7}, draw opacity={1.0}, line width={1}, solid, mark={-}, mark size={1.5 pt}, mark repeat={1}, mark options={color={rgb,1:red,0.0627;green,0.4706;blue,0.5843}, draw opacity={1.0}, fill={rgb,1:red,0.0627;green,0.4706;blue,0.5843}, fill opacity={1.0}, line width={0.75}, rotate={0}, solid}]
        table[row sep={\\}]
        {
            \\
            8.0  0.07632196970194266  \\
            8.0  0.09546951900249417  \\
        }
        ;
    \addplot[color={rgb,1:red,0.0627;green,0.4706;blue,0.5843}, name path={c93d90b3-c20f-4ae8-974b-2a70e06e72a7}, draw opacity={1.0}, line width={1}, solid, mark={-}, mark size={1.5 pt}, mark repeat={1}, mark options={color={rgb,1:red,0.0627;green,0.4706;blue,0.5843}, draw opacity={1.0}, fill={rgb,1:red,0.0627;green,0.4706;blue,0.5843}, fill opacity={1.0}, line width={0.75}, rotate={0}, solid}]
        table[row sep={\\}]
        {
            \\
            9.0  0.07828152799363353  \\
            9.0  0.09997764980133814  \\
        }
        ;
    \addplot[color={rgb,1:red,0.0627;green,0.4706;blue,0.5843}, name path={c93d90b3-c20f-4ae8-974b-2a70e06e72a7}, draw opacity={1.0}, line width={1}, solid, mark={-}, mark size={1.5 pt}, mark repeat={1}, mark options={color={rgb,1:red,0.0627;green,0.4706;blue,0.5843}, draw opacity={1.0}, fill={rgb,1:red,0.0627;green,0.4706;blue,0.5843}, fill opacity={1.0}, line width={0.75}, rotate={0}, solid}]
        table[row sep={\\}]
        {
            \\
            10.0  0.06417213297614667  \\
            10.0  0.09003904766839625  \\
        }
        ;
    \addplot[color={rgb,1:red,0.0627;green,0.4706;blue,0.5843}, name path={c93d90b3-c20f-4ae8-974b-2a70e06e72a7}, draw opacity={1.0}, line width={1}, solid, mark={-}, mark size={1.5 pt}, mark repeat={1}, mark options={color={rgb,1:red,0.0627;green,0.4706;blue,0.5843}, draw opacity={1.0}, fill={rgb,1:red,0.0627;green,0.4706;blue,0.5843}, fill opacity={1.0}, line width={0.75}, rotate={0}, solid}]
        table[row sep={\\}]
        {
            \\
            11.0  0.049790362357256875  \\
            11.0  0.08016951300475869  \\
        }
        ;
    \addplot[color={rgb,1:red,0.0627;green,0.4706;blue,0.5843}, name path={c93d90b3-c20f-4ae8-974b-2a70e06e72a7}, draw opacity={1.0}, line width={1}, solid, mark={-}, mark size={1.5 pt}, mark repeat={1}, mark options={color={rgb,1:red,0.0627;green,0.4706;blue,0.5843}, draw opacity={1.0}, fill={rgb,1:red,0.0627;green,0.4706;blue,0.5843}, fill opacity={1.0}, line width={0.75}, rotate={0}, solid}]
        table[row sep={\\}]
        {
            \\
            12.0  0.04131807703652064  \\
            12.0  0.07751366433420881  \\
        }
        ;
    \addplot[color={rgb,1:red,0.0627;green,0.4706;blue,0.5843}, name path={c93d90b3-c20f-4ae8-974b-2a70e06e72a7}, draw opacity={1.0}, line width={1}, solid, mark={-}, mark size={1.5 pt}, mark repeat={1}, mark options={color={rgb,1:red,0.0627;green,0.4706;blue,0.5843}, draw opacity={1.0}, fill={rgb,1:red,0.0627;green,0.4706;blue,0.5843}, fill opacity={1.0}, line width={0.75}, rotate={0}, solid}]
        table[row sep={\\}]
        {
            \\
            13.0  0.08211277545671364  \\
            13.0  0.13422013914603215  \\
        }
        ;
    \addplot[color={rgb,1:red,0.2422;green,0.6433;blue,0.3044}, name path={3614f829-2a1a-49ca-be34-ef82da986ca6}, only marks, draw opacity={1.0}, line width={0}, solid, mark={*}, mark size={1.5 pt}, mark repeat={1}, mark options={color={rgb,1:red,0.0627;green,0.4706;blue,0.5843}, draw opacity={1.0}, fill={rgb,1:red,0.0627;green,0.4706;blue,0.5843}, fill opacity={1.0}, line width={0.75}, rotate={0}, solid}]
        table[row sep={\\}]
        {
            \\
            0.0  0.014229535477912251  \\
            1.0  0.04295254223294624  \\
            2.0  0.041127790489099106  \\
            3.0  0.037993688728945284  \\
            4.0  0.04202976338216933  \\
            5.0  0.04909200605705892  \\
            6.0  0.05999521070306513  \\
            7.0  0.07452006163128595  \\
            8.0  0.08589574435221842  \\
            9.0  0.08912958889748583  \\
            10.0  0.07710559032227146  \\
            11.0  0.06497993768100778  \\
            12.0  0.059415870685364726  \\
            13.0  0.10816645730137289  \\
        }
        ;
\end{axis}
\end{tikzpicture}

    \end{adjustbox}
\end{subfigure}
\hfill
\begin{subfigure}[b]{0.45\textwidth}
    \caption{$\log$ Wholesale Retail Employment}
    \begin{adjustbox}{width=\textwidth, center}
        % Recommended preamble:
% \usetikzlibrary{arrows.meta}
% \usetikzlibrary{backgrounds}
% \usepgfplotslibrary{patchplots}
% \usepgfplotslibrary{fillbetween}
% \pgfplotsset{%
%     layers/standard/.define layer set={%
%         background,axis background,axis grid,axis ticks,axis lines,axis tick labels,pre main,main,axis descriptions,axis foreground%
%     }{
%         grid style={/pgfplots/on layer=axis grid},%
%         tick style={/pgfplots/on layer=axis ticks},%
%         axis line style={/pgfplots/on layer=axis lines},%
%         label style={/pgfplots/on layer=axis descriptions},%
%         legend style={/pgfplots/on layer=axis descriptions},%
%         title style={/pgfplots/on layer=axis descriptions},%
%         colorbar style={/pgfplots/on layer=axis descriptions},%
%         ticklabel style={/pgfplots/on layer=axis tick labels},%
%         axis background@ style={/pgfplots/on layer=axis background},%
%         3d box foreground style={/pgfplots/on layer=axis foreground},%
%     },
% }

\begin{tikzpicture}[/tikz/background rectangle/.style={fill={rgb,1:red,1.0;green,1.0;blue,1.0}, draw opacity={1.0}}, show background rectangle]
\begin{axis}[point meta max={nan}, point meta min={nan}, legend cell align={left}, legend columns={1}, title={}, title style={at={{(0.5,1)}}, anchor={south}, font={{\fontsize{14 pt}{18.2 pt}\selectfont}}, color={rgb,1:red,0.0;green,0.0;blue,0.0}, draw opacity={1.0}, rotate={0.0}, align={center}}, legend style={color={rgb,1:red,0.0;green,0.0;blue,0.0}, draw opacity={1.0}, line width={1}, solid, fill={rgb,1:red,1.0;green,1.0;blue,1.0}, fill opacity={1.0}, text opacity={1.0}, font={{\fontsize{8 pt}{10.4 pt}\selectfont}}, text={rgb,1:red,0.0;green,0.0;blue,0.0}, cells={anchor={center}}, at={(1.02, 1)}, anchor={north west}}, axis background/.style={fill={rgb,1:red,1.0;green,1.0;blue,1.0}, opacity={1.0}}, anchor={north west}, xshift={1.0mm}, yshift={-1.0mm}, width={112.3mm}, height={74.2mm}, scaled x ticks={false}, xlabel={Event Time}, x tick style={color={rgb,1:red,0.0;green,0.0;blue,0.0}, opacity={1.0}}, x tick label style={color={rgb,1:red,0.0;green,0.0;blue,0.0}, opacity={1.0}, rotate={0}}, xlabel style={at={(ticklabel cs:0.5)}, anchor=near ticklabel, at={{(ticklabel cs:0.5)}}, anchor={near ticklabel}, font={{\fontsize{11 pt}{14.3 pt}\selectfont}}, color={rgb,1:red,0.0;green,0.0;blue,0.0}, draw opacity={1.0}, rotate={0.0}}, xmajorgrids={true}, xmin={-23.05}, xmax={14.05}, xticklabels={{$-20$,$-10$,$0$,$10$}}, xtick={{-20.0,-10.0,0.0,10.0}}, xtick align={inside}, xticklabel style={font={{\fontsize{8 pt}{10.4 pt}\selectfont}}, color={rgb,1:red,0.0;green,0.0;blue,0.0}, draw opacity={1.0}, rotate={0.0}}, x grid style={color={rgb,1:red,0.0;green,0.0;blue,0.0}, draw opacity={0.1}, line width={0.5}, solid}, axis x line*={left}, x axis line style={color={rgb,1:red,0.0;green,0.0;blue,0.0}, draw opacity={1.0}, line width={1}, solid}, scaled y ticks={false}, ylabel={Coefficient}, y tick style={color={rgb,1:red,0.0;green,0.0;blue,0.0}, opacity={1.0}}, y tick label style={color={rgb,1:red,0.0;green,0.0;blue,0.0}, opacity={1.0}, rotate={0}}, ylabel style={at={(ticklabel cs:0.5)}, anchor=near ticklabel, at={{(ticklabel cs:0.5)}}, anchor={near ticklabel}, font={{\fontsize{11 pt}{14.3 pt}\selectfont}}, color={rgb,1:red,0.0;green,0.0;blue,0.0}, draw opacity={1.0}, rotate={0.0}}, ymajorgrids={true}, ymin={-0.4}, ymax={0.2}, yticklabels={{$-0.4$,$-0.3$,$-0.2$,$-0.1$,$0.0$,$0.1$,$0.2$}}, ytick={{-0.4,-0.30000000000000004,-0.2,-0.1,0.0,0.1,0.2}}, ytick align={inside}, yticklabel style={font={{\fontsize{8 pt}{10.4 pt}\selectfont}}, color={rgb,1:red,0.0;green,0.0;blue,0.0}, draw opacity={1.0}, rotate={0.0}}, y grid style={color={rgb,1:red,0.0;green,0.0;blue,0.0}, draw opacity={0.1}, line width={0.5}, solid}, axis y line*={left}, y axis line style={color={rgb,1:red,0.0;green,0.0;blue,0.0}, draw opacity={1.0}, line width={1}, solid}, colorbar={false}]
    \addplot[color={rgb,1:red,0.0;green,0.0;blue,0.0}, name path={b276d4c6-2f1b-48ea-82ee-af8bfe92e83f}, draw opacity={1.0}, line width={1}, dashed]
        table[row sep={\\}]
        {
            \\
            -60.150000000000006  0.0  \\
            51.150000000000006  0.0  \\
        }
        ;
    \addplot[color={rgb,1:red,0.6039;green,0.1412;blue,0.0824}, name path={15299133-55cf-4bd4-90e3-63e1debeb400}, draw opacity={1.0}, line width={1}, solid, mark={-}, mark size={1.5 pt}, mark repeat={1}, mark options={color={rgb,1:red,0.6039;green,0.1412;blue,0.0824}, draw opacity={1.0}, fill={rgb,1:red,0.6039;green,0.1412;blue,0.0824}, fill opacity={1.0}, line width={0.75}, rotate={0}, solid}]
        table[row sep={\\}]
        {
            \\
            -22.0  -0.11811075309958707  \\
            -22.0  0.09969234808252125  \\
        }
        ;
    \addplot[color={rgb,1:red,0.6039;green,0.1412;blue,0.0824}, name path={15299133-55cf-4bd4-90e3-63e1debeb400}, draw opacity={1.0}, line width={1}, solid, mark={-}, mark size={1.5 pt}, mark repeat={1}, mark options={color={rgb,1:red,0.6039;green,0.1412;blue,0.0824}, draw opacity={1.0}, fill={rgb,1:red,0.6039;green,0.1412;blue,0.0824}, fill opacity={1.0}, line width={0.75}, rotate={0}, solid}]
        table[row sep={\\}]
        {
            \\
            -21.0  -0.0933228483661511  \\
            -21.0  0.05714626720362426  \\
        }
        ;
    \addplot[color={rgb,1:red,0.6039;green,0.1412;blue,0.0824}, name path={15299133-55cf-4bd4-90e3-63e1debeb400}, draw opacity={1.0}, line width={1}, solid, mark={-}, mark size={1.5 pt}, mark repeat={1}, mark options={color={rgb,1:red,0.6039;green,0.1412;blue,0.0824}, draw opacity={1.0}, fill={rgb,1:red,0.6039;green,0.1412;blue,0.0824}, fill opacity={1.0}, line width={0.75}, rotate={0}, solid}]
        table[row sep={\\}]
        {
            \\
            -20.0  -0.023278862375561517  \\
            -20.0  0.0960168092364593  \\
        }
        ;
    \addplot[color={rgb,1:red,0.6039;green,0.1412;blue,0.0824}, name path={15299133-55cf-4bd4-90e3-63e1debeb400}, draw opacity={1.0}, line width={1}, solid, mark={-}, mark size={1.5 pt}, mark repeat={1}, mark options={color={rgb,1:red,0.6039;green,0.1412;blue,0.0824}, draw opacity={1.0}, fill={rgb,1:red,0.6039;green,0.1412;blue,0.0824}, fill opacity={1.0}, line width={0.75}, rotate={0}, solid}]
        table[row sep={\\}]
        {
            \\
            -19.0  -0.041376512977161654  \\
            -19.0  0.06102630403669212  \\
        }
        ;
    \addplot[color={rgb,1:red,0.6039;green,0.1412;blue,0.0824}, name path={15299133-55cf-4bd4-90e3-63e1debeb400}, draw opacity={1.0}, line width={1}, solid, mark={-}, mark size={1.5 pt}, mark repeat={1}, mark options={color={rgb,1:red,0.6039;green,0.1412;blue,0.0824}, draw opacity={1.0}, fill={rgb,1:red,0.6039;green,0.1412;blue,0.0824}, fill opacity={1.0}, line width={0.75}, rotate={0}, solid}]
        table[row sep={\\}]
        {
            \\
            -18.0  -0.017211190138190332  \\
            -18.0  0.06483205853308431  \\
        }
        ;
    \addplot[color={rgb,1:red,0.6039;green,0.1412;blue,0.0824}, name path={15299133-55cf-4bd4-90e3-63e1debeb400}, draw opacity={1.0}, line width={1}, solid, mark={-}, mark size={1.5 pt}, mark repeat={1}, mark options={color={rgb,1:red,0.6039;green,0.1412;blue,0.0824}, draw opacity={1.0}, fill={rgb,1:red,0.6039;green,0.1412;blue,0.0824}, fill opacity={1.0}, line width={0.75}, rotate={0}, solid}]
        table[row sep={\\}]
        {
            \\
            -17.0  -0.019591332257781233  \\
            -17.0  0.051701467582442845  \\
        }
        ;
    \addplot[color={rgb,1:red,0.6039;green,0.1412;blue,0.0824}, name path={15299133-55cf-4bd4-90e3-63e1debeb400}, draw opacity={1.0}, line width={1}, solid, mark={-}, mark size={1.5 pt}, mark repeat={1}, mark options={color={rgb,1:red,0.6039;green,0.1412;blue,0.0824}, draw opacity={1.0}, fill={rgb,1:red,0.6039;green,0.1412;blue,0.0824}, fill opacity={1.0}, line width={0.75}, rotate={0}, solid}]
        table[row sep={\\}]
        {
            \\
            -16.0  -0.011958412496419418  \\
            -16.0  0.046251914817888276  \\
        }
        ;
    \addplot[color={rgb,1:red,0.6039;green,0.1412;blue,0.0824}, name path={15299133-55cf-4bd4-90e3-63e1debeb400}, draw opacity={1.0}, line width={1}, solid, mark={-}, mark size={1.5 pt}, mark repeat={1}, mark options={color={rgb,1:red,0.6039;green,0.1412;blue,0.0824}, draw opacity={1.0}, fill={rgb,1:red,0.6039;green,0.1412;blue,0.0824}, fill opacity={1.0}, line width={0.75}, rotate={0}, solid}]
        table[row sep={\\}]
        {
            \\
            -15.0  -0.025575851115026835  \\
            -15.0  0.025491346391711214  \\
        }
        ;
    \addplot[color={rgb,1:red,0.6039;green,0.1412;blue,0.0824}, name path={15299133-55cf-4bd4-90e3-63e1debeb400}, draw opacity={1.0}, line width={1}, solid, mark={-}, mark size={1.5 pt}, mark repeat={1}, mark options={color={rgb,1:red,0.6039;green,0.1412;blue,0.0824}, draw opacity={1.0}, fill={rgb,1:red,0.6039;green,0.1412;blue,0.0824}, fill opacity={1.0}, line width={0.75}, rotate={0}, solid}]
        table[row sep={\\}]
        {
            \\
            -14.0  -0.020900286845124596  \\
            -14.0  0.023960935945314028  \\
        }
        ;
    \addplot[color={rgb,1:red,0.6039;green,0.1412;blue,0.0824}, name path={15299133-55cf-4bd4-90e3-63e1debeb400}, draw opacity={1.0}, line width={1}, solid, mark={-}, mark size={1.5 pt}, mark repeat={1}, mark options={color={rgb,1:red,0.6039;green,0.1412;blue,0.0824}, draw opacity={1.0}, fill={rgb,1:red,0.6039;green,0.1412;blue,0.0824}, fill opacity={1.0}, line width={0.75}, rotate={0}, solid}]
        table[row sep={\\}]
        {
            \\
            -13.0  -0.02282897747927034  \\
            -13.0  0.01748386942757533  \\
        }
        ;
    \addplot[color={rgb,1:red,0.6039;green,0.1412;blue,0.0824}, name path={15299133-55cf-4bd4-90e3-63e1debeb400}, draw opacity={1.0}, line width={1}, solid, mark={-}, mark size={1.5 pt}, mark repeat={1}, mark options={color={rgb,1:red,0.6039;green,0.1412;blue,0.0824}, draw opacity={1.0}, fill={rgb,1:red,0.6039;green,0.1412;blue,0.0824}, fill opacity={1.0}, line width={0.75}, rotate={0}, solid}]
        table[row sep={\\}]
        {
            \\
            -12.0  -0.013878137588780225  \\
            -12.0  0.024118849488103854  \\
        }
        ;
    \addplot[color={rgb,1:red,0.6039;green,0.1412;blue,0.0824}, name path={15299133-55cf-4bd4-90e3-63e1debeb400}, draw opacity={1.0}, line width={1}, solid, mark={-}, mark size={1.5 pt}, mark repeat={1}, mark options={color={rgb,1:red,0.6039;green,0.1412;blue,0.0824}, draw opacity={1.0}, fill={rgb,1:red,0.6039;green,0.1412;blue,0.0824}, fill opacity={1.0}, line width={0.75}, rotate={0}, solid}]
        table[row sep={\\}]
        {
            \\
            -11.0  -0.010163359796419516  \\
            -11.0  0.02628206988642582  \\
        }
        ;
    \addplot[color={rgb,1:red,0.6039;green,0.1412;blue,0.0824}, name path={15299133-55cf-4bd4-90e3-63e1debeb400}, draw opacity={1.0}, line width={1}, solid, mark={-}, mark size={1.5 pt}, mark repeat={1}, mark options={color={rgb,1:red,0.6039;green,0.1412;blue,0.0824}, draw opacity={1.0}, fill={rgb,1:red,0.6039;green,0.1412;blue,0.0824}, fill opacity={1.0}, line width={0.75}, rotate={0}, solid}]
        table[row sep={\\}]
        {
            \\
            -10.0  -0.017220018681168726  \\
            -10.0  0.017550414449809076  \\
        }
        ;
    \addplot[color={rgb,1:red,0.6039;green,0.1412;blue,0.0824}, name path={15299133-55cf-4bd4-90e3-63e1debeb400}, draw opacity={1.0}, line width={1}, solid, mark={-}, mark size={1.5 pt}, mark repeat={1}, mark options={color={rgb,1:red,0.6039;green,0.1412;blue,0.0824}, draw opacity={1.0}, fill={rgb,1:red,0.6039;green,0.1412;blue,0.0824}, fill opacity={1.0}, line width={0.75}, rotate={0}, solid}]
        table[row sep={\\}]
        {
            \\
            -9.0  -0.020096598832633105  \\
            -9.0  0.01330351731631264  \\
        }
        ;
    \addplot[color={rgb,1:red,0.6039;green,0.1412;blue,0.0824}, name path={15299133-55cf-4bd4-90e3-63e1debeb400}, draw opacity={1.0}, line width={1}, solid, mark={-}, mark size={1.5 pt}, mark repeat={1}, mark options={color={rgb,1:red,0.6039;green,0.1412;blue,0.0824}, draw opacity={1.0}, fill={rgb,1:red,0.6039;green,0.1412;blue,0.0824}, fill opacity={1.0}, line width={0.75}, rotate={0}, solid}]
        table[row sep={\\}]
        {
            \\
            -8.0  -0.022925519667182762  \\
            -8.0  0.010474596481762983  \\
        }
        ;
    \addplot[color={rgb,1:red,0.6039;green,0.1412;blue,0.0824}, name path={15299133-55cf-4bd4-90e3-63e1debeb400}, draw opacity={1.0}, line width={1}, solid, mark={-}, mark size={1.5 pt}, mark repeat={1}, mark options={color={rgb,1:red,0.6039;green,0.1412;blue,0.0824}, draw opacity={1.0}, fill={rgb,1:red,0.6039;green,0.1412;blue,0.0824}, fill opacity={1.0}, line width={0.75}, rotate={0}, solid}]
        table[row sep={\\}]
        {
            \\
            -7.0  -0.01837539758798212  \\
            -7.0  0.015024718560963624  \\
        }
        ;
    \addplot[color={rgb,1:red,0.6039;green,0.1412;blue,0.0824}, name path={15299133-55cf-4bd4-90e3-63e1debeb400}, draw opacity={1.0}, line width={1}, solid, mark={-}, mark size={1.5 pt}, mark repeat={1}, mark options={color={rgb,1:red,0.6039;green,0.1412;blue,0.0824}, draw opacity={1.0}, fill={rgb,1:red,0.6039;green,0.1412;blue,0.0824}, fill opacity={1.0}, line width={0.75}, rotate={0}, solid}]
        table[row sep={\\}]
        {
            \\
            -6.0  -0.013578696784691747  \\
            -6.0  0.019821419364254  \\
        }
        ;
    \addplot[color={rgb,1:red,0.6039;green,0.1412;blue,0.0824}, name path={15299133-55cf-4bd4-90e3-63e1debeb400}, draw opacity={1.0}, line width={1}, solid, mark={-}, mark size={1.5 pt}, mark repeat={1}, mark options={color={rgb,1:red,0.6039;green,0.1412;blue,0.0824}, draw opacity={1.0}, fill={rgb,1:red,0.6039;green,0.1412;blue,0.0824}, fill opacity={1.0}, line width={0.75}, rotate={0}, solid}]
        table[row sep={\\}]
        {
            \\
            -5.0  -0.01987583038205497  \\
            -5.0  0.013524285766890774  \\
        }
        ;
    \addplot[color={rgb,1:red,0.6039;green,0.1412;blue,0.0824}, name path={15299133-55cf-4bd4-90e3-63e1debeb400}, draw opacity={1.0}, line width={1}, solid, mark={-}, mark size={1.5 pt}, mark repeat={1}, mark options={color={rgb,1:red,0.6039;green,0.1412;blue,0.0824}, draw opacity={1.0}, fill={rgb,1:red,0.6039;green,0.1412;blue,0.0824}, fill opacity={1.0}, line width={0.75}, rotate={0}, solid}]
        table[row sep={\\}]
        {
            \\
            -4.0  -0.00988994615169889  \\
            -4.0  0.023510169997246855  \\
        }
        ;
    \addplot[color={rgb,1:red,0.6039;green,0.1412;blue,0.0824}, name path={15299133-55cf-4bd4-90e3-63e1debeb400}, draw opacity={1.0}, line width={1}, solid, mark={-}, mark size={1.5 pt}, mark repeat={1}, mark options={color={rgb,1:red,0.6039;green,0.1412;blue,0.0824}, draw opacity={1.0}, fill={rgb,1:red,0.6039;green,0.1412;blue,0.0824}, fill opacity={1.0}, line width={0.75}, rotate={0}, solid}]
        table[row sep={\\}]
        {
            \\
            -3.0  -0.009180664961551462  \\
            -3.0  0.024219451187394284  \\
        }
        ;
    \addplot[color={rgb,1:red,0.6039;green,0.1412;blue,0.0824}, name path={15299133-55cf-4bd4-90e3-63e1debeb400}, draw opacity={1.0}, line width={1}, solid, mark={-}, mark size={1.5 pt}, mark repeat={1}, mark options={color={rgb,1:red,0.6039;green,0.1412;blue,0.0824}, draw opacity={1.0}, fill={rgb,1:red,0.6039;green,0.1412;blue,0.0824}, fill opacity={1.0}, line width={0.75}, rotate={0}, solid}]
        table[row sep={\\}]
        {
            \\
            -2.0  -0.011311452512218701  \\
            -2.0  0.022088663636727044  \\
        }
        ;
    \addplot[color={rgb,1:red,0.6039;green,0.1412;blue,0.0824}, name path={15299133-55cf-4bd4-90e3-63e1debeb400}, draw opacity={1.0}, line width={1}, solid, mark={-}, mark size={1.5 pt}, mark repeat={1}, mark options={color={rgb,1:red,0.6039;green,0.1412;blue,0.0824}, draw opacity={1.0}, fill={rgb,1:red,0.6039;green,0.1412;blue,0.0824}, fill opacity={1.0}, line width={0.75}, rotate={0}, solid}]
        table[row sep={\\}]
        {
            \\
            -1.0  -0.013145214856473316  \\
            -1.0  0.02025490129247243  \\
        }
        ;
    \addplot[color={rgb,1:red,0.8889;green,0.4356;blue,0.2781}, name path={882d324b-4762-4c2e-aaa0-d9e39cb39f84}, only marks, draw opacity={1.0}, line width={0}, solid, mark={*}, mark size={1.5 pt}, mark repeat={1}, mark options={color={rgb,1:red,0.6039;green,0.1412;blue,0.0824}, draw opacity={1.0}, fill={rgb,1:red,0.6039;green,0.1412;blue,0.0824}, fill opacity={1.0}, line width={0.75}, rotate={0}, solid}]
        table[row sep={\\}]
        {
            \\
            -22.0  -0.009209202508532917  \\
            -21.0  -0.01808829058126342  \\
            -20.0  0.0363689734304489  \\
            -19.0  0.00982489552976523  \\
            -18.0  0.02381043419744699  \\
            -17.0  0.016055067662330804  \\
            -16.0  0.01714675116073443  \\
            -15.0  -4.225236165781047e-5  \\
            -14.0  0.0015303245500947152  \\
            -13.0  -0.0026725540258475057  \\
            -12.0  0.0051203559496618145  \\
            -11.0  0.008059355045003153  \\
            -10.0  0.00016519788432017523  \\
            -9.0  -0.003396540758160233  \\
            -8.0  -0.006225461592709889  \\
            -7.0  -0.001675339513509248  \\
            -6.0  0.0031213612897811253  \\
            -5.0  -0.0031757723075820986  \\
            -4.0  0.006810111922773984  \\
            -3.0  0.00751939311292141  \\
            -2.0  0.0053886055622541715  \\
            -1.0  0.0035548432179995566  \\
        }
        ;
    \addplot[color={rgb,1:red,0.0627;green,0.4706;blue,0.5843}, name path={b2174313-6f84-48dd-9175-0857a2f5395e}, draw opacity={1.0}, line width={1}, solid, mark={-}, mark size={1.5 pt}, mark repeat={1}, mark options={color={rgb,1:red,0.0627;green,0.4706;blue,0.5843}, draw opacity={1.0}, fill={rgb,1:red,0.0627;green,0.4706;blue,0.5843}, fill opacity={1.0}, line width={0.75}, rotate={0}, solid}]
        table[row sep={\\}]
        {
            \\
            0.0  -0.013233046571421842  \\
            0.0  0.020167069577523903  \\
        }
        ;
    \addplot[color={rgb,1:red,0.0627;green,0.4706;blue,0.5843}, name path={b2174313-6f84-48dd-9175-0857a2f5395e}, draw opacity={1.0}, line width={1}, solid, mark={-}, mark size={1.5 pt}, mark repeat={1}, mark options={color={rgb,1:red,0.0627;green,0.4706;blue,0.5843}, draw opacity={1.0}, fill={rgb,1:red,0.0627;green,0.4706;blue,0.5843}, fill opacity={1.0}, line width={0.75}, rotate={0}, solid}]
        table[row sep={\\}]
        {
            \\
            1.0  -0.03195984432177887  \\
            1.0  0.0018400594661422912  \\
        }
        ;
    \addplot[color={rgb,1:red,0.0627;green,0.4706;blue,0.5843}, name path={b2174313-6f84-48dd-9175-0857a2f5395e}, draw opacity={1.0}, line width={1}, solid, mark={-}, mark size={1.5 pt}, mark repeat={1}, mark options={color={rgb,1:red,0.0627;green,0.4706;blue,0.5843}, draw opacity={1.0}, fill={rgb,1:red,0.0627;green,0.4706;blue,0.5843}, fill opacity={1.0}, line width={0.75}, rotate={0}, solid}]
        table[row sep={\\}]
        {
            \\
            2.0  -0.0439291878683475  \\
            2.0  -0.009674512079506526  \\
        }
        ;
    \addplot[color={rgb,1:red,0.0627;green,0.4706;blue,0.5843}, name path={b2174313-6f84-48dd-9175-0857a2f5395e}, draw opacity={1.0}, line width={1}, solid, mark={-}, mark size={1.5 pt}, mark repeat={1}, mark options={color={rgb,1:red,0.0627;green,0.4706;blue,0.5843}, draw opacity={1.0}, fill={rgb,1:red,0.0627;green,0.4706;blue,0.5843}, fill opacity={1.0}, line width={0.75}, rotate={0}, solid}]
        table[row sep={\\}]
        {
            \\
            3.0  -0.05545775446466068  \\
            3.0  -0.020666203546658  \\
        }
        ;
    \addplot[color={rgb,1:red,0.0627;green,0.4706;blue,0.5843}, name path={b2174313-6f84-48dd-9175-0857a2f5395e}, draw opacity={1.0}, line width={1}, solid, mark={-}, mark size={1.5 pt}, mark repeat={1}, mark options={color={rgb,1:red,0.0627;green,0.4706;blue,0.5843}, draw opacity={1.0}, fill={rgb,1:red,0.0627;green,0.4706;blue,0.5843}, fill opacity={1.0}, line width={0.75}, rotate={0}, solid}]
        table[row sep={\\}]
        {
            \\
            4.0  -0.06240543691073114  \\
            4.0  -0.027073108310282267  \\
        }
        ;
    \addplot[color={rgb,1:red,0.0627;green,0.4706;blue,0.5843}, name path={b2174313-6f84-48dd-9175-0857a2f5395e}, draw opacity={1.0}, line width={1}, solid, mark={-}, mark size={1.5 pt}, mark repeat={1}, mark options={color={rgb,1:red,0.0627;green,0.4706;blue,0.5843}, draw opacity={1.0}, fill={rgb,1:red,0.0627;green,0.4706;blue,0.5843}, fill opacity={1.0}, line width={0.75}, rotate={0}, solid}]
        table[row sep={\\}]
        {
            \\
            5.0  -0.07058166475633552  \\
            5.0  -0.03401413948967197  \\
        }
        ;
    \addplot[color={rgb,1:red,0.0627;green,0.4706;blue,0.5843}, name path={b2174313-6f84-48dd-9175-0857a2f5395e}, draw opacity={1.0}, line width={1}, solid, mark={-}, mark size={1.5 pt}, mark repeat={1}, mark options={color={rgb,1:red,0.0627;green,0.4706;blue,0.5843}, draw opacity={1.0}, fill={rgb,1:red,0.0627;green,0.4706;blue,0.5843}, fill opacity={1.0}, line width={0.75}, rotate={0}, solid}]
        table[row sep={\\}]
        {
            \\
            6.0  -0.06916932007200594  \\
            6.0  -0.03136361713677786  \\
        }
        ;
    \addplot[color={rgb,1:red,0.0627;green,0.4706;blue,0.5843}, name path={b2174313-6f84-48dd-9175-0857a2f5395e}, draw opacity={1.0}, line width={1}, solid, mark={-}, mark size={1.5 pt}, mark repeat={1}, mark options={color={rgb,1:red,0.0627;green,0.4706;blue,0.5843}, draw opacity={1.0}, fill={rgb,1:red,0.0627;green,0.4706;blue,0.5843}, fill opacity={1.0}, line width={0.75}, rotate={0}, solid}]
        table[row sep={\\}]
        {
            \\
            7.0  -0.06112997917075079  \\
            7.0  -0.020348751522400853  \\
        }
        ;
    \addplot[color={rgb,1:red,0.0627;green,0.4706;blue,0.5843}, name path={b2174313-6f84-48dd-9175-0857a2f5395e}, draw opacity={1.0}, line width={1}, solid, mark={-}, mark size={1.5 pt}, mark repeat={1}, mark options={color={rgb,1:red,0.0627;green,0.4706;blue,0.5843}, draw opacity={1.0}, fill={rgb,1:red,0.0627;green,0.4706;blue,0.5843}, fill opacity={1.0}, line width={0.75}, rotate={0}, solid}]
        table[row sep={\\}]
        {
            \\
            8.0  -0.07843564005550842  \\
            8.0  -0.03428233107548566  \\
        }
        ;
    \addplot[color={rgb,1:red,0.0627;green,0.4706;blue,0.5843}, name path={b2174313-6f84-48dd-9175-0857a2f5395e}, draw opacity={1.0}, line width={1}, solid, mark={-}, mark size={1.5 pt}, mark repeat={1}, mark options={color={rgb,1:red,0.0627;green,0.4706;blue,0.5843}, draw opacity={1.0}, fill={rgb,1:red,0.0627;green,0.4706;blue,0.5843}, fill opacity={1.0}, line width={0.75}, rotate={0}, solid}]
        table[row sep={\\}]
        {
            \\
            9.0  -0.08395954659794688  \\
            9.0  -0.03392935449309094  \\
        }
        ;
    \addplot[color={rgb,1:red,0.0627;green,0.4706;blue,0.5843}, name path={b2174313-6f84-48dd-9175-0857a2f5395e}, draw opacity={1.0}, line width={1}, solid, mark={-}, mark size={1.5 pt}, mark repeat={1}, mark options={color={rgb,1:red,0.0627;green,0.4706;blue,0.5843}, draw opacity={1.0}, fill={rgb,1:red,0.0627;green,0.4706;blue,0.5843}, fill opacity={1.0}, line width={0.75}, rotate={0}, solid}]
        table[row sep={\\}]
        {
            \\
            10.0  -0.09544340313736127  \\
            10.0  -0.035795567331350864  \\
        }
        ;
    \addplot[color={rgb,1:red,0.0627;green,0.4706;blue,0.5843}, name path={b2174313-6f84-48dd-9175-0857a2f5395e}, draw opacity={1.0}, line width={1}, solid, mark={-}, mark size={1.5 pt}, mark repeat={1}, mark options={color={rgb,1:red,0.0627;green,0.4706;blue,0.5843}, draw opacity={1.0}, fill={rgb,1:red,0.0627;green,0.4706;blue,0.5843}, fill opacity={1.0}, line width={0.75}, rotate={0}, solid}]
        table[row sep={\\}]
        {
            \\
            11.0  -0.16095314663709953  \\
            11.0  -0.09090031616593469  \\
        }
        ;
    \addplot[color={rgb,1:red,0.0627;green,0.4706;blue,0.5843}, name path={b2174313-6f84-48dd-9175-0857a2f5395e}, draw opacity={1.0}, line width={1}, solid, mark={-}, mark size={1.5 pt}, mark repeat={1}, mark options={color={rgb,1:red,0.0627;green,0.4706;blue,0.5843}, draw opacity={1.0}, fill={rgb,1:red,0.0627;green,0.4706;blue,0.5843}, fill opacity={1.0}, line width={0.75}, rotate={0}, solid}]
        table[row sep={\\}]
        {
            \\
            12.0  -0.2124007923239128  \\
            12.0  -0.12893554439152172  \\
        }
        ;
    \addplot[color={rgb,1:red,0.0627;green,0.4706;blue,0.5843}, name path={b2174313-6f84-48dd-9175-0857a2f5395e}, draw opacity={1.0}, line width={1}, solid, mark={-}, mark size={1.5 pt}, mark repeat={1}, mark options={color={rgb,1:red,0.0627;green,0.4706;blue,0.5843}, draw opacity={1.0}, fill={rgb,1:red,0.0627;green,0.4706;blue,0.5843}, fill opacity={1.0}, line width={0.75}, rotate={0}, solid}]
        table[row sep={\\}]
        {
            \\
            13.0  -0.19213241296321226  \\
            13.0  -0.07197538957627514  \\
        }
        ;
    \addplot[color={rgb,1:red,0.2422;green,0.6433;blue,0.3044}, name path={dd052b66-0b33-4b0d-8cf3-fd9ccb8643fb}, only marks, draw opacity={1.0}, line width={0}, solid, mark={*}, mark size={1.5 pt}, mark repeat={1}, mark options={color={rgb,1:red,0.0627;green,0.4706;blue,0.5843}, draw opacity={1.0}, fill={rgb,1:red,0.0627;green,0.4706;blue,0.5843}, fill opacity={1.0}, line width={0.75}, rotate={0}, solid}]
        table[row sep={\\}]
        {
            \\
            0.0  0.0034670115030510304  \\
            1.0  -0.015059892427818288  \\
            2.0  -0.02680184997392701  \\
            3.0  -0.038061979005659344  \\
            4.0  -0.044739272610506704  \\
            5.0  -0.052297902123003746  \\
            6.0  -0.0502664686043919  \\
            7.0  -0.04073936534657582  \\
            8.0  -0.05635898556549704  \\
            9.0  -0.05894445054551891  \\
            10.0  -0.06561948523435607  \\
            11.0  -0.1259267314015171  \\
            12.0  -0.17066816835771725  \\
            13.0  -0.1320539012697437  \\
        }
        ;
\end{axis}
\end{tikzpicture}

    \end{adjustbox}
\end{subfigure}
\end{figure}
\end{frame}

%%%%%%%%%%%%%%%%%%%%%%%%%%%%%%%%%%%%%%%%%%%%%%%%%%%%%%%%%%%%%%%%%%%%%%%



%%%%%%%%%%%%%%%%%%%%%%%%%%%%%%%%%%%%%%%%%%%%%%%%%%%%%%%%%%%%%%%%%%%%%%%



%%%%%%%%%%%%%%%%%%%%%%%%%%%%%%%%%%%%%%%%%%%%%%%%%%%%%%%%%%%%%%%%%%%%%%%

\begin{frame}{Conclusions}
    Provide general identification results for ATTs under linear factor models. 
    \begin{itemize}
        \item Let $T$ be fixed or grow.
        \item Can use multiple estimators for the factor space. 
        \begin{itemize}
            \item Brown et al. (2022) for CCE.
        \end{itemize}
        \item Results nest TWFE. 
        \begin{itemize}
            \item Give sufficient condition for consistency of TWFE: no systemic heterogeneity.
        \end{itemize}
    \end{itemize}

    Implement quasi-long-differencing estimator. 
    \begin{itemize}
        \item GMM estimator makes inference easy. 
        \item Can test basic features of the model. 
    \end{itemize}

\end{frame}


%%%%%%%%%%%%%%%%%%%%%%%%%%%%%%%%%%%%%%%%%%%%%%%%%%%%%%%%%%%%%%%%%%%%%%%

\begin{frame}{Proof of Lemma 2.3}
\label{Lemma 2.3 proof}
Let $\bm A^*$ be the $p \times p$ rotation that generates the Ahn et al. (2013) normalization for $\tilde{\bm F}$.

\begin{align*}
        \bm P (\tilde{\bm F}_{t > T_0}, \tilde{\bm F}_{t \leq T_0}) 
        &= \tilde{\bm F}_{t > T_0} (\tilde{\bm F}_{t \leq T_0}' \tilde{\bm F}_{t \leq T_0})^{-1} \tilde{\bm F}_{t \leq T_0}' \\
        &= \tilde{\bm F}_{t > T_0} \bm A^* (\bm A^{*'} \tilde{\bm F}_{t \leq T_0}' \tilde{\bm F}_{t \leq T_0} \bm A^*)^{-1} \bm A^{*'} \tilde{\bm F}_{t \leq T_0}' \\
        &= \tilde{\bm F}( \bm \theta)_{t > T_0} (\tilde{\bm F}(\bm \theta)_{t \leq T_0}' \tilde{\bm F}(\bm \theta)_{t \leq T_0})^{-1} \tilde{\bm F}(\bm \theta)_{t \leq T_0}' \\
        &= \bm P(\tilde{\bm F}( \bm \theta)_{t > T_0}, \tilde{\bm F}(\bm \theta)_{t \leq T_0})
    \end{align*}

    \hyperlink{Lemma 2.3 proof back}{\beamerbutton{Back.}}
\end{frame}

%%%%%%%%%%%%%%%%%%%%%%%%%%%%%%%%%%%%%%%%%%%%%%%%%%%%%%%%%%%%%%%%%%%%%%%

\begin{frame}{Proof of Theorem 2.1}
\label{Theorem 2.1 proof}

\begin{equation*}
    \condexpec{\tilde{y}_{it}}{D_i = 1} = \condexpec{\tilde{y}_{it}(1)}{D_i = 1} 
\end{equation*}

and

\begin{align*}
    &\condexpec{ \bm P(\tilde{\bm f}_{t}', \tilde{\bm F}_{t \leq T_0}) \tilde{\bm y}_{i,t \leq T_0} }{D_i = 1} \\
    =& \condexpec{ \tilde{\bm f}_{t}' (\tilde{\bm F}_{t \leq T_0}' \tilde{\bm F}_{t \leq T_0})^{-1} \tilde{\bm F}_{t \leq T_0}' \tilde{\bm y}_{i,t \leq T_0} }{D_i = 1} \\
    =& \condexpec{ \tilde{\bm f}_{t}' (\tilde{\bm F}_{t \leq T_0}' \tilde{\bm F}_{t \leq T_0})^{-1} \tilde{\bm F}_{t \leq T_0}' \big[ \tilde{\bm F}_{t \leq T_0} \tilde{\bm \gamma}_i  \big] }{D_i = 1} \\
    =& \condexpec{ \tilde{\bm f}_{t}' \tilde{\bm \gamma}_i}{D_i = 1} \\
    =& \condexpec{ \tilde{y}_{it}(0) }{D_i = 1} 
\end{align*}

\hyperlink{Theorem 2.1 proof back}{\beamerbutton{Back.}}
\end{frame}




\end{document}